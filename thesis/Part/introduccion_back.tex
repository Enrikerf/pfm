 
 En este proyecto se va a analizar cómo afrontar un proyecto de desarrollo de software desde el punto de vista de un ingeniero industrial. En mi experiencia profesional he podido ver cómo se aprecia como valor añadido la visión de nuestra profesión a la hora de abordar un proyecto de este tipo. Al fin y al cabo la gestión de cualquier proyecto encontrar puntos en común, optimizar, estandarizar y sacar el mayor partido a los recursos de los que se dispone es el núcleo de lo que se enseña en esta escuela.
 
 
Los principales problemas del desarrollo comercial de software no difieren tanto de lo que podría parecer de la gestión de proyectos de cualquier industria. Vamos a analizarlo Centrándonos en las bases más obvias de un proyecto: 

\begin{itemize}
 	\item desafío técnico
	\item gestión de tiempo/presupuesto
	\item adaptación al cambio
\end{itemize} 


El desafío técnico en la primera fase de un arte la encara cada artesano e innova en su técnica dando lugar a los productos y resultados dispares, los clientes tiene complicado tener una referencia para saber lo que compran.

Cuando los proyectos son más grandes el artesano empieza a ser un factor de riesgo, la mala elección del profesional sin estandarizar procesos puede malograr ingentes cantidades de recursos.

Quizá es en el último punto, la adaptación al cambio, donde la industria del software se diferencia un poco más del resto de industrias. el proyecto de software nunca termina. Es extremadamente volátil, caduco y cambiante.

En cada uno de estos puntos en la industria hay cantidad de métodos, herramientas, procesos y literatura que va encaminada a reducir la incertidumbre. La mayoría de ellos tienen más de 20 años y los que son más nuevos suelen ser métodos antiguos renombrados o reformulados. Aunque se han demostrado útiles no han penetrado en el software comercial a todos los niveles, muchas veces porque el cliente no sabe de su existencia o no está dispuesto a pagar por ello, a nadie se le ocurriría comprar un coche sin homologación que garantice que ha pasado por las pruebas de seguridad correspondiente, en el software se sigue vendiendo programas sin las más básicas pruebas de seguridad.

Es todavía común encontrar un cliente con un desconocimiento total a la hora de comprar software, hay pocas garantías y mucho dinero en juego. Hoy en día incluso las grandes consultoras tienen proyectos de software que a día de hoy son un fracaso a todos los niveles.

Las principales son:

\begin{itemize}
	\item como se va entregando valor de forma iterativa
		\subitem  scrum, kanban, agile, extreme programing 
	\item como se garantiza la expectativa del cliente con el producto
		\subitem Testing: TDD,BDD... diseño de casos de uso UML, UX, UI
	\item como se garantiza la posibilidad de cambio
		\subitem arquitectura de capas, hexagonal, DDD... 
\end{itemize} 

La creación de software ha vivido desde sus fases iniciales hasta hoy un proceso análogo al que hemos observado en cualquier industria. En una primera fase los artesanos, luego los gremios, estandarización producción en masa, optimización continua e innovacion.


el punto diferencial llega en la estandarización. En la cuestión de la gestión de los proyectos hay un despliege de medios en la industria que permiten al cliente comprar con garantias. En el software esto es complicado y este proceso está innmaduro. no son pocos los proyectos que no llegan a buen puerto, incluso los que disponen de un soporte financiero virtualmente ilimitado.


En el software, se ha hecho un esfuerzo constante en la estandarización de procesos para poder controlar la gestión de los proyectos

 En mi experiencia profesional me he encontrado con los problemas de la asincronia que php no resuelve directamente, hay proyectos de terceros y ahora ya se ha desarrollado, pero era interesante ver como lo resuelve golang y porque tiene tanta fama y si compensa el coste de transicion del lenguaje. Y porque no node? O 

Porque golang añade otra capa de que fuerza en tiempo de compilacion el estilo. Por ejemplo las interfaces porque tambien me he encontrado con muchos problemas de mantenimiento del codigo

El coste mas importante de un proyectos software siempre es el mismo: tiempo de lectura de codigo. Ademas es un coste que no añade valor. Por el que el cliente no va a pagar mas
Es por esto que hay que reducirlo al máximo.

Este proyecto analiza desde el punto de vista de un manager cuáles son los puntos de ventaja que da este lenguaje de programacion en su dia a dia

Esto no solo se consigue con el lenguaje de programación, al fin y al cabo es una herramienta y como tal hay muchas pàrecidas, al final la elección de una herramienta puede dar una ventaja en algún aspecto en particular pero no va a ser nunca el factor diferencial de cara al buen desempeño del proyecto.

El diseño y el estudio previo del proyecto si

Arquitectura de capas: hablar en terminos del negocio que se quiere manejar y separarlo de su implementacion real. Soy un firme defensor que la parte del codigo por la que realmente paga un cliente deberia ser capaz de leerla el mismo. Debe hacerse el esfuerzo de que la parte por la que se paga de un software sea legible por gente ajena al mundo del desarrollo de software

Problemas con el REST y el beneficio del RCP


Es todavía común encontrar un cliente con un desconocimiento total a la hora de comprar software, hay pocas garantías y mucho dinero en juego. Hoy en día incluso las grandes consultoras tienen proyectos de software que a día de hoy ni siquiera llegan a utilizarse. O grandes clientes que malogran muchos recursos sin ser capaces de entender a duras penas el resultado. Siendo la administración muchas veces un claro ejemplo de ello.