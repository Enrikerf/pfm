\documentclass[12pt,a4paper,spanish]{article}
\usepackage[margin=2.54cm,]{geometry}
\usepackage[pdftex]{graphicx}
\usepackage{pdfpages}
\usepackage{color}
\usepackage{listing}
\usepackage{listings}
\usepackage{dirtree}
\usepackage{float}
\usepackage{hyperref}
\usepackage{ulem}
\usepackage{xcolor}
\usepackage{listingsutf8}
\usepackage{titlesec}
\usepackage{etoolbox}
\usepackage{afterpage}
\usepackage[utf8]{inputenc}
\usepackage[toc,acronym]{glossaries}
\usepackage[noabbrev,capitalise,nameinlink]{cleveref}
\hypersetup{colorlinks={true},linkcolor={blue},citecolor=green}
\usepackage{caption}

\makenoidxglossaries
\newglossaryentry{API}{name={API},description={{(Application Programming Interface) Conjunto de reglas y protocolos de programación que especifican cómo los componentes de software deben interactuar entre sí sin tener que conocer los detalles subyacentes de implementación}}}
\newglossaryentry{REST}{name={REST},description={{(Representational State Transfer) es un estilo de arquitectura de software para sistemas distribuidos que se utiliza para diseñar servicios web que son fáciles de escalar, mantener y extender}}}
\newglossaryentry{RPC}{name={RPC},description={{(Remote Procedure Call) Protocolo de comunicación utilizado en sistemas distribuidos que permite que un programa solicite servicios o funciones a través de la red a otro programa en una máquina remota.}}}
\newglossaryentry{CRUD}{name={CRUD},description={{Acrónimo de las cuatro operaciones básicas en la gestión de datos: Crear (Create), Leer (Read), Actualizar (Update) y Eliminar (Delete)}}}
\newglossaryentry{Backend}{name={Backend},description={{Conjunto de servidores, aplicaciones y servicios que se ejecutan en segundo plano y proporcionan una interfaz de programación de aplicaciones (API) para que los clientes (front-end) puedan interactuar con ellos}}}
\newglossaryentry{Manager} {name={Manager},description={Servicio web o Backend diseñado en este TFM para la gestión de tareas}}
\newglossaryentry{Client} {name={Client},description={Servicio web o Backend diseñado en este TFM para la ejecucion de tareas albergados en sistemas remotos}}
\newglossaryentry{LayerArchitecture} {name={Layered Architecture},description={{(Arquitectura de capas) es un enfoque de diseño de software en el que los componentes del sistema se dividen en capas lógicas y se comunican entre sí a través de interfaces definidas}}}
\newglossaryentry{HexagonalArchitecture} {name={Hexagonal Architecture},description={{(Arquitectura hexagonal o arquitectura de puertos y adaptadores) Implementación concreta de la arquitectura de capas.
Basada en el diseño de un núcleo del sistema (Dominio) independiente de su entorno con el que se comunican los agentes externos a través de puertos y adaptadores}}}
\newglossaryentry{DDD} {name={DDD},description={(Domain-Driven Design o Diseño Dirigido por el Dominio) capa central en arquitectura de capas o hexagonal.
Entidades, objetos, conceptos y relaciones que conforman el núcleo del negocio.
Refleja y dar soporte al negocio que se está modelando.
Define un lenguaje común (ubiquitous language) que se comparte entre los expertos del dominio de la aplicación y los desarrolladores de software}}
\newglossaryentry{CQRS} {name={CQRS},description={{(Command Query Responsibility Segregation) Segregación de responsabilidades de comandos y consultas.
Es una técnica de diseño de arquitectura de software que separa la lógica de escritura (comandos) de la lógica de lectura (consultas) en sistemas de información}}}
\newglossaryentry{Entity} {name={Entity},description={(Entidad) Es un objeto de Dominio que tiene identidad única y representa un concepto del mundo real, del negocio.
Pueden ser almacenados, recuperados y modificados en la base de datos}}
\newglossaryentry{Value Object} {name={ValueObject},description={(Objeto de valor) Representa la descripción de un objeto, pero no tiene identidad propia.
Son inmutables.
Se utilizan para representar datos que no necesitan ser almacenados individualmente en base de datos}}
\newglossaryentry{Service} {name={Service},description={{(Servicio) Componente software que proporciona funcionalidades y operaciones que no están directamente relacionadas con ninguna entidad o valor específico.
Los servicios encapsulan la lógica de negocio compleja y se utilizan para ejecutar operaciones que involucran múltiples entidades o valores}}}
\newglossaryentry{Repository} {name={Repository},description={(Repositorio) Es un objeto que proporciona una interfaz para interactuar con la base de datos y abstrae la lógica de almacenamiento de datos del resto de la aplicación.
Los repositorios se utilizan para realizar operaciones de almacenamiento, recuperación, actualización y eliminación de entidades}}
\newglossaryentry{InfrastructureLayer} {name={Infrastructure Layer},description={(Capa de infraestructura) Proporciona una interfaz para interactuar con componentes como bases de datos, sistemas de archivos o servicios web.
Implementa detalles técnicos que dependen de una tecnología concreta y no deben afectar al modelado del negocio}}
\newglossaryentry{Adapter} {name={Adapter},description={{(Adaptador) Elemento de software que implementa una interfaz para conectar la capa de aplicación o dominio con los componentes de infraestructura}}}
\newglossaryentry{ApplicationLayer}{ name={Application layer},description={{(Capa de aplicación) Capa que contiene los casos de uso específicos del negocio.
Proporciona una interfaz para interactuar con el dominio}}}
\newglossaryentry{UseCase} {name={Use Case},description={{(Caso de uso) Escenario específico de uso de la aplicación que involucra múltiples operaciones y procesos.
Los casos de uso se utilizan para representar la funcionalidad de la aplicación desde la perspectiva del usuario}}}
\newglossaryentry{Command}{name={Comando},description={{(Comando) Operación que realiza un cambio en el estado del sistema}}}
\newglossaryentry{SUT}{name={SUT},description={{(System Under Test Sistema bajo testeo)}}}
\newglossaryentry{DOC}{name={DOC},description={{Depended On Component Dependiente del componente}}}
\newglossaryentry{DTO}{name={DTO},description={{(Data Transfer Object Objeto de transferencia de datos) Objeto inmutable que sólo sirve para transportar información de una capa a otra}}}
\newglossaryentry{UL}{name={UL},description={{(Ubiquitous Language) Lenguaje común en un diseño guiado por DDD para describir los componentes del problema}}}
\newglossaryentry{IDL}{name={IDL},description={{interface definition language}}}
\newglossaryentry{Query} {name={Command},description={{(Consulta) Operación que recupera información del sistema sin modificar su estado}}}
\newglossaryentry{E2E} {name={E2E},description={{(End to End o extremo a extremo) Tipología de test que prueba todo el sistema}}}
\newglossaryentry{CI/CD} {name={CI/CD},description={{(Continuous delivery/Continuous integration) Entrega Continua/Integración Continua.
Prácticas y herramientas para automatizar el proceso de garantizar el correcto desarrollo de software desde la creación del código hasta su despliegue en producción}}}

\begin{document}
    \titleformat{\paragraph}
{\normalfont\normalsize\bfseries}{\theparagraph}{1em}{}
\titlespacing*{\paragraph}
{0pt}{3.25ex plus 1ex minus .2ex}{1.5ex plus .2ex}
\setlength{\parskip}{\baselineskip}%
\setlength{\parindent}{0pt}%

\newcommand\blankpage{%
    \null
    \thispagestyle{empty}%
    \addtocounter{page}{-1}%
    \newpage}
\patchcmd{\thebibliography}{\section*{\refname}}{}{}{}
\titleformat{\subparagraph}
{\normalfont\normalsize\bfseries}{\thesubparagraph}{1em}{}
\titlespacing*{\subparagraph}
{0pt}{3.25ex plus 1ex minus .2ex}{1.5ex plus .2ex}
\setcounter{tocdepth}{5}
\setcounter{secnumdepth}{5}
\lstset{language=Go,
    basicstyle=\ttfamily\scriptsize,
    keywordstyle=\color{blue}\ttfamily,
    stringstyle=\color{red}\ttfamily,
    commentstyle=\color{green}\ttfamily}

\lstdefinelanguage{docker}{
    keywords={FROM, RUN, COPY, ADD, ENTRYPOINT, CMD,  ENV, ARG, WORKDIR, EXPOSE, LABEL, USER, VOLUME, STOPSIGNAL, ONBUILD, MAINTAINER},
    keywordstyle=\color{blue}\bfseries,
    identifierstyle=\color{black},
    sensitive=false,
    comment=[l]{\#},
    commentstyle=\color{purple}\ttfamily,
    stringstyle=\color{red}\ttfamily,
    morestring=[b]',
    morestring=[b]"
}

\lstdefinelanguage{docker-compose}{
    keywords={image, environment, ports, container_name, ports, volumes, links},
    keywordstyle=\color{blue}\bfseries,
    identifierstyle=\color{black},
    sensitive=false,
    comment=[l]{\#},
    commentstyle=\color{purple}\ttfamily,
    stringstyle=\color{red}\ttfamily,
    morestring=[b]',
    morestring=[b]"
}
\lstdefinelanguage{docker-compose-2}{
    keywords={version, volumes, services},
    keywordstyle=\color{blue}\bfseries,
    keywords=[2]{image, environment, ports, container_name, ports, links, build},
    keywordstyle=[2]\color{olive}\bfseries,
    identifierstyle=\color{black},
    sensitive=false,
    comment=[l]{\#},
    commentstyle=\color{purple}\ttfamily,
    stringstyle=\color{red}\ttfamily,
    morestring=[b]',
    morestring=[b]"
}

\lstset{basicstyle=\ttfamily,
    showstringspaces=false,
    commentstyle=\color{red},
    keywordstyle=\color{blue},
    inputencoding=utf8,
    extendedchars=true
}
    

\renewcommand{\contentsname}{Índice General}
\renewcommand{\listtablename}{Lista de Tablas}
\renewcommand{\tablename}{Tabla}


\begin{figure}[H]
        \hspace{-0.5cm}	
		\includegraphics[scale=0.25]{./1-Portada/img/logoUma}
		\hspace{6cm}
		\includegraphics[scale= 0.25]{./1-Portada/img/logoEscuela}\label{fig:figure3}
\end{figure}
\vspace*{0.2in}
\begin{center}
	\begin{large}
		\textbf {ESCUELA DE INGENIERÍAS INDUSTRIALES\\}
	\end{large}	
	\vspace*{0.5cm}
	\begin{large}
		\textbf {DEPARTAMENTO\\ de Ingeniería de Sistemas y Automática\\}		
	\end{large}	
	\vspace*{1cm}
	\begin{large}
		\textbf {ÁREA DE CONOCIMIENTO\\Ingeniería de Sistemas y Automática\\}
	\end{large}	
	\vspace*{1cm}	
	\begin{Huge}
		\textbf {PROYECTO FIN DE MÁSTER\\}
	\end{Huge}
	\vspace*{0.3cm}
	\begin{LARGE}
		\textbf {Desarrollo de plataforma de control remoto en Golang Go\\}
	\end{LARGE}
	\vspace*{0.3cm}
	\rule{5cm}{0.01cm}\\
	\vspace*{1cm}
	\begin{large}
		\textbf {Autor: Enrique Arrabal Almagro\\}
		\vspace*{0.5cm}
		\textbf {Tutor: ?? \\}
		\vspace*{1cm}
		\textbf{Titulación: Master en Ingeniería Industrial}
	\end{large}	
	\vspace*{3cm}
\begin{center}
\bf{MÁLAGA, Febrero de 2023}
\end{center}					
\end{center}
\thispagestyle{empty}


\newpage
\thispagestyle{empty}
\begin{flushright}
    \phantom{blank}
    \vspace{25mm}

    A mi madre, que me enseñó a soñar \\
    A mi padre, que me enseñó a pensar  \\
    A mi hermanos, en especial a Cris  \\
    A los que hice daño, por si sirve de algo  \\
    A los que me lo hicieron a mi, porque me enseñó a seguir  \\
    Pero sobre todo a tí, Carmina, que me faltas tanto  \\
\end{flushright}
\newpage
\textbf{Resumen}

Se ha diseñado y desarrollado un sistema capaz de gestionar y ejecutar tareas en servidores remotos.
Como ejemplo de aplicación remota se dispone de un programa de control PID para un motor de corriente continua.
Se ha obtenido un ejemplo práctico de aplicación de una metodología de desarrollo de software desde su diseño hasta su entrega.
Un diseño estructurado en capas, separando la solución tecnológica concreta de la lógica que resuelve el problema.
Se ha creado un lenguaje expresivo en el nombre y la estructura de todos los componentes del software transmitiendo el problema.
Dicho lenguaje facilita la comunicación efectiva del problema enfrentado a todo aquel que necesite interactuar con el sistema.
El diseño está enfocado en la gestión del principal problema de desarrollo del software: la evolución.
El sistema consta de tres programas independientes:

\begin{itemize}
    \item Un gestor de la interacción con los usuarios.
    \item Un cliente que responde a las ordenes de ejecución del gestor.
    Ejecutable en paralelo en tantas máquinas como se requiera.
    \item Un control PID para ser ejecutado por el programa cliente.
\end{itemize}

El trabajo evalúa Go ante un proyecto enfocado en la interacción con actuadores o hardware para el control automático.
Se han puesto a prueba los conceptos clave del lenguaje: asincronía y gestión de concurrencia y simplicidad.

Se ha realizado el ejercicio de adaptación de los documentos utilizados en un proyecto de ejecución de tipo industrial a un proyecto de software.
El diseño, en un documento de proyecto ejecutivo.
El desarrollo, en un documento de ejecución de obra.

\textbf{Palabras Clave}

Go,
API,
REST,
RPC,
CRUD,
control PID,
Arquitectura de capas,
Arquitectura hexagonal,
DDD,
CQRS,
lenguaje ubicuo,
CI/CD.

\newpage

\textbf{Summary}

A system capable of managing and executing tasks in remote systems has been designed and developed.
As an example of a remote application, the system also consists of a PID control program for a DC motor.
A practical example of the application of a software development methodology from design to delivery has been obtained.
The design is structured in layers, separating the concrete technological solution from the logic that solves the problem, facilitating the iterative evolution of the software.
An expressive language has been created to convey the problem solved.
This language facilitates effective communication among all project participants.
The design is focused on effectively handling the main problem of software development: evolution.
The resulting system consists of three independent programs:

\begin{itemize}
    \item A user interaction manager.
    \item A client that responds to the manager's execution commands.
    Executable in parallel on as many machines as required.
    \item A PID control to be executed by the client program.
\end{itemize}

The work evaluates Go to face a project focused on interaction with actuators or hardware for automatic control.
The key concepts of the language have been tested: asynchrony and concurrency management and simplicity.

An exercise has been carried out to adapt the documents used in an industrial execution project to a software project.
The design in an executive project document.
The development in a work execution document.

\textbf{Key words}

Go,
API,
REST,
RPC,
CRUD,
PID control,
LayerArchitecture,
HexagonalArchitecture,
DDD,
CQRS,
Ubiquitous Language,
CI/CD.

\includepdf[pages=-]{./part/DeclaracionOriginalidad.pdf}
\tableofcontents
\newpage
\cleardoublepage
%	\addcontentsline{lof}{chapter}{Lista de figuras} % para que aparezca en el indice de contenidos
\listoffigures % indice de figuras
%\cleardoublepage
%	\addcontentsline{lot}{chapter}{Lista de tablas} % para que aparezca en el indice de contenidos
\listoftables % indice de tablas

\lstlistoflistings

\newpage	
    \glsaddall
    \printnoidxglossary[style=tree,title=Glosario,nonumberlist]\label{sec:glossary}

    \newpage
    \section{Objetivos}\label{sec:objetivos}
    El objetivo general de este trabajo será el estudio de Golang Go para su aplicación en el desarrollo de servicios web API orientados al control remoto. Se va a desarrollar un servicio API gestor de tareas que ejecutará acciones en dispositivos remotos. En particular nos centraremos en el control PID de un motor de corriente continua.

El enfoque particular del PFM será adaptar el diseño de un Proyecto Ejecutivo Industrial a un proyecto software. Usando dicho Proyecto Ejecutivo para luego realizar un documento, también adaptado, de Dirección de obra. Terminaremos con una presentación de resultados y pruebas para confirmar la adecuación del resultado al objetivo fijado.

Los objetivos específicos:
\begin{itemize}
    \item Dominio de las sintaxis y características concretas del lenguaje para poder diseñar la arquitectura del software: gestión del protocolo HTTP, enrutamiento, inyección de dependencias, POO, gestión del SO para ejecución de comandos y desarrollo de los algoritmos de control.
    \item Desarrollo del proyecto de ejecución: con el conocimiento adquirido se diseñará el sistema que estará recogido en un proyecto de ejecución donde vendrá especificado el contexto del trabajo, su alcance, la arquitectura diseñada, planos de relación entre componentes y de la base de datos necesaria y hardware seleccionado junto con su justificación de uso.
    \begin{itemize}
        \item Las especificaciones de la API
        \begin{itemize}
            \item CRUD de direciones IP o DNS de las máquinas contenedoras del código de control
            \item CRUD de comandos a ejecutar en dichas máquinas
            \item CRUD de los resultados de dicha ejecución
            \item Diseño de los test unitarios
            \item Diseño de la base de datos para la persistencia de datos
        \end{itemize}
        \item Las especificaciones del código cliente
        \begin{itemize}
            \item Recepción de comandos a través del protocolo HTTP
            \item Ejecución de comandos en la máquina cliente
            \item Envío de resultados a través de llamadas a la API
            \item Diseño de los test unitarios
        \end{itemize}
        \item Especificaciones del programa de control
        \begin{itemize}
            \item Controlador PID en posición de un motor CC con encoder mediante PWM
        \end{itemize}
    \end{itemize}
    \item Documento de dirección de obras que contenga el seguimiento, la memoria explicativa de cambios si fuera necesario y los planos definitivos.
\end{itemize}

%Desde un punto de vista técnico, se llevará a cabo el desarrollo de un sistema capaz de almacenar y administrar tareas para su ejecución en sistemas remotos. Para lograr esto, se requerirá la implementación de un servicio web API (application programming interface) y dos servicios clientes: uno para manejar la interfaz gráfica (frontend) y otro para ejecutar las tareas.
%
%Con el fin de utilizar la herramienta seleccionada de manera más eficiente y explorar su uso en nuestro campo, la tarea a ejecutar en remoto será el control PID en velocidad y posición de un motor de corriente continua.
%
%El sistema contará con la capacidad de añadir, editar, eliminar y obtener las tareas almacenadas, así como los resultados de su ejecución. Además, reaccionará de forma asíncrona a los eventos de creación, edición y eliminación, permitiendo llevar a cabo procesos como la ejecución o detención de las tareas.
%
%Por su parte, el servicio cliente remoto tendrá la capacidad de recibir un comando, ejecutarlo y devolver el resultado correspondiente. El programa de control será capaz de administrar el control de un motor de corriente continua mediante un PID.
%
%En cuanto a la gestión del proyecto, se hará uso de herramientas profesionales para garantizar la calidad, estabilidad y progreso de la solución, evitando desconexiones entre el objetivo del cliente y su ejecución. Todo ello se llevará a cabo a través de un proyecto de ejecución, un documento de dirección de obra o ejecución.
%
%
%Aportacion de valor: estructurar un documento de proyecto de ingeniería con las metodología RUP se basa en una arquitectura de cuatro capas, cada una de las cuales representa una fase del proceso de desarrollo de software:
%
%Inicio: se enfoca en establecer los objetivos del proyecto, identificar los stakeholders y determinar la viabilidad del proyecto.
%Elaboración: se enfoca en la definición de los requisitos, la arquitectura y el plan de iteraciones del proyecto.
%Construcción: se enfoca en la implementación y pruebas del software.
%Transición: se enfoca en la preparación del software para su entrega al usuario final, incluyendo pruebas de aceptación, entrenamiento y documentación.


    \newpage
    \section{Resumen}\label{sec:resumen}
    
Se ha diseñado y desarrollado un sistema capaz de gestionar y ejecutar tareas en sistemas remotos.
Como ejemplo de aplicación remota, el sistema consta también de un programa de control PID para un motor de corriente continua.
Respondiendo al objetivo principal se ha obtenido un ejemplo práctico de aplicación de una metodología que abarca el proceso de desarrollo de software, desde su diseño hasta su entrega.
La metodología se basa en un diseño estructurado en capas, separando la solución tecnológica concreta de la lógica que resuelve el problema, facilitando la evolución iterativa del software.
Se ha realizado un esfuerzo de diseño para utilizar un lenguaje expresivo que transmite el problema resuelto.
Dicho lenguaje facilita la comunicación efectiva en todos los participantes del proyecto.
El sistema resultante consta de tres programas independientes y una plataforma hardware sobre la que actúan:

\begin{itemize}
    \item Un programa que gestiona la interacción entre los distintos sistemas y los usuarios.
    \item Un programa cliente que responde a las ordenes de ejecución de las tareas definidas en el gestor.
    Ejecutable en paralelo en tantas máquinas como se requiera.
    \item Un programa de control PID para ser ejecutado por el cliente.
    \item Un montaje sobre una placa de pruebas que consta de un puente H y un motor de corriente continua para ser controlado por el control PID\@.
\end{itemize}

El diseño está enfocado en manejar de forma efectiva el principal problema de desarrollo del software: la evolución.
La evolución y el cambio del software requiere de la interacción entre personas con diferentes conocimientos: técnicos, comerciales, financieros, clientes y todo aquel afectado.
Personas que comienzan a interactuar con el sistema sin conocimientos previos o gente que deja de interactuar con el mismo eliminando conocimiento del mismo.
También intervienen cambios tecnológicos que no deberían afectar al problema que resuelve.
Por ejemplo, un cambio en el sistema de comunicación, cambiar el uso del correo electrónico a mensajería instantánea no debe suponer un peligro o incertidumbre a la estabilidad de ejecución del sistema.

El presente trabajo evalúa cómo responde Golang ante este tipo de proyectos y metodologías.
En particular enfocado a soluciones que requieren de la interacción con actuadores o hardware para el control automático.
Se han puesto a prueba los conceptos clave que el lenguaje promueve: asincronía y gestión de concurrencia y simplicidad.

Se ha realizado el ejercicio de adaptación de los documentos utilizados en un proyecto de ejecución de tipo industrial a un proyecto de software para evaluar su encaje.
Se ha documentado el primer paso del proceso, el diseño, en un documento de proyecto ejecutivo.
Para el segundo paso, el desarrollo, se adapta un documento de ejecución de obra.




    \newpage
    \section{Proyecto Ejecutivo}\label{sec:proyecto_ejecutivo}
    \subsection{Memoria descriptiva}\label{subsec:memoria-descriptiva}
	En un proyecto de desarrollo de un producto mediante software, hay un diseño inicial, pero también hay un proceso iterativo para descubrir la solución final que se desea. En este punto no se asemeja a la construcción de un bien inmobiliario en el que hay unos requisitos una fase de diseño y de construcción. Después se puede seguir iterando, construyendo más, eliminando partes o sustituyendolas. Es por esto que la arquitectura con la que se desarrolla tiene que estar enfocada siempre a enfrentar ese proceso iterativo con las mayores garantías posibles de que el diseño del programa no entorpece dicho proceso iterativo.

Para exponer el primer diseño de una solución de forma que sea útil a la hora de tomar decisiones de cara a invertir recursos en un proyecto tiene que exponerse de forma que se entienda lo que se va a desarrollar y el valor que aporta. Para combinarlo con una metodología ágil que suponga la entrega de valor de forma iterativa, evitar la parálisis por análisis, no centrarse en el imposible de intentar describir detalles técnicos que han de ser investigados, analizados o puestos a prueba con su propia implementación. Se va ha realizar este proyecto ejecutivo a forma de primer entregable explicativo del diseño a implementar. El objetivo es encontrar un balance entre detalle explicativo que elimine incertidumbre en el diseño y pragmatísmo de no definir en exceso y encontrarse con mucha documentación que luego tiende a cambiar por el propio devenir del método iterativo.

\sout{En este aspecto El proyecto ejecutivo difiere En uno industrial en que hay que dejar cerrado lo que se puede dejar cerrado y señalar los puntos de incertidumbre claramente. Estimando el coste de solventar dicha incertidumbre. En ese aspecto el presupuesto marca hasta que punto esa incertidumbre se resuelve o se solventa. el resultado puede ser que afecte a una funcionalidad basica del proyecto, no se consiga solventar por falta de presupuesto pero siempre quede una funcionalidad entregable unicamente pendiente de resolver ese punto. Es decir el programa debe ser valido si eso se resuelve para no emplear recursos a fondo perdido}

\subsubsection{Información previa: antecedentes y condiciones}
    
Para poder empezar la descripción primero hay que presentar una explicación del vocabulario básico. Lo cierto es que acerca de DDD y arquitecturas de software hay ingente documentación. Realizar un exposición de toda la teoría sería extenso. Como referencia fundamentar para iniciarse en el tema tomamos como referencia la devopedia\cite{devopediaDDD}. Intentar resumirlo mejor carece de valor. Este trabajo lo que intenta no es ser un manual teórico más de DDD, si no una aplicación práctica de dichos conceptos para entender en un caso real el alcance de dicha teoría. Las referencias básicas que se van a tomar son:

\begin{itemize}
    \item Domain-Driven Design: Tackling Complexity in the Heart of Software. De Eric Evans\cite{EricEvans2003DDTC}
    \item Implementing Domain-Driven Design. De Vaughn Vernon\cite{VaughnVernon2013IDD}
    \item Get Your Hands Dirty on Clean Architecture de Tom Hombergs \cite{TomHombergs2019GYHD}
\end{itemize}

Toda teoría del DDD va enfocada a desarrollar un lenguaje común a todos los interesados que intervienen en un problema y su solución: clientes, vendedores, técnicos, financieros, etc. para desarrollar ese lenguaje se divide el problema en contextos delimintados. En un ejemplo rápido afiliado en un club deportivo significa facturas, números de identificación fiscal para los financieros y para la gente de operaciones significa reserva de pistas y cancelaciones. y para los de ventas significan descuentos, promociones. Ect. El tener un lenguaje común donde todos puedan expresarse y hablar de la misma solución es el reto de este proceso. Es lo que se define como \textbf{Ubiquitous Language} o \textbf{UL}

Intentar expresar en un mismo contexto todos esos significados termina en lo que se conoce como ¨Big Ball of Mud¨ o Gran bola de barro. Los componentes de este UL se pueden apreciar en la figura  \ref{fig:DomainDrivenDesignReference}, se puede apreciar también la relación entre ellos

\begin{figure}[H]
    \centering
    \includegraphics[height=0.5\textheight]{./part/Proyecto_ejecutivo/memoria_descriptiva/infoPreviaAntecedentes/img/DomainDrivenDesignReference}
    \caption{DomainDrivenDesignReference\cite{EricEvans2003DDTC}}\label{fig:DomainDrivenDesignReference}
\end{figure}

De todo este diagrama los conceptos en los que nos vamos a centrar

\begin{itemize}
    \item Entity: Un Objeto que tiene atributos pero principalmente definido por un identificador
    \item Value Object: Un Objeto que tiene atributos pero no identificador
    \item Domain Event: Un Objeto que define una suceso inducido por la interacción entre los componentes del dominio.
    \item Aggregate: Es un cluster de objetos tratado como una unidad. Las referencias o acciones externas sobre sus elementos siempre se hacen a través de un único elemento de este cluster conocido como Aggregate root. Tiene reglas definidas de consistencia dentro de su delimitación. External references are restricted to only one member, called the Aggregate Root.
    \item Repository: Es un mecanismo de interaccion para encapsular el acceso a tecnologías, normalmente de almacenamiento para interactuar con ellas, cuya implementación no concierne al dominio.
    \item Service: Es una funcionalidad de interacción con el dominio que garantiza la interacción con el mismo de una forma consistente
\end{itemize}

Esta definición de dominio se implementa normalmente con un paradigma de diseño conocido como arquitectura de capa. Lo que se busca es aislar esos contextos que definen nuestro dominio de implementaciones concretas ya sea para acceder al mismo o a las que accede el dominio. Por ejemplo, aislarlo de que se ejecute un servicio mediante una consola de comandos o desde una llamada http y que se guarde en una base de datos la información o se guarde en un archivo.

El tipico diagráma cuando se habla de arquitectura hexagonal es tal y como se muestra en la figura\ref{fig:hexagonalDiagram} Si bien consideramos que no tiene mucho sentido y de cara a la parte didáctica confunde, ya que el hexágono es una simple licencia estética. en el caso de existir más puertos de salida y entrada que los representados el hexágono pierde todo el sentido y cuando se enfrenta por primera vez este diagrama se tiende a intentar descifrar el sentído del hexagono.

\begin{figure}[H]
    \centering
    \includegraphics[height=0.3\textheight]{./part/Ejecucion/Seguimiento/CreateTaskUseCase/img/HexagonalDiagram}
    \caption{Hexagonal architecture diagram\cite{TomHombergs2019GYHD}}\label{fig:hexagonalDiagram}
\end{figure}

Vemos en este diagrama que el dominio, representado por la Entity no accede a elementos exteriores. La aplicación respresentada por los UseCase utiliza el dominio y depende de él. pero se aisla del exterior, la infraestructura obligando a utilizar interfaces a los elementos que acceden a el y obligando a implementar las interfaces definidas por la aplicación para la infraestructura que sirve para acceder al exterior de la aplicación.

En el diagrama \ref{fig:layers} podemos ver simplificado que el objetivo es que la dependencia de las capas, expresada por las flechas, sea siempre de fuera hacia adentro. Queremos preservar del cambio el interior y exponer al cambio el exterior. Separar lo propenso al cambio de lo que no. El UL de los detalles de implementación que tienen su propio lenguaje.

\begin{figure}[H]
    \centering
    \includegraphics[height=0.3\textheight]{./part/Proyecto_ejecutivo/memoria_descriptiva/infoPreviaAntecedentes/img/PFM - Layer}
    \caption{layered architecture}\label{fig:layers}
\end{figure}

Junto con el concepto de arquitectura hexagonal y el DDD vamos a aplicar el paradigma de diseño conocido como CQRS

CQRS significa Command Query Responsibility Segregation (Segregación de responsabilidades de comandos y consultas, en español). Es una técnica de diseño de arquitectura de software que separa la lógica de escritura (comandos) de la lógica de lectura (consultas) en sistemas de información.

La idea es que las operaciones de escritura (comandos) y las operaciones de lectura (consultas) se manejen por separado, ya que tienen necesidades y características distintas. Mientras que las operaciones de escritura son responsables de modificar el estado de la aplicación, las operaciones de lectura son responsables de devolver información sobre ese estado.

Al separar estas dos responsabilidades, se pueden optimizar las operaciones de lectura para que sean más rápidas y escalables. Además, se puede diseñar una arquitectura de software más flexible, permitiendo una mayor adaptabilidad y evolución del sistema a medida que cambian los requisitos de la aplicación.

Tenemos entonces que vamos a diseñar una arquitectura hexagonal con un enfoque DDD en el dominio y un enfoque CQRS en los casos de uso. Es decir se diseñará un dominio rico y con un UL y se accederá a su funcionalidad a través de casos de uso que sigan el criterio de ser comandos o queries.

Una vez definidas las tres capas y resumido el concepto. Un punto importante a comentar es el concepto de la estrategia de mapping que hay entre capas. Cada capa requiere sus objetos de trabajo para estar desacoplada de las demás. Pero como en todo aspecto está sometido a discusión acerca de seguir la teoría a rajatabla y el pragmatismo de no verse envuelto en redundancias y sobredimensionar las soluciones.

En un extracto del libro Get Your Hands Dirty on Clean Architecture\cite{TomHombergs2019GYHD} podemos leer:
"\textit{ The argument might have gone something like this:}

\begin{itemize}
    \item \textit{Pro-Mapping Developer:}
    \subitem  \textit{ If we don’t map between layers, we have to use the same model in both layers which means that the layers will be tightly coupled!}
    \item \textit{Contra-Mapping Developer:}
    \subitem \textit{ But if we do map between layers, we produce a lot of boilerplate code which is overkill for many use cases, since they’re only doing CRUD and have the same model across layers anyways!}
\end{itemize}
\textit{As is often the case in discussions like this, there’s truth to both sides of the argument. Let’s discuss some mapping strategies with their pros and cons and see if we can help those developers make a decision.}"

Hay tantas estrategias como atajos dentro de este paradigma queramos asumir. Los tipos de mapping que se documentan en este libro:

\begin{itemize}
    \item The NoMapping Strategy \ref{fig:nomapping}
    \item The Two-Way MappingStrategy \ref{fig:twowaymapping}
    \item The Full MappingStrategy \ref{fig:fullmapping}
    \item The One-Way MappingStrategy \ref{fig:onWaymapping}
\end{itemize}

En todos estos casos vemos simplificado el dominio a una única entidad Account y vemos como separa dicho dominio del acceso al proceso que envía dinero a una cuenta y de dónde se guarda la información del envío de ese dinero.

\begin{figure}[H]
    \centering
    \includegraphics[height=0.1\textheight]{./part/Ejecucion/Seguimiento/CreateTaskUseCase/img/nomapping}
    \caption{No mapping strategy \cite{TomHombergs2019GYHD}}\label{fig:nomapping}
\end{figure}

\begin{figure}[H]
    \centering
    \includegraphics[height=0.1\textheight]{./part/Ejecucion/Seguimiento/CreateTaskUseCase/img/twowaymapping}
    \caption{two way mapping strategy \cite{TomHombergs2019GYHD}}\label{fig:twowaymapping}
\end{figure}

\begin{figure}[H]
    \centering
    \includegraphics[height=0.1\textheight]{./part/Ejecucion/Seguimiento/CreateTaskUseCase/img/onWaymapping}
    \caption{One way mapping strategy \cite{TomHombergs2019GYHD}}\label{fig:onWaymapping}
\end{figure}

\begin{figure}[H]
    \centering
    \includegraphics[height=0.1\textheight]{./part/Ejecucion/Seguimiento/CreateTaskUseCase/img/fullmapping}
    \caption{Full mapping strategy \cite{TomHombergs2019GYHD}}\label{fig:fullmapping}
\end{figure}

Todo programa creado en este proyecto: gestor de tareas, cliente y controlador va a estar dividido por tanto en dichas capas y seguir los paradigmas de diseño aquí definidos. En cuanto a la estrategia de mapping tomaremos una decisión cuando nos enfrentemos al problema.
\subsubsection{Descripción del proyecto}
    \begin{figure}[H]
    \centering
    \includegraphics[height=0.4\textheight]{./part/Proyecto_ejecutivo/memoria_descriptiva/descripcionDelProyecto/manager/uml/systemConcept}
    \caption{Diagrama UML de despliegue del sistema}\label{fig:Diagrama UML de despliegue del sistema}
\end{figure}

Se puede ver un diagrama conceptual del sistema en la~\cref{fig:Diagrama UML de despliegue del sistema}. Los actores en este sistema diseñado serán:
\begin{itemize}
    \item el servidor que contenta el programa \gls{Manager} guardará las tareas, las ejecutará y guardará el resultado de dicha ejecución.
    \item los servidores que contengan una copia del programa \gls{Client} recepcionarán las llamadas del manager con el comando y las ejecutará devolviendo los resultados al Manager.
    \item los mismos servidores clientes contendrán una copia del programa a ejecutar, pudiendo ser tantos como se deseen, en nuestro caso implementaremos el programa de control PID par aun motor de corriente continua.
\end{itemize}

La memoria descriptiva de cada uno de los componentes estará compuesta de los siguientes elementos:

\begin{itemize}
    \item diagrama de los despliegue: interacción entre componentes del subsistema
    \item diagrama de objetos: diagrama de su diseño\textit{DDD}
    \item casos de uso: descripción y diagrama de actividad
    \item estructura de carpetas para la organización del código
\end{itemize}

\paragraph{Manager Server}
\input{./part/Proyecto_ejecutivo/memoria_descriptiva/descripcionDelProyecto/manager/manager.tex}

\paragraph{Client Program}
\input{./part/Proyecto_ejecutivo/memoria_descriptiva/descripcionDelProyecto/client/client.tex}

\paragraph{Control Program}
\input{./part/Proyecto_ejecutivo/memoria_descriptiva/descripcionDelProyecto/control/control.tex}

\subsubsection{Prestaciones}
    \paragraph{Funcionalidades del sistema}

El usuario el sistema tendrá la capacidad de:

\begin{itemize}
    \item Un programa gestor para introducir y gestionar tareas para ejecutar en sistemas remotos de forma automática o manual.
    \item Un programa cliente para instalar en cualquier servidor con la capacidad de recibir tareas desde el manager que se ejecutarán de forma automática.
    Cualquier programa instalado en dicho servidor podrá ser ejecutado.
    \item Un programa de control instalable en cualquiera de los servidores clientes para actuar sobre un pid que controle motor de corriente continua.
\end{itemize}

\paragraph{Garantías de seguridad y calidad}\label{par:testing}
    Conforme a los principios teóricos establecidos para el diseño de tests la cobertura de este proyecto se centrará en hacer tests unitarios para el dominio.
    En la~\cref{fig:piramidTest} se muestran las distribuciones de test que representan el ideal a la derecha y a el que tienen los sistemas a la izquierda.
    Las pruebas unitarias sólo son posibles cuando hay un buen diseño del software.
    En los proyectos que no disponen de un diseño que posibilite hacer test unitarios se ven abocados a hacer uso de tests que prueban demasiado código y características al mismo tiempo.
    Esto provoca que sean tests que tienen a fallar rápidamente y deben cambiarse habitualmente ante cualquier pequeño cambio.
    Las pruebas de más alto nivel incluyen incluso las interfaces de usuario.
    Debieran estar destinadas a funcionalidades muy establecidas y cuya tendencia al cambio sea muy escasa.

    \begin{figure}[H]
        \centering
        \includegraphics[height=0.35\textheight]{./part/Proyecto_ejecutivo/memoria_descriptiva/prestaciones/TestPiramid}
        \caption{Piramides de distribución de pruegbas: patrón y antipatrón}\label{fig:piramidTest}
    \end{figure}

\paragraph{Garantías de robustez ante nuevos desarrollo}
\input{part/Proyecto_ejecutivo/memoria_descriptiva/prestaciones/docker/docker}







\subsection{Memoria constructiva}\label{subsec:memoria-constructiva}
	\subsubsection{Golang}


\textbf{Golang}


Uno de los puntos centrales de este trabajo es exponer el estado del arte en el uso de Golang para el entorno industrial. En dicho estudio se ha consultado distintos papers y publicaciones para extraer primeras conclusiones:

Uno de los puntos más importantes para el desarrollo de software industrial es los recursos consumidos una de las primeras conclusines extraidas de la eficiencia del uso de golang interaccionando con mysql, una de las combinaciones más comerciales concluye que: "... combination of Go and MySQL is superior regarding CPU utilization and memory usage, while Node.js and MySQL combination is superior regarding response time~\cite{Effendy20211955}".

Un estudio más centrado en el uso de recursos ante problemas de algorítmia con altos requerimientos, en particular la implementacion de un árbol de decisiones. ~\cite{Dymora20201} Donde encontramos un punto importante: golang vuelve a no dar ventajas, pero tampoco inconvenientes en materia de tiempos de ejecucion. pero si que pierde en uso de cpu claramente y empata en materia de uso de memoria para mas de 500K registros en este problema en particular. Aunque se admite que la optimización de dicho mecanismo para este problema. lo cual es posible. "Thus, the Go language garbage collector
supports programmers by automatically releasing their programs’ memory when it is no longer needed.
However, tracking and cleaning the memory requires additional resources such as CPU time. The effect
of this can be seen in Figure 5. Of course, the scope of optimizing"~\cite{Dymora20201}

lo que si nos permite extraer es una conclusion y es que para usos intensivos golang es una opción viable pero no da ventaja en este aspecto.

Esto se debe al mecanismo que le da una ventaja tan notoria en el uso menos intensivo: el garbage collector que le requiere un uso adicional de memoria y cpu para ejecuciones con un gran número de registros. Concluye este estudio diciendo: ". Go can be an
attractive alternative in the area of DevOps tools. It is attractive to build something small–medium
that works natively without using a lot of RAM and which runs fast with many things needed for
this task in the language itself"~\cite{Dymora20201}

\begin{figure}[H]
	\centering
	\includegraphics[height=0.3\textheight]{./part/Proyecto_ejecutivo/memoria_constructiva/golang/img/memory_usage}
	\includegraphics[height=0.3\textheight]{./part/Proyecto_ejecutivo/memoria_constructiva/golang/img/cpuUsage}
	\includegraphics[height=0.3\textheight]{./part/Proyecto_ejecutivo/memoria_constructiva/golang/img/compTime}
	\caption[performance golang on algorithm]{performance golang on algorithm\cite{Dymora20201}}\label{fig:performance golang}
\end{figure}

El verdadero motivo para optar por golang como lenguaje para un software viene más de la mano de la facilidad de mantenimiendo, la rapidez de compilacion, el manejo fácil para la concurrencia y eficiente en el uso de recursos. Implementar mecanismos de memoria compartida para procesos concurrente es donde golang si optimiza recursos. Uno de los trabajadores de google que ha contribuido al ecosistema de go nos dice "Everyone knows and thinks about Google in terms of scale of users and scale of servers, but one thing that's not talked about as often is the scale of engineering effort."~\cite{Meyerson2014104+101}

Esto quiere decir que en la mayoría de las compañías el mayor coste no es el de infraestructura, si no el de ingeniería. Es un tradeoff muy importante el reducir el numero de horas dedicadas a mantener y desarrollar software que el hecho de optimizar el uso de máquinas. Si encima nos encontramos con un problema que no requiere de uso intensivo en cuestión algorítmica o tiempo de respuesta encontramos lo mejor de los dos: tenemos la ventaja de reducir los recuros necesarios de infraestructura y los de ingeniería. Si se diera el caso de que el software se enfrenta con el tiempo a problemas de escala siempre nos quedará la opción de aumentar las prestaciones de las máquinas utilizadas pero no tener problemas de aumento de costes de ingeniería

Es en la implementación de algoritmos que sacan ventaja de la concurrencia donde golang puede sacar ventaja respecto a otros lenguages como java o python ~\cite{Jenkins201714}

Estas son las principales conclusiones extraidas de la lectura de las publicaciones consultadas.
\cite{Effendy20211955}
\cite{Dymora20201}
\cite{Meyerson2014104+101}
\cite{Ray202110857}
\cite{Jenkins201714}
\cite{Ding2021321}
\cite{Taheri2021138}
\cite{NoAuthor2021179}
\cite{Dilley2019377}
\cite{Qiu2018}
\cite{Shoumik20181}
\cite{Mladenovic2018}
\cite{Benedict2017437}
\cite{Irawan2017}
\cite{Samaniego2017116}
\cite{Khaitan20152909}
\cite{Leokhin2015656}
\cite{Komendantskaya2014121}
\cite{Mittal2014292}
\cite{WhiteheadII2011209}
Nos permiten concluir que está justificado el uso en nuestro objetivo de diseñar un sistema de las carácteristicas de este proyecto con Golang.

Concluyo mencionando \cite{WhiteheadII2011209}"Despite the relatively young age of the programming language, we believe that Go helps
to fill an interesting niche in the field of programming languages. The unique feature-set and
aims of the language make it worth investigation for systems-level concurrent programming." Habiendo madurado el lenguaje unos años desde la publicación de este artículo encontramos que sigue siendo interesante, está altamente valorado en la comunidad.


\subsubsection{RPC vs REST}
\input{./part/Proyecto_ejecutivo/memoria_constructiva/rpc_vs_rest.text.tex}
\subsubsection{Docker}
\input{./part/Proyecto_ejecutivo/memoria_constructiva/docker.text.tex}
\subsubsection{Raspberry}
vs arduino, aunque aportaría más valor de cara a probar la compilación a una plataforma que suponga más reto que la raspberry resta mucho tiempo de cara a tiempos de desarrollo y no se le ve tanto interés
\subsubsection{Motor CC y puente H}
\subsubsection{Mysql}
\subsubsection{CI/CD}
%\subsection{Memoria económica}\label{subsec:memoria-economica}
%Uno de los puntos más complicados en el mundo del software es la estimación de costes, tanto de implementación como de mantenimiento. Descuadre entre espectativas-capital disponible-necesidad real. hay problemas por parte del cliente al tener presente el coste real de sus espectativas. hay costes ocultos de mantenimiento, de implementación rápida vs calidad, de malas decisiones por presion en los deadlines. Se quiere plantear esta sección como una estimación del orden de magnitud que puede adquirir el desarrollo de software

    \newpage
    \section{Ejecución}\label{sec:ejecucion}
    \subsection{Desarrollo y Seguimiento}\label{subsec:seguimiento}
    
En esta sección vamos a exponer la ejecución del diseño realizado. Haciendo foco en los casos de uso más representativos de toda la metodología que este trabajo ha desarrollado en su fase de diseño. Se expondrá de una forma gráfica con diagramas que hagan comprensible la estructura y no tanto los detalles de implementación. El repositorio con el código al completo lo podemos encontrar en \href{https://github.com/Enrikerf/pfm}{github}

Una de las primeras características que cabe resaltar de golang es que no tiene clases ni herencia. Como estructura principal de datos tiene los structs. Si queremos controlar la instanciación de los elementos de dominio, por ejemplo, la declaración inicial de variables de dichos structs, debemos implementar patrones más allá del elemento de datos en sí. Para poder garantizar que dichos elementos se creen bajo una misma lógica que no se pueda evitar manteniendo la consistencia. Otro punto importante a entender desde el inicio es que Golang sólo expone elementos, ya sea structs, funciones o demás tipos fuera de un módulo si están escritos en mayúscula. Por último, en Golang un módulo son todos los archivos dentro de un mismo directorio.

Entendido estos tres puntos, una forma de crear una estructura equivalente a una clase es definir una interfaz pública y crear un struct privado que la implemente. Los struct pueden implementar metodos pasandose a si mismos por referencia en una funcion. Podemos ver una implementación a modo de ejemplo en la~\cref{fig:golang class equivalent}. En la interfaz \textbf{C}lassName la “C“ es mayúscula exponiendolo al exterior del módulo y \textbf{c}lassName struct en minuscula encapsulándolo. Podemos verlo más claramente en el diagrama UML~\cref{fig: uml Diagram Class Equivalent in Golang}

Los motivos por los que es importante concebir esta estructura en lugar de utilizar structs directamente son:

\begin{itemize}
    \item Evita la instanciación no consistente, lo cual sólo se garantiza a traves del método exportado NewClass.
    \item Evita accesos no deseados a la modificación de variables. Siendo publicas seria posible. Garantiza la inmutabilidad en la que se basan los \textit{Value Objects}.
    \item Limpieza en gestión errores en Go. Ante la respuesta múltiple, que contemple devolver un resultado o error, evitamos instanciar el objeto con contenido vacío creando un elemento inconsistente sólo por exigencias del lenguaje. De esta forma admite retornar null.
\end{itemize}

\begin{figure}[H]
    \centering
    \includegraphics[height=0.5\textheight]{./part/Ejecucion/Seguimiento/classExample}
    \caption{Golang class equivalent}\label{fig:golang class equivalent}
\end{figure}

\begin{figure}[H]
    \centering
    \includegraphics[height=0.25\textheight]{./part/Proyecto_ejecutivo/memoria_constructiva/ClassEquivalentInGolang}
    \caption{UML diagram: Class equivalent in Golang}\label{fig: uml Diagram Class Equivalent in Golang}
\end{figure}

Como contrapartida tenemos que escribir más codigo. Poniendo como comparación una clase Java o C++ no podemos decir que en cantidad de lineas escritas se aumente o disminuya, aunque estéticamente pueda resultar incómodo. En cuestión de conceptos aprendidos en Golang sólamente trabaja con interfaces y structs contra el concepto de clase y herencia.

Una vez presentado estos conceptos pasamos a exponer el código. Los casos de uso más complejos y completos para ver todo el diseño aplicado son: CreateTaskUseCase y TaskEventHandler.

\subsubsection{CreateTaskUseCase}\label{subsubsec:CreateTaskUseCase}
    
Se toma como referencia la~\cref{fig:hexagonalDiagram}, el diagrama más extendido de ejemplo para una arquitectura hexagonal, para una explicación de alto nivel para la funcionalidad de crear tareas \textit{CreateTaskUseCase} que se aprecia en la~\cref{fig:CreateTaskHexagonalDiagram}.
Los componentes que intervienen son:
\begin{itemize}
    \item \textit{WebAdapter}: contiene la lógica relacionada con la recepción de llamadas RPC está ligada a la Infraestructura necesaria para ello, es decir importa las librerías para ello.
    A través de una interfaz se le ha de inyectar un puerto de entrada que le permita ejecutar la lógica del caso de uso.
    \item UseCase: contiene la lógica necesaria para garantizar la correcta llamada a la lógica de Dominio para la creación de una Tarea.
    Contiene las verificaciones, mediante la instanciación, entre los tipos primitivos y los objetos de Dominio.
    \item \textit{Creator}: servicio de creación de tareas.
    Contiene la lógica que obligatoriamente ha de ejecutarse de forma simultanea cuando se crea una tarea.
    La persistencia en la base de datos a través de un puerto de salida; una interfaz que se le inyecta y le permite ejecutar dicha lógica sin conocerla.
    También de emite un evento de Dominio comunicando la creación mediante el acceso a un puerto de salida.
    \item \textit{PersistenceAdapter}: contiene la lógica relacionada con la persistencia de datos.
    está ligado a la Infraestructura de la base de datos y contiene por tanto las importaciones de las librerías necesarias.
    \item \textit{DispatcherAdapter}: este puerto de salida contiene la lógica para gestionar los eventos emitidos por el Dominio.
    Para ello, necesita que le sean inyectados los casos de uso, eventHandler en este caso al no ser ejecutados directamente por un usuario.
    De esta forma, a la vez que un puerto de salida se convierte en un Adaptador de entrada que ejecuta puertos de entrada de nuevo a la Aplicación.
    \item \textit{Looper}: Contiene la lógica necesaria para obtener de base de datos, a través de un puerto de salida, todas las tareas sobre las que hay que iterar para ejecutarlas.
    \item \textit{Entity}: en este caso de uso serán las Tareas (\textit{Tasks})
\end{itemize}

Encontramos en la figura la numeración en orden de ejecución.
\begin{enumerate}
    \item se recibe una llamada RPC de tipo \textit{CreateTask} mediante el WebAdapter y es atendida
    \item A través de puerto de entrada se ejecuta el caso de uso
    \item El caso de uso instancia los elementos de Dominio necesarios y ejecuta el servicio \textit{Creator}
    \item El servicio \textit{Creator} comunica la información a persistir a través del puerto de salida
    \item La interfaz del el Adaptador de persistencia que ha de ser inyectada en el \textit{Creator}
    \item El componente inyectado en el \textit{Creator} que cumple con la interfaz 5 persiste la información en base de datos
    \item El servicio \textit{Creator} comunica el evento de creación a través del puerto de salida 7
    \item El servicio \textit{Dispatcher} inyectado en el creator que cumple con la interfaz 7 ejecuta el caso de uso correspondiente a ese evento a través de la interfaz
    \item La interfaz del caso de uso, en este caso \textit{eventHandler} de la creación de tareas
    \item El caso de uso que gestiona los eventos de creación de tareas.
    Cumple con la interfaz 9 y se encarga de comprobar que si el tipo de tarea creada requiere la activación del servicio de ejecución.
    \item El servicio de ejecución de tareas automáticas o \textit{Looper}.
    Se encarga de obtener a través de el puerto de salida 5, de la base de datos, y ejecutar todas las tareas definidas como automáticas.
\end{enumerate}

\begin{figure}[H]
    \centering
    \includegraphics[height=0.3\textheight]{./part/Ejecucion/Seguimiento/CreateTaskUseCase/img/CreateTaskHexagonalDiagram}
    \caption{Hexagonal architecture diagram}\label{fig:CreateTaskHexagonalDiagram}
\end{figure}

Se puede apreciar que el UL ayuda a la comprensión de los pasos y las funcionalidades.
El diseño ayuda a diferenciar la resolución del problema: representado por la creación de tareas y su ejecución.
Separando el porqué y dónde se ejecuta: se activa la resolución del problema por una llamada RPC, en una Base de datos y en un sistema remoto.
Si cualquiera de los elementos que acceden a el programa que resuelve el caso de uso quiere ser intercambiado, la única exigencia es implementar las interfaces requeridas por la capa de Aplicación.

La implementación concreta para componer la capa de Aplicación correspondiente a este caso de uso la encontramos en el~\cref{lst:bootstrapingExample}.
El orden de instanciación e inyección de cada uno de los componentes necesarios para el armado de la aplicación se vuelve un poco más confuso a la hora de llevarlo a la práctica, es por esto que los ejemplos gráficos son de gran ayuda.

\phantom{blank}
\vspace{10mm}
\hrule
\begin{lstlisting}[language=Go,caption={Ejemplo de \textit{Bootstraping} del sistema },breaklines=true,label={lst:bootstrapingExample}]

package Bootstrap

func Bootstrap() {
    //infra
    var persistenceAdapter := PersistenceAdapter.New(<infra dependencies>)
    //Domain
    var looper := Looper.New(persistenceAdapter)
    var createdEventHandler := CreatedEventHandler(looper)
    //Application
    var dispatcherAdapter := DispatcherAdapter.New(createdEventHandler)
    //Domain
    var creator := creator.New(dispatcherAdapter)
    //Application
    var useCase := UseCase.New(creator)
    //Infra
    var webAdapter := WebAdapter.New(<infra dependencies>,useCase)
}

\end{lstlisting}
\hrule

Para una explicación conceptual de la arquitectura como técnica de diseño los diagramas hexagonales pueden tener un valor didáctico, sin embargo, para un caso de uso más completo como el que desarrolla esta sección pierde el sentido.
No logra ser explicativo, ya que hay puertos de salida que se convierten en puertos de entrada como es el caso del Dispatcher, y al quitar lógica de negocio del caso de uso mediante servicios de Dominio que impidan el acceso directo a los repositorios de persistencia dicho diagrama se va quedando pequeño.
Es preferible diagramas UML completos como el desarrollado en la~\cref{fig:createTaskUseCaseArchitecture}.
En el diagrama~\cref{fig:createTaskUseCaseArchitectureFolderStructure} se muestra interacción de componentes atendiendo a su distribución en nuestro diseño organizativo del código en ficheros.

Como estrategia de mapeo entre capas o \textit{mapping} Go casi obliga al uso general de interfaces entre todos los elementos, incluidos las clases.
Al necesitar el equivalente de la implementación de clase para garantizar la cohesión ya se hace uso de  interfaces para todos los elementos.
En el desarrollo de la aplicación se ha optado por una estrategia \textit{Full Mapping} con adaptaciones.
Dentro de la estrategia \textit{Full Mapping} se debe implementar un modelo tanto de entrada, como ya se contempla, como de salida.
Es decir, la respuesta que hay entre cada capa también debe ser mapeada mediante un DTO, el patrón adaptado se muestra en la figura~\cref{fig:GetHandMapping}

Esto aumenta la burocracia y finalmente la opción implementada ha sido una adaptación personalizada que se muestra en la figura~\cref{fig:CreateTaskUseCaseMapping}.
En rojo se han marcado los atajos que se han tomado, es decir, el mapeo que no se ha implementado.
El objetivo es aislar bien de los puertos de entrada, es decir el GRPC, y no tanto de los puertos de salida;
ya que la persistencia está bien aislada mediante las interfaces y somos propietarios y consumidores internos de ellas.
Al ser los únicos consumidores de los puertos de salida se dispone de más libertad de cambio sin afectar a terceros.
Dentro de los adaptadores se hace el trabajo de mapeo entre las entidades de Dominio y los modelos de persistencia; traduciendo de nuevo a Dominio para responder.

\begin{figure}[H]
    \centering
    \includegraphics[angle=90,height=1\textheight]{./part/Ejecucion/Seguimiento/CreateTaskUseCase/img/createTaskUseCaseArchitecture}
    \caption{CreateTaskUseCase hexagonal architecture diagram}\label{fig:createTaskUseCaseArchitecture}
\end{figure}

\begin{figure}[H]
    \centering
    \includegraphics[height=0.5\textheight]{./part/Ejecucion/Seguimiento/CreateTaskUseCase/img/PFM - CreateUseCaseFolderStructure}
    \caption{CreateTaskUseCase folder Structure}\label{fig:createTaskUseCaseArchitectureFolderStructure}
\end{figure}

\begin{figure}[H]
    \centering
    \includegraphics[height=0.2\textheight]{./part/Ejecucion/Seguimiento/CreateTaskUseCase/img/PFM - GetHandMapping}
    \caption{\textit{Full Mapping} con DTO de salida y de entrada\cite{TomHombergs2019GYHD}}\label{fig:GetHandMapping}
\end{figure}

\begin{figure}[H]
    \centering
    \includegraphics[height=0.2\textheight]{./part/Ejecucion/Seguimiento/CreateTaskUseCase/img/PFM - FinalMapping}
    \caption{\textit{CreateTaskUseCase Mapping} o Mapeo}\label{fig:CreateTaskUseCaseMapping}
\end{figure}

El código concreto para este caso de uso más relevante se compone de:
\begin{itemize}
    \item TaskController~\cref{lst:TaskControler}
    \item CreateTaskUseCase~\cref{lst:CreateTaskUseCaseCode}
    \item Creator~\cref{lst:Creator}
    \item SavePort~\cref{lst:SavePort}
    \item PersistAdapter~\cref{lst:SaveAdapter}
    \item DispatcherPort~\cref{lst:Dispatcher}
\end{itemize}

En el controller del~\cref{lst:TaskControler} se hace uso de la request entrante y se compone el comando que corresponde para el caso de uso.
Mediante la interfaz inyectada en el controller en el \textit{bootstrapping} se ejecuta el caso de uso resaltado en rojo.
También es interesante resaltar que una vez ejecutado el caso de uso se utiliza la respuesta, que contiene el id de la nueva Task creada para devolverlo.
Sin embargo, debido la estrategia utilizada de \textit{mapping}, definida en el diagrama~\cref{fig:CreateTaskUseCaseMapping}, no hacemos uso directo de esta respuesta si no que componemos una nueva para el protocolo de comunicación que está utilizando el cliente, en este caso RPC.
En los imports de este componente se aprecia que sólo hay dependencias de la Infraestructura, gRPC en este caso, y de la capa de Aplicación haciendo uso del comando y la interfaz del caso de uso.

La implementación del caso de uso se muestra en el~\cref{lst:CreateTaskUseCaseCode}.
Los imports son una forma rápida de comprobar la independencia de capas, de que la arquitectura, y de que los límites se están respetando.
En este caso todos corresponden a Dominio, no tiene imports de librerías de terceros, es decir, Infraestructura.
La interfaz, utilizada en el controlador, \textit{UseCase} se encuentra escrita con la primera letra en mayúsculas, exponiéndola hacia afuera, mientras que el struct \textit{useCase} está en minúscula no pudiendo ser utilizado fuera del package.
El caso de uso realiza la instanciación de los \textit{ValueObjects} necesarios para la creación de una \textit{Task}.
Asegurando que no dan ningún error los datos de entrada y luego, haciendo uso del servicio se procede a crearla.
Aparece marcado en rojo el uso del servicio \textit{Creator} de Dominio.

En este caso se junta en un mismo fichero la interfaz y la implementación, al pertenecer a la misma capa.
No ocurre lo mismo dentro de \textit{Creator}~\cref{lst:Creator} Donde las interfaces de \textit{SaveRepository.Persist}~\cref{lst:SavePort} y \textit{Dispatcher.Dispatch}~\cref{lst:Dispatcher} se encuentra dentro del Dominio, pero la implementación\textit{SaveAdapter}~\cref{lst:SaveAdapter} se encuentra en la capa de Infraestructura.

El servicio de Dominio Creator se encarga de encapsular y atomizar los procesos que tienen que ocurrir en bloque siempre que se desee crear una Task.
Entre los cuales está: instanciar la tarea con los ValueObjects que recibe como parámetros;
Persistir la misma en la base de datos a través del puerto de salida, interfaz que implementará un adaptador;
y emitir el evento de creación a través del otro puerto de salida, el Dispatcher interfaz que implementará otro adaptador.
En la sección de \textit{imports} se aprecia que no hay elementos de Infraestructura o Aplicación.
Se limita al uso de Dominio.
Aislados del exterior.

\input{./part/Ejecucion/Seguimiento/CreateTaskUseCase/codes/CreateTaskController}
\input{./part/Ejecucion/Seguimiento/CreateTaskUseCase/codes/createTaskUseCaseCode}
\input{./part/Ejecucion/Seguimiento/CreateTaskUseCase/codes/Creator}
\input{./part/Ejecucion/Seguimiento/CreateTaskUseCase/codes/SavePort}
\input{./part/Ejecucion/Seguimiento/CreateTaskUseCase/codes/Dispatcher}
\input{./part/Ejecucion/Seguimiento/CreateTaskUseCase/codes/SaveAdapter}


\subsubsection{TaskEventHandler}
    

El adaptador de salida \textit{Dispatcher}~\cref{lst:DispatcherAdapter} es a su vez un elemento que actuará como adaptador de entrada.
Ejecutará el caso de uso \textit{TaskEventHandlerUseCase}~\cref{lst:TaskEventHandlerUseCase} cuando el evento lanzado sea de tipo .\textit{TaskCreated}
Podemos ver en~\cref{lst:DispatcherAdapter} que el gestor de eventos está preparado para añadir más eventos y gestionar otros casos de uso en consecuencia.

\textit{TaskEventHandlerUseCase}~\cref{lst:TaskEventHandlerUseCase} Habilitará el \textit{Looper}~\cref{lst:Looper} si la tarea creada es de tipo AUTOMATIC y el status es PENDING\@.
El servicio de Dominio \textit{Looper} buscará todas las tareas automáticas pendientes y las ejecutará de forma asíncrona.

En el servicio \textit{Looper}~\cref{lst:Looper}, señalado en rojo está el bloque de gestión de las operaciones asíncronas.
Por cada tarea se crea un hilo de ejecución paralelo.
Cuando todos los hilos terminan el comando \textit{wg.Wait()} desbloquea la ejecución del \textit{Looper}.

\input{./part/Ejecucion/Seguimiento/TaskEventHandler/codes/EventDispatcherAdapter}
\input{./part/Ejecucion/Seguimiento/TaskEventHandler/codes/TaskEventHandlerUseCase}
\input{./part/Ejecucion/Seguimiento/TaskEventHandler/codes/Looper}

\subsubsection{PID Control}\label{subsubsec:pidControl}
    
En el código correspondiente del algoritmo de control \textit{Control Algorithm}~\cref{lst:Looper} el aspecto más relevante se encuentra en el cálculo de la variable controlada.
Para ello Se hace uso del Encoder; dentro del Dominio del controlador automático se ha desarrollado un modelo que lo representa.
El \textit{Encoder}~\cref{lst:Encoder} es una interfaz que actúa como puerto de salida hacia la implementación.
Define función de observación que llamaremos \textit{watchdog} de las lectura de dichos pines y una función a la que llamaremos para obtener la posición en el momento que deseemos, \textit{getPosition}.
También atañe al Dominio saber que hay que restablecer los parámetros del encoder cuando se requiera y que hay que desactivar el hardware cuando se deje de utilizar para que no haya inconsistencias a la hora de volver a ejecutar el programa.
Para ello dispone de las otras funciones de la interfaz.

El adaptador\textit{Encoder Model}~\cref{lst:EncoderAdapter} hace uso de una librería en Go para Raspberry que permite la interacción con los pines de lectura con los que está conectado el encoder y realiza las operaciones pertinentes para obtener la posición.
El dispositivo particular en el que se va a ejecutar la lógica de control, el encoder físicamente conectado y todos los detalles finales del hardware no son más que un detalle de implementación secundario como lo es una base de datos.
El punto de más relevancia se encuentra enmarcado en rojo en el~\cref{lst:EncoderAdapter}.
Cuando se ejecuta la función de \textit{watchdog} se ponen en marcha dos \textit{Gorutines}; una para vigilar cada lectura de los dos pines del Encoder.

Se queda en un bucle infinito la primera sentencia de dicho bucle es quedarse bloqueado hasta que haya un cambio en el pin de lectura, ya sea de 0 a 1 o de 1 a 0.
Si esto ocurre lo primero que hace es bloquear todas las demás \textit{Gorutines}, porque va a hacer cambios en memoria de forma atómica, escribe la información de dicha lectura en un array de lecturas y desbloquea las \textit{Gorutines}, se termina y vuelve a esperar.
Independientemente, el watchdog, una vez lanzadas las \textit{Gorutines} de lectura se queda en bucle infinito que lo que hace es bloquear las \textit{Gorutines}.
Extraer un elemento del array de lecturas y procesarlo.

Tal y como están diseñadas la \textit{Gorutines}, se van encolando y esperan su momento en la CPU para ejecutarse.
Si hay bloqueos también esperan su turno.
De esta forma, aunque el procesamiento de una lectura fuera bloqueado por el sistema operativo, se irían encolando las \textit{Gorutines} a la espera de poder añadirse en el array de lecturas, pero no se perderían.
De esta forma se consigue leer exactamente los 360 pulsos por vuelta del encoder sin perder ninguno.

\input{./part/Ejecucion/Seguimiento/PidControl/code/pid}
\input{./part/Ejecucion/Seguimiento/PidControl/code/encoder}
\input{./part/Ejecucion/Seguimiento/PidControl/code/encoderAdapter}
\subsubsection{Testing}
    
Proteger el sistema frente al error humano es esencial para garantizar la entrega de código con seguridad y calidad. Para ello se emplearán herramientas de \gls{CI/CD} que se encargarán de realizar comprobaciones antes de permitir la llegada a servicio del nuevo código y el despliegue automático de dicho código en el entorno final. Este proyecto se centra en los primeros, CI o Continuous integration, debido a que no disponemos de servidores sobre los que desplegar de forma final.

En términos técnicos, CI/CD implica el uso de sistemas de control de versiones, servidores de automatización de compilación y pruebas, y herramientas de automatización de implementación y entrega. Estas herramientas permiten que los desarrolladores integren sus cambios de código en un repositorio central, donde se desencadenan automáticamente pruebas y compilaciones para detectar cualquier error y garantizar que el código se pueda implementar sin problemas.

Una vez que el código ha pasado todas las pruebas y ha sido aprobado por los revisores, el proceso de entrega y/o implementación se activa automáticamente. Esto implica la creación de un paquete de implementación, la realización de pruebas de aceptación automatizadas y el despliegue en producción. Este proceso ayuda a mejorar la eficiencia del equipo de desarrollo y a reducir los errores y riesgos en el proceso de entrega de software, permitiendo una implementación más rápida y frecuente de nuevas características y mejoras en el software.

Los tests son un apartado importante en la garantía de entrega de código libre de errores. En este proyecto se va a centrar el control de calidad en el dominio. Es decir, va a existir una suite de test unitarios por cada elemento de dominio que exista. Los test unitarios de aplicación e infraestructura quedan como un elemento deseable a tener, pero dependerá de los tiempos. los test de integración quedan fuera de esta primera versión del sistema. Las técnicas utilizadas para el desarrollo de los tests encuentran su origen en las siguentes condiciones de contorno:

\begin{itemize}
    \item si el código de pruebas no ahorra más esfuerzo de lo que cuesta desarrollarlo entonces se está perdiendo dinero.
    \item no se puede entregar un producto que no tiene ningúna clase de garantía de su correcto funcionamiento o acotado su margen de error.
\end{itemize}

Debemos trabajar dentro de esos dos límites: tener tests que garanticen y especifíquen el grado de calidad acotado al presupuesto que se dispone. Para poder expresar de una forma cuantitativa que los tests están correctamente diseñados y desarrollados primero definamos ineficaz e ineficiente. definición segun RAE:

\begin{itemize}
    \item eficaz: Capacidad de lograr el efecto que se desea o se espera.
    \item eficiente (2 accepción): Capacidad de lograr los resultados deseados con el mínimo posible de recursos
\end{itemize}

atendamos a la siguiente frase: si existiese un numero exacto de pruebas optimas, para cubrir todas la posibilidades con tests, si haces menos estas siendo ineficaz y si haces más ineficiente. Viendo estas definiciones queda claro que, cuando nos enfrentamos a la realidad, el trabajo de ingeniería constará en tomar una solución de compromiso entre eficacia y eficiencia. No podemos no probar y no tendremos recursos para probar todo.

En un ejemplo muy sencillo: para un método que recibe la información de un formulario de 8 campos con 10 posibles valores diferentes cada uno se obtendría un conjunto de \[ 8^{10} = 1,073,741,824 \] casos de prueba. Esta táctica de pruebas siendo la más obvia la mayor parte de las veces será inasumible.

El estado del arte y la técnica en diseño de test es amplia y variada. Este proyecto utilizará pruebas de caja negra. Las pruebas de caja negra se basan en diseñar un test sin suponer cómo va a estar desarrollado el código. Se diseña el caso de uso al que va a tener que enfretarse y el resultado esperado. Para evitar la prueba exhaustiva existen las siguientes técnicas:

\begin{itemize}
    \item variables independientes
    \subitem clases de equivalencia.
    \subitem metodo de valores límite
    \item variables dependientes
    \subitem vector de pares
\end{itemize}

\textbf{Variables independientes}

Su enfoque es reducir el conjunto de los valores posibles de cada entrada a un subconjunto representativo y reducido del conjunto original que asegure una cierta garantía de cobertura. Entonces la combinación de todos estos valores representativos de todas las entradas reducirá drásticamente el conjunto de casos de prueba final. Estas técnicas son Clases de Equivalencia y Análisis de Valores Límite; la segunda se aplica sobre la anterior.

Las clases de equivalencia son un subconjunto de valores de la entrada que el sistema debería manejar de forma equivalente en el ejercicio del \gls{SUT}(System Under Test). Para cada entrada deben repartirse todos sus posibles valores en un número finito de clases de equivalencia que cumplan la siguiente propiedad: la prueba de un valor representativo de una clase de equivalencia permite suponer “razonablemente“ que el resultado obtenido será un proceso similar que el obtenido probando cualquier otro valor de esa clase de equivalencia.

Es un proceso heurístico y se obtiene analizando:
\begin{itemize}
    \item la especificación (caja negra) que describe las características del proceso que debe cumplir la implementación.
    \item las salidas del SUT, que pueden aportar criterios para la partición en clases de equivalencia.
    \item las precondiciones del SUT, que aportan clases de equivalencia de error.
\end{itemize}

los pasos para diseñar el test son:
\begin{itemize}
    \item encontrar las Clases de Equivalencia de cada factor
    \item escoger un valor cualquiera de cada clase de equivalencia de cada factor
    \item si existen varios factores, generar todas las combinaciones de todos los valores anteriores de cada factor
    \item eliminar aquella combinaciones que no son factibles por la combinación de los valores de entrada
    \item añadir la salida correspondiente a cada combinación de valores de entrada
\end{itemize}

Para escoger dentro de una clase de equivalencia un valor se aplica la técnica de los valores límites. Esta técnica es aplicable a la partición de clases de equivalencia cuando sus valores tiene un orden total, o sea, que para toda pareja de valores distintos se puede comprobar si uno es mayor que otro o viceversa. Por ejemplo, enteros, reales, caracteres por su código, cadenas de caracteres por su orden lexicográfico, enumerados por su ordinal, fechas o horas. Son valores dentro de esa clase de equivalencia para el que cambia el comportamiento del SUT respecto del valor anterior.

La justificación se basa en la evidencia experimental de que “los errores se esconden en los rincones y se aglomeran en los límites” [Beizer] y por tanto se aumenta el conjunto de casos de prueba para mejorar la eficacia de encontrar errores. Es decir, que dentro de una clase de equivalencia puede requerirse el testeo de varios valores.

Vamos a exponer dos ejemplos que muestran la diferencia entre: parámetro de entrada y factor; y entre factor y clase de equivalencia.

Supongamos una función que acepta como parámetro de entrada un texto.

\begin{verbatim}
    funcion de nextDay(string day){...}
\end{verbatim}

Como estamos diseñando el test, no podemos saber la implementación. La entrada es un texto cualquiera. pero si quisieramos testear todos los textos posibles que pueden introducirse las pruebas necesarias serían infinitas. Aquí entra el concepto de factor. El factor es que para esa entrada existen 3 posibilidades: que la entrada sea invalida, o que sea válida. Para esos dos factores el conjunto que define la clases de equivalencia son

\begin{itemize}
    \item invalido: cualquier texto invalido
    \item válido: “lunes“,“martes“,“miércoles“,“jueves“,“viernes“,“sábado“ o “domingo“.
\end{itemize}

por lo tanto podríamos coger cualquier valor dentro de esas dos clases, por ejemplo:

\begin{itemize}
    \item invalido: “cualquierCadenaInvalida“ \(\longrightarrow\) resultado esperado: ERROR
    \item válido: “lunes“ \(\longrightarrow\) resultado esperado: “martes“
\end{itemize}

Vemos como podemos siguiendo esta metodología podemos dar una garantía. Explicar metodológicamente porqué se han hecho los tests y los valores escogidos y que covertura se está ofreciendo. Ahora si aplicamos la técnica de los valores límites sabemos que el “domingo“ es el último día y se debe volver al “lunes“. Es decir, es el valor límite dentro de nuestra clase de equivalencia. Por lo tanto si queremos garantizar y elegir con critero dentro de los casos de nuestra clase de equivalencia los tests quedarían:

\begin{itemize}
    \item invalido: “cualquierCadenaInvalida“ \(\longrightarrow\) resultado esperado: ERROR
    \item válido limite inicial: “lunes“ \(\longrightarrow\) resultado esperado: “martes“
    \item válido limite final: “domingo“ \(\longrightarrow\) resultado esperado: “lunes“
\end{itemize}

\textbf{Variables dependientes}

El método de pairwise, también conocido como pruebas de combinaciones de pares, es una técnica de diseño de pruebas que permite reducir el número de casos de prueba necesarios para lograr una cobertura exhaustiva de las combinaciones posibles de parámetros en un sistema o aplicación. En lugar de probar todas las combinaciones posibles de los parámetros, el método de pairwise identifica las combinaciones de pares que tienen el potencial de causar problemas o errores y las incluye en los casos de prueba. Estas combinaciones de pares se seleccionan utilizando un algoritmo que busca minimizar el número de casos de prueba necesarios sin sacrificar la calidad de la cobertura de pruebas.

El método de pairwise fue introducido por David C. Kuhn en un artículo titulado~\cite{37051} “Practical Combinatorial Testing“ publicado en 1997. Desde entonces, ha sido ampliamente utilizado en la industria de software y ha sido objeto de numerosos estudios y mejoras por parte de investigadores y practicantes.

Entre los beneficios del método de pairwise se encuentra la reducción del número de casos de prueba necesarios para una cobertura exhaustiva, lo que puede ahorrar tiempo y recursos. Sin embargo, es importante tener en cuenta que el método de pairwise no es una técnica infalible y que puede no detectar ciertos tipos de errores o problemas en el sistema o aplicación bajo prueba.

En conclusión, el método de pairwise es una técnica efectiva para la reducción del número de casos de prueba necesarios para lograr una cobertura exhaustiva de las combinaciones posibles de parámetros en un sistema o aplicación. Su aplicación puede mejorar la eficiencia y efectividad del proceso de pruebas, aunque debe ser utilizado en conjunto con otras técnicas de diseño de pruebas para lograr una cobertura completa.

un ejemplo extraido de~\cite{Bach04pairwisetesting} lo ejemplifica perfectamente. En un software con un menú que tenga doce botones para activar o desactivar tendremos 4096 casos diferentes que testear, para un software comercial, semejante coste de calidad es simplemente inasumible. y tal y como se menciona ¨Pairwise testing normally begins by selecting values for the system’s input variables. These individual values are often selected using domain partitioning. The values are then permuted to achieve coverage of all the pairings. This is very tedious to do by hand. Practical techniques used to create pairwise test sets include Orthogonal Arrays¨\cite{Bach04pairwisetesting} En este mismo paper reducen el caso tipico de menú de un programa de 12,288 a 10 casos con este método. Es una metodología para escoger los tests que más aportan valor de cara a la seguridad sin perder pragmatismo y economía. Para el cálculo de los pares hay varios software disponible como páginas web que permiten el cálculo de forma rápida.

Con estos métodos vamos a realizar pruebas que cumplan con las características de:

\begin{itemize}
    \item inocuas
    \item Automatizadas
    \item autoverificables
    \item repetibles
    \item independientes
    \item rápidas
    \item vector de pares
\end{itemize}

Con respecto al punto de inocuas significa que no introducen nuevos errores: para desarrollar el test tenemos que modificar el código no estamos siendo inocuos. Se requiere un diseño correcto y no es trivial. Un código se demuestra que es correcto cuando es fácil de testear de ahí que el enfoque TDD sea diseñar primero el test y luego el código, para conseguir esto se requiere de muchísima experiencia. En el desarrollo de software las veces que se puede hacer un TDD puro son escasas y por lo tanto por lo menos tenemos que tener claro que si a la hora de diseñar un test se complica el aislar la característica a testear, definir los factores y las clases de equivalencia, o nos salen un número inabarcable de tests estamos ante un código mejorable.



\subsection{Final de obra}\label{subsec:final-de-obra}
    
\paragraph{Pipeline}\label{par:pipeline}

En el proceso de Pull Request, tal y como se diseño~\nameref{par:testing}, se ejecuta el pipeline diseñado para garantizar la entrega de software testeado y con el estandar de calidad requerido. Podemos ver en la figura~\cref{fig:githubActions} que los pasos son:

\begin{itemize}
    \item ejecutar los tests
    \item pasar el chequeo de estandar del código
    \item compilar en modo producción
\end{itemize}

faltarían el CD o continous delivery que ejecutaría los pasos de subir dicho ejecutable a un servidor, al menos para el programa manager.

\begin{figure}[H]
    \centering
    \includegraphics[height=0.2\textheight]{./part/Ejecucion/Seguimiento/PuestaAPunto/img/githubPipelines}
    \caption{Pipeline en Github}\label{fig:githubActions}
\end{figure}

\paragraph{Montaje de prueba}\label{par:montaje}

Para la prueba conjunta del sistema se ha desplegado un montaje como se ve en la figura~\cref{fig:Control-Diagrama UML de despliegue}

\begin{figure}[H]
    \centering
    \includegraphics[height=0.2\textheight]{./part/Ejecucion/Seguimiento/PuestaAPunto/img/deploy}
    \caption{Diagrama de despliegue de prueba}\label{fig:despliegue de prueba}
\end{figure}

La imagen~\cref{fig:montaje en protoboard} muestra el montaje físico de conexión entre el el servidor cliente. La Raspberry se conecta en una protoboard donde se realiza la conexión con el puente H y el motor de corriente continua.

\begin{figure}[H]
    \centering
    \includegraphics[height=0.2\textheight]{./part/Ejecucion/Seguimiento/PuestaAPunto/img/montajeProtoboard}
    \caption{Montaje en protoboard}\label{fig:montaje en protoboard}
\end{figure}

\paragraph{Interfaz gráfica}\label{par:interfaz}

Para controlar el programa manager se ha desarrollado como añadido una interfaz gráfica. En la figura~\cref{fig:frontend} podemos ver el listado de las tareas creadas, el estado en el que se encuentran y un boton para ejecutar manualmente las que son manuales. Para ejecutar manualmente una tarea disponimos de una gráfica para ver en tiempo real la evolución de la variable controlada tal y como podemos ver en el~\cref{fig:runner}

\begin{figure}[H]
    \centering
    \includegraphics[height=0.2\textheight]{./part/Ejecucion/Seguimiento/PuestaAPunto/img/frontend}
    \caption{Frontend: task list}\label{fig:frontend}
\end{figure}

\begin{figure}[H]
    \centering
    \includegraphics[height=0.2\textheight]{./part/Ejecucion/Seguimiento/PuestaAPunto/img/runner}
    \caption{Frontend: runner}\label{fig:runner}
\end{figure}
\subsection{Memoria explicativa de cambios}\label{subsec:memoria explicativa de cambios}
    
\subsubsection{Problemas GRPC en golang e interfaces de usuario}\label{subsec:problemas-rpc-en-golang}
Se quiso hacer un frontal para goland. Aquí nos enfrentamos al problema del estado del arte en dos puntos.
El primero es que para golang el realizar interfaces web de usuario habría supuesto mucha más investigación de la proyectada y lo segundo que se encuentra en un estado muy experimental

La opción adoptada era el uso de typescript con un framework como react que nos facilitara la implementación de una interfaz de forma rápida y eficiente, pero teníamos el segundo problema del estado del arte, GRPC en web todavía no es compatible para todos los métodos de comunicación~\cref{GRPCcompatibilidadConNavegadores}

En la ejecución manual de tareas se definió el caso de uso para su uso en la interfaz de usuario. Ver en tiempo real la gráfica de los resultados obtenidos durante la ejecución de la tarea. En este punto se pensaba explotar la opción del stream bidireccional para poder enviar los steps a requerimiento del usuario y obtener la respuestas en paralelo.

Al no poder hacer uso del stream bidireccional se ha tenido que diseñar un flujo alternativo.

el flujo original~\cref{ref:X} donde tenemos el TaskLoop~\cref{fig:Use Case-TaskLoop} y el task Processor~\cref{fig:Use Case-Task Processor} han tenido que ser modificados para poder hacer uso desde el frontend de este flujo. El looper se ha dejado igual pero se ha desarrollado un procesador para las tareas manuales particular.

Los efectos que ha tenido esto es:
\begin{itemize}
    \item las tareas bidireccionales tienen que tener como maximo 2 steps: una de comienzo y otra de fin, pueden no tener de fin
    \item se ha requerido de que el flujo bidireccional conste de un proceso en el que se envía el primer step y se queda en un loop infinito a la escucha que sólo terminará cuando mediante una acción paralela alguien cambien manualmente, mediante la edición de la Task a través de la llamada correspondiente, el status a DONE.
\end{itemize}

y los nuevos flujos diseñados los podemos apreciar en~\cref{fig:1-ExecuteTaskManuallyInteractionV2}~\cref{fig:1-TaskProcessorV2}

\begin{figure}[H]
    \centering
    \includegraphics[height=0.3\textheight]{./part/Ejecucion/Seguimiento/MemoriaExplicativaDeCambios/img/1-ExecuteTaskManuallyInteraction}
    \caption{Use Case: 1-ExecuteTaskManuallyInteraction V2}\label{fig:1-ExecuteTaskManuallyInteractionV2}
\end{figure}

\begin{figure}[H]
    \centering
    \includegraphics[height=0.55\textheight]{./part/Ejecucion/Seguimiento/MemoriaExplicativaDeCambios/img/1-TaskProcessor}
    \caption{Use Case: 1-TaskProcessor V2}\label{fig:1-TaskProcessorV2}
\end{figure}

\subsubsection{Caso práctico de las ventajas del DDD y arquitectura de capas}

En el proceso de desarrollo se evaluó que el diseño original de separar el programa a ejecutar del cliente requería de una nueva implementación de intercomunicación que añadía complejidad y tiempo. Al ser un proyecto amplio y querer tocar todo el proceso de desarrollo no se ha visto interesante esa complejidad añadida. Por lo tanto se introdujo la lógica del programa de control dentro del programa cliente. Aquí se pudo ver la utilidad de la arquitectura al no encontrar ninguna fricción en este proceso. Dos programas con el mismo diseño estructural y organización para unirse o separarse sólo requiere mover archivos y no tocar código en sí.

El dominio se une sin tocarse el uno al otro ya que son idependientes todos los elementos. En la aplicación son casos de uso nuevos que tampoco interacciónan unos con otros. y la infraestructura lo mismo. Donde mejor se puede apreciar es en la estructura de carpetas de la figura

En el proyecto constará de 4 carpetas principales

\begin{figure}[H]
    \setlength{\DTbaselineskip}{10pt}
    \DTsetlength{0.2em}{1em}{0.2em}{0.4pt}{1.6pt}
    \dirtree{%
        .1 Project .
        .2 Domain.
        .2 Application.
        .2 Adapter.
        .2 Bootstrap.
    }
    \caption{Client: Estructura de carpetas de Proyecto}\label{fig:Client- Estructura de carpetas de Proyecto}
\end{figure}

\textbf{Dominio}

\begin{figure}[H]
    \setlength{\DTbaselineskip}{10pt}
    \DTsetlength{0.2em}{1em}{0.2em}{0.4pt}{1.6pt}
    \dirtree{%
        .1 Domain.
        .2 Step.
        .3 StepVo.
        .3 ResultVo.
        .3 Repository.
        .4 consoleWrite.
        .3 Services.
        .4 UnaryExecutor.
        .4 ClientStreamExecutor.
        .4 ServerStreamExecutor.
        .4 BidirectionalExecutor.
    }
    \caption{Client: Estructura de carpetas de Dominio}\label{fig:Client-Estructura de carpetas de Dominio}
\end{figure}

\textbf{Aplicación}

\begin{figure}[H]
    \setlength{\DTbaselineskip}{10pt}
    \DTsetlength{0.2em}{1em}{0.2em}{0.4pt}{1.6pt}
    \dirtree{%
        .1 Application.
        .2 Port.
        .3 in.
        .4 Step.
        .5 ExecuteUnary.
        .6 Command.
        .6 UseCase.
        .5 ExecuteServerStream (Command).
        .5 ExecuteClientStream (Command).
        .5 ExecuteBidi (Command).
    }
    \caption{Client: Estructura de carpetas de Aplicación}\label{fig:Client-Estructura de carpetas de Aplicación}
\end{figure}

\textbf{Adapters}

\begin{figure}[H]
    \setlength{\DTbaselineskip}{10pt}
    \DTsetlength{0.2em}{1em}{0.2em}{0.4pt}{1.6pt}
    \dirtree{%
        .1 Adapter.
        .2 in.
        .3 GRPC.
        .4 Harán uso de los useCases de aplicación cuando llegue una request RPC.
        .2 out.
        .3 console.
        .4 implementacion de consoleWrite para ejecutar los comandos en el sistema y obtener los resultados.
    }
    \caption{Client: Estructura de carpetas de Infraestructura}\label{fig:Client-Estructura de carpetas de Infraestructura}
\end{figure}



    \newpage
    \section{Conclusiones}\label{sec:conclusiones}
    
Golang es un lenguaje enfocado en programas con una envergadura muy acotada o scripting. Es sencillo y con una curva de aprendizaje aceptable para los objetivos básicos que abarcan este tipo de programas. Permitiendo extraer todo el partido a la asincronía, la limpieza y la sencillez. Tiene paquetes base muy potentes que permiten abordar pequeñas tareas con muchas garantías sin necesidad de añadir dependencias de terceros. Pequeños endpoints, scripts de procesamiento, algoritmia, paralelismo. Lo cual lo vuelve un candidato muy viable para lo que se suele utilizar python.

Sin embargo, casi todos los proyectos grandes empiezan siendo pequeños. Los programas que son utilizados evolucionan. Se van añadiendo funcionalidades y terminan necesitando estructura y diseño para soportar dicho crecimiento, que es lo que ha puesto a prueba principalmente este proyecto. Si bien se puede argumentar que esto no debiera ocurrir, que es un error de diseño no modularizar y permitir crecer un proyecto hasta ese punto, la realidad es que estos errores ocurren. Introducir este tipo de arquitecturas protege ante esta tipología de errores y permite dividir el proyecto de forma sencilla cuando se hace evidente que se requiere. No siempre es tan evidente hasta que puede llegar a ser demasiado tarde.

En estos casos, el desarrollador que provenga de otros lenguaje más maduros y con sintaxis más amplias, puede llegar a sentir que Golang tiende a necesitar sobreingeniería al abordar una estructura lo suficientemente granulada. Se necesita de estructuras más elaboradas: el equivalente a la clase, el manejo de errores, la no existencia de excepciones y el enfoque de las interfaces, por ejemplo. Es cierto que brillan por su sencillez y no dejan tanto espacio a la inventiva; que si bien puede dar lugar a verdaderos ingenios en buenas manos, suele terminar en el desorden y complejidad. Un ejemplo muy claro es la inexistencia de la herencia, foco de tantos problemas en otros lenguajes.

\textbf{Nomenclatura}

Es cierto que la guia de estilo tiene una filosofía muy bien enfocada y lo suficientemente amplia como para dar libertad. Sin embargo, lo que se aprecia en la comunidad es que, precisamente por ese enfoque a scripting, se tiende a elegir variables muy cortas, poco explicativas. Tienen sus argumentos para justificarlo: realizar diseños pequeños, muy modulares, donde la distancia entre la declaración de la variable y su uso no sobrepase una distancia suficiente como para necesitar nombres elaborados. Esto choca de lleno, con la tendencia de abusar a poner mucho código en un mismo fichero y evitar la división. Si sumamos el nombrado de variables que abusan de acrónimos, palabras incompletas y caracteres sueltos, a abundante código en un mismo fichero se abre la puerta a un código dificil de leer. Se pierde de vista la cohesión.

A pesar de los argumentos favorables a la nomenclatura que proponen, no deja de recordar a los comienzos de php, javascript o demás lenguajes de scripting. Es ampliamente conocido cómo acaban estas tendencias: ficheros de muchas lineas, variables imcomprensibles y código difícil de leer. Refactorizar un nombre largo y explicativo a uno corto siempre será mucho más sencillo que extraer el significado de una variable nombrada con una letra e intentar darle un nombre mejor. Siempre será más fácil juntar código en un mismo fichero que separarlo. Lo segundo obliga a pensar en su cohesión y sus dependencias.

Por lo tanto en este proyecto, que además pretendía ser didáctico, se ha optado por el uso de nombres altamente explicativos. Sin temor a caer en la redundancia, el abuso. También se a optado por dividir cada componente en su propio archivo, algo nada normativo en este lenguaje.

\textbf{Interfaces}

La teoría apoya la decisión de eliminar la palabra reservada \textit{implement} indicando al compilador que una estructura va a implementar una interfaz. El argumento a favor más potente que tiene Golang es evitar la necesidad de implementar \textit{wrappers} de librerias de terceros a las que no quieres acoplarte. Diseñar una interfaz que cumpla esa librería que quieres utilizar ya te aisla de ella. De no ser así, al ser de terceros, habría de modificar la librería e implementarla explicitamente; lo cual es absurdo. Por lo que se desarrolla una clase que implemententa nuestra interfaz y hace uso de cada función de la librería; es decir un \textit{wrapper}. Lo cual aumenta la cantidad de código y la complejidad.

También cierto que si la signatura de una función se repite debiera ser la misma interfaz y no escribir dos veces la misma. Sin embargo, en una arquitectura de capas esto ocurre; ya sea por mal diseño o por que es inevitable en algunos casos. No poder elegir en la implementación de la interfaz con qué paquete está dicho acoplamiento provoca que pueda ser confuso. las interfaces de desarrolo, IDEs, suelen ayudarte y advertirte de qué funciones debe implementar. Además de que si buscas cuántas funciones implementan tu interfaz aparecerán muchas que no tienen, en el fondo, ninguna relación. Son perdidas en la limpieza y capacidad de lectura.

No suele provocar tanto tempor el exceso de código burocrático, como son los wrappers, como a la dificultad de lectura o falta de orden. No disponer de la declaración explícita de la interfaz favorece el desorden. No siendo un punto crítico, es sin embargo la característica que menos amigable de Golang con respecto a la sintáxis.

\textbf{Equivalente de clases}

Es un punto que incomoda y desconcierta al principio. Que se quiera evitar la herencia no parece justificar de la palabra reservada para crear una clase. Tener que crear más código a través de la creación de una interfaz que oculte la estructura implementada se antoja innecesario. Una vez aceptado es cierto que es quizá lo más cómodo de Golang. Poder devolver \textit{nil} como resultado y no una estructura vacía, lo cual puede ser peligroso, casa muy bien con su forma de manejar errores. Además de que obligar a firmar un contrato de cómo actua la clase ayuda a modularizar por defecto. Se puede considerar sobreingeniería, pero no añade demasiados inconvenientes a cambio. Mi conclusión es que no utilizaría structs sueltos en casi ningún punto. A lo mejor muy encapsulados en una función que enfrente una algoritmia complicada y requiera estructurar datos de una forma rápida de una forma totalmente aislada del exterior.

\textbf{Paquetes y división del código en golang}

En la comunidad se tiende a crear paquetes con muchos archivos, ya que crear directorios para ordenar el código en Golang implica modularizar. Se echa en falta alguna herramienta para ordenar el código a nivel estructura de carpetas sin afectar al desarrollo; poder escoger y acotar la visibilidad de los elementos públicos dentro de un submodulo. Cuando se aborda una arquitectura DDD se puede llegar a complicar al no tener la libertad de paquetizar según entidades lógicas acotadas. Obligando a tener en cuenta la visibilidad y ciclicidad de dependencias. El contrapunto es que Golang está pensado para ello, te obliga a seguir el orden establecido.

\textbf{Testing}

El estandar es disponer de los test unitarios al lado de la clase relacionada. Esto provoca imports ajenos en el dominio. El codigo para crear los dobles en medio de tu codigo de producción. Dificulta la navegación por la estructura de carpetas al tener demasiados archivos ajenos a la estructura del programa. La limpieza y el naming también se aplican a los archivos y organización de ficheros, este estándar lo dificulta. Puede venir bien para gestión de equipos que vengan de scripting porque sentirán mayor orden, pero me aventuro a asegurar que es un punto difícil de aceptar para casi todos los profesionales que vengan de otros lenguajes.

\textbf{Gestión del proyecto}

La arquitectura y el diseño que se han empleado busca que el propio código sea la documentación. Un entregable es requisito imprecintible para un tfm, pero en la profesión, el mantenimiento de esta documentación se vuelve muchas veces algo extremadamente costoso de mantener. Sin embargo, existe un un minimo de documentación imprescindible para los profesionales no técnicos que intervienen en el proyecto, ahí es donde se ha visto el encaje de adaptar un proyecto de tipo industrial. Ofrecer una forma de introducirse y mantenerse informado del UL y entender los casos de uso disponibles para interactuar con el sistema.

Hay mucha teoría al respecto de la gestión de proyectos y documentación del software, pero se ha querido explorar la opción de partir de un estándar ya definido y bien depurado como es la presentacion de proyectos de carácter industrial. La conclusión es que lo más importante en la gestión es estandarizar procesos. De esta forma, cuando los profesionales entran y salgen de proyectos sólo requieren aprender el proyecto en sí, no todo el sistema de gestión. Determinar una estructura estándar de documentación y entrega es el motivo de las normas existentes en la industria y la construcción. Son fruto y razón por la que se ha llegado hasta dónde se encuentran el día de hoy. Un proceso de aprendizaje de cientos de años para la construcción y la industria. El desarrollo de software, ya sea con las mismas normas o distintas habrá de llegar a ese punto.


\begin{itemize}
    \item crear un sistema para que las tareas automaticas se puedan ejecutar en un tiempo requerido y configurado. Esto es una tarea básica, pero crear un sistema de temporización parecido al CRONTAB de linux se escapaba del ámbito del proyecto. Además el sistema de ejecución automática asíncrona y gestión por eventos deja muy localizada y de fácil acceso esta iteración.
    \item la ejecución manual de las tareas de forma bidireccional cuando los navegadores lo permitan o investigar alternativas
    \item separación del programa cliente del programa de control
    \item creación de programas de control variados que pongan a prueba el sistema
    \item test de flujo de carga
    \item mejora del hardware, evitar el motaje provisional en protoboard
    \item implementar el concepto de context nativo de golang para el control de timeouts de las gorutines.
\end{itemize}


    \newpage
    \section{References}\label{sec:references}
    \bibliographystyle{plain}
    \bibliography{/Users/enrikerf/workspace/thesis/out/refs}
\end{document}