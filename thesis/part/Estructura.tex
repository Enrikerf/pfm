La estructura del presente documento va a incluir todo el proceso que se siguen en la ejecución de un proyecto, desde los requisitos hasta la entrega. Por lo tanto se ha buscado seguir lo más fielmente posible los documentos según norma para los proyectos industriales.

El núcleo del trabajo se compondrá por tanto de los siguientes puntos:

\begin{itemize}
    \item Proyecto de ejecución
    \begin{itemize}
        \item memoria descriptiva
        \item memoria constructiva
        \item pliego de condiciones
        \item \sout{mediciones}
        \item \sout{presupuesto}
    \end{itemize}
    \item Dirección de obras
    \begin{itemize}
        \item seguimiento del proyecto
        \item memoria explicativa de cambios
        \item planos definitivos
    \end{itemize}
\end{itemize}

vemos que hemos adaptado la cantidad de puntos de los que se compone según norma un proyecto de ejecución debido a que algunos no aplican a un proyecto de este tipo. Al seguir una metodología ágil, el proyecto ejecutivo busca una definición lo suficientemente detallada como para captar y estimar el alcance y la aportación de valor de cada componente del proyecto, pero hay aspectos técnicos que se van a descubrir durante la ejecución del mismo; esto se recogerá en el documento de dirección de obras.

Después se añadiran los puntos adicionales que ha de contemplar un trabajo de fin de master de estas características:
\begin{itemize}
    \item Aplicaciones del sistema: pruebas y resultados.
    \item Conclusiones: se expondrán las distintas conclusiones que se han extraido con respecto a los puntos objetivo. Golang, protocolo de comunicación, diseño de software y testing
    \item Lineas futuras

\end{itemize}

\sout{Latencias!, Control de luces, Control del motor???}