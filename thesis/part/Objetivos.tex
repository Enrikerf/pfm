Desde el punto de vista técnico se va a crear un sistema capaz de albergar y gestionar tareas para su ejecución en remoto.

Para ellos se requerirá un servicio web API (application programming interface) y dos servicios clientes uno para manejarlo el frontend y otro para ejecutar las tareas

Para hacer uso en mayor medidas de la herramienta escogida y para explorar su uso en nuestro campo la tarea a ejecutar en remoto sera un control PID en velocidad y posición de un motor de corriente continua.

Tendremos la capacidad de:

\begin{itemize}
	\item Añadir editar, eliminar y obtener las tareas contenidas en el sistema
	\item Añadir editar, eliminar y obtener los resultados de la ejecución de dichas tareas
	\item Añadir editar, eliminar y obtener las direcciones de los equipos remotos donde se ejecutarán las tareas
\end{itemize} 

El servicio cliente remoto tendrá la capacidad de recibir un comando, ejecutarlo y devolver el resultado.

Desde un punto de vista de gestión del proyecto: se va a hacer uso de las herramientas que se usan en un ámbito profesional para la gestión del mismo dar garantía de calidad, estabilidad, iteración de la solución aportando valor progresivamente evitando desconexión entre el objetivo por parte del cliente y el que ejecuta.