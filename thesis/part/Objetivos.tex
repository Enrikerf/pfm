Desde un punto de vista técnico, se llevará a cabo el desarrollo de un sistema capaz de almacenar y administrar tareas para su ejecución en sistemas remotos. Para lograr esto, se requerirá la implementación de un servicio web API (application programming interface) y dos servicios clientes: uno para manejar la interfaz gráfica (frontend) y otro para ejecutar las tareas.

Con el fin de utilizar la herramienta seleccionada de manera más eficiente y explorar su uso en nuestro campo, la tarea a ejecutar en remoto será el control PID en velocidad y posición de un motor de corriente continua.

El sistema contará con la capacidad de añadir, editar, eliminar y obtener las tareas almacenadas, así como los resultados de su ejecución. Además, reaccionará de forma asíncrona a los eventos de creación, edición y eliminación, permitiendo llevar a cabo procesos como la ejecución o detención de las tareas.

Por su parte, el servicio cliente remoto tendrá la capacidad de recibir un comando, ejecutarlo y devolver el resultado correspondiente. El programa de control será capaz de administrar el control de un motor de corriente continua mediante un PID.

En cuanto a la gestión del proyecto, se hará uso de herramientas profesionales para garantizar la calidad, estabilidad y progreso de la solución, evitando desconexiones entre el objetivo del cliente y su ejecución. Todo ello se llevará a cabo a través de un proyecto de ejecución, un documento de dirección de obra o ejecución.


Aportacion de valor: estructurar un documento de proyecto de ingeniería con las metodología RUP se basa en una arquitectura de cuatro capas, cada una de las cuales representa una fase del proceso de desarrollo de software:

Inicio: se enfoca en establecer los objetivos del proyecto, identificar los stakeholders y determinar la viabilidad del proyecto.
Elaboración: se enfoca en la definición de los requisitos, la arquitectura y el plan de iteraciones del proyecto.
Construcción: se enfoca en la implementación y pruebas del software.
Transición: se enfoca en la preparación del software para su entrega al usuario final, incluyendo pruebas de aceptación, entrenamiento y documentación.
