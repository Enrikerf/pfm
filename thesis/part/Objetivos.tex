
El objetivo principal es el desarrollo de un sistema que aborde todos los puntos de complejidad en la creación de software para la evaluación de Golang como lenguaje de programación en su interacción con los paradigmas de diseño y procesos más habituales en el software comercial.
Los puntos a tratar son: el diseño, la creación, garantía de calidad en la entrega, el mantenimiento y la evolución iterativa.
El diseño debe garantizar que, en todas la etapas posteriores, la comunicación entre las personas que intervengan en su uso y desarrollo se puedan realizar de forma efectiva y rentable.
Como sistema se propone un servicio \gls{API} (Application Programming Interface) que permita la interacción con los usuarios con un programa gestor de tareas.
El gestor de tareas enviará las consignas a el programa cliente instalado en los dispositivos remotos.
Como caso de aplicación se propone un programa de control PID de un motor de corriente continua que pueda ser controlado por el programa cliente.

El enfoque particular del trabajo propone adaptar la estructura de un proyecto ejecutivo de carácter industrial a un proyecto de diseño de software.
Para la documentación del proceso de ejecución se va a adaptar un documento de dirección de obra.

Para lograr estos objetivos generales se consideran los siguientes objetivos específicos:
\begin{itemize}
    \item Describir los fundamentos teóricos de los paradigmas de diseño utilizados así como las especificaciones del sistema: arquitectura, planos de relación entre componentes, estructura de la base de datos, garantía de calidad mediante tests automáticos, hardware utilizado y justificación de uso.
    \item El programa gestor permitirá la gestión básica de la información (\gls{CRUD} Create, Read, Update, Delete) a través del protocolo de comunicación \gls{RPC} (Remote Procedure Call) con la información persistible:
    \begin{itemize}
        \item Direcciones de las máquinas remotas.
        \item Comandos a ejecutar.
        \item Resultados de ejecución.
    \end{itemize}
    \item El programa cliente permitirá:
    \begin{itemize}
        \item La recepción de comandos a través del protocolo RPC\@.
        \item Ejecución de comandos en la máquina cliente.
        \item Envío de resultados.
    \end{itemize}
    \item El programa de control PID permitirá el control en velocidad de un motor CC
    \item Se realizará un estudio de la idoneidad de Golang para el desarrollo de sistemas con las características requeridas.
    \item Se realizará un estudio de la idoneidad de RPC como protocolo de comunicación.
\end{itemize}
