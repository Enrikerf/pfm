
En el código correspondiente del algoritmo de control \textit{Control Algorithm}~\cref{lst:Looper} el aspecto más relevante se encuentra en el cálculo de la variable controlada.
Para ello Se hace uso del Encoder; dentro del Dominio del controlador automático se ha desarrollado un modelo que lo representa.
El \textit{Encoder}~\cref{lst:Encoder} es una interfaz que actúa como puerto de salida hacia la implementación.
Define función de observación que llamaremos \textit{watchdog} de las lectura de dichos pines y una función a la que llamaremos para obtener la posición en el momento que deseemos, \textit{getPosition}.
También atañe al Dominio saber que hay que restablecer los parámetros del encoder cuando se requiera y que hay que desactivar el hardware cuando se deje de utilizar para que no haya inconsistencias a la hora de volver a ejecutar el programa.
Para ello dispone de las otras funciones de la interfaz.

El adaptador\textit{Encoder Model}~\cref{lst:EncoderAdapter} hace uso de una librería en Go para Raspberry que permite la interacción con los pines de lectura con los que está conectado el encoder y realiza las operaciones pertinentes para obtener la posición.
El dispositivo particular en el que se va a ejecutar la lógica de control, el encoder físicamente conectado y todos los detalles finales del hardware no son más que un detalle de implementación secundario como lo es una base de datos.
El punto de más relevancia se encuentra enmarcado en rojo en el~\cref{lst:EncoderAdapter}.
Cuando se ejecuta la función de \textit{watchdog} se ponen en marcha dos \textit{Gorutines}; una para vigilar cada lectura de los dos pines del Encoder.

Se queda en un bucle infinito la primera sentencia de dicho bucle es quedarse bloqueado hasta que haya un cambio en el pin de lectura, ya sea de 0 a 1 o de 1 a 0.
Si esto ocurre lo primero que hace es bloquear todas las demás \textit{Gorutines}, porque va a hacer cambios en memoria de forma atómica, escribe la información de dicha lectura en un array de lecturas y desbloquea las \textit{Gorutines}, se termina y vuelve a esperar.
Independientemente, el watchdog, una vez lanzadas las \textit{Gorutines} de lectura se queda en bucle infinito que lo que hace es bloquear las \textit{Gorutines}.
Extraer un elemento del array de lecturas y procesarlo.

Tal y como están diseñadas la \textit{Gorutines}, se van encolando y esperan su momento en la CPU para ejecutarse.
Si hay bloqueos también esperan su turno.
De esta forma, aunque el procesamiento de una lectura fuera bloqueado por el sistema operativo, se irían encolando las \textit{Gorutines} a la espera de poder añadirse en el array de lecturas, pero no se perderían.
De esta forma se consigue leer exactamente los 360 pulsos por vuelta del encoder sin perder ninguno.

\input{./part/Ejecucion/Seguimiento/PidControl/code/pid}

\phantom{blank}
\vspace{5mm}
\hrule
\begin{lstlisting}[language=Go,caption={Encoder.go},breaklines=true,label={lst:Encoder}]
package Entity

type Encoder interface {
	Watchdog()
	ResetPosition()
	GetPosition() int64
	TearDown()
}
\end{lstlisting}
\hrule
\input{./part/Ejecucion/Seguimiento/PidControl/code/encoderAdapter}