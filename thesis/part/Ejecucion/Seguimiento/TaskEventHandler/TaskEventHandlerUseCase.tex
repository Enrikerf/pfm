

El adaptador de salida \textit{Dispatcher}~\cref{lst:DispatcherAdapter} es a su vez un elemento que actuará como adaptador de entrada.
Ejecutará el caso de uso \textit{TaskEventHandlerUseCase}~\cref{lst:TaskEventHandlerUseCase} cuando el evento lanzado sea de tipo .\textit{TaskCreated}
Podemos ver en~\cref{lst:DispatcherAdapter} que el gestor de eventos está preparado para añadir más eventos y gestionar otros casos de uso en consecuencia.

\textit{TaskEventHandlerUseCase}~\cref{lst:TaskEventHandlerUseCase} Habilitará el \textit{Looper}~\cref{lst:Looper} si la tarea creada es de tipo AUTOMATIC y el status es PENDING\@.
El servicio de Dominio \textit{Looper} buscará todas las tareas automáticas pendientes y las ejecutará de forma asíncrona.

En el servicio \textit{Looper}~\cref{lst:Looper}, señalado en rojo está el bloque de gestión de las operaciones asíncronas.
Por cada tarea se crea un hilo de ejecución paralelo.
Cuando todos los hilos terminan el comando \textit{wg.Wait()} desbloquea la ejecución del \textit{Looper}.


\phantom{blank}
\vspace{5mm}
\hrule
\begin{lstlisting}[language=Go,caption={DispatcherAdapter.go},breaklines=true,label={lst:DispatcherAdapter}]
package EventDispatcherAdapter

import (
	"github.com/Enrikerf/pfm/commandManager/app/Application/Port/In/Task/TaskEventHandler"
	"github.com/Enrikerf/pfm/commandManager/app/Domain/Event"
	TaskEvent "github.com/Enrikerf/pfm/commandManager/app/Domain/Task/Event"
)

type EventDispatcherAdapter interface {
	Dispatch(event Event.Event)
}

type eventDispatcherAdapter struct {
	taskEventHandler TaskEventHandler.UseCase
}

func New(
	taskEventHandler TaskEventHandler.UseCase,
) EventDispatcherAdapter {
	self := &eventDispatcherAdapter{
		taskEventHandler,
	}
	return self
}

func (e *eventDispatcherAdapter) Dispatch(event Event.Event) {
	switch event.GetName() {
	case TaskEvent.TaskCreatedEventName:
		go e.taskEventHandler.Handle(event)
	}
}


\end{lstlisting}
\hrule


El adaptador de salida \textit{Dispatcher}~\cref{lst:DispatcherAdapter} es a su vez un elemento que actuará como adaptador de entrada.
Ejecutará el caso de uso \textit{TaskEventHandlerUseCase}~\cref{lst:TaskEventHandlerUseCase} cuando el evento lanzado sea de tipo .\textit{TaskCreated}
Podemos ver en~\cref{lst:DispatcherAdapter} que el gestor de eventos está preparado para añadir más eventos y gestionar otros casos de uso en consecuencia.

\textit{TaskEventHandlerUseCase}~\cref{lst:TaskEventHandlerUseCase} Habilitará el \textit{Looper}~\cref{lst:Looper} si la tarea creada es de tipo AUTOMATIC y el status es PENDING\@.
El servicio de Dominio \textit{Looper} buscará todas las tareas automáticas pendientes y las ejecutará de forma asíncrona.

En el servicio \textit{Looper}~\cref{lst:Looper}, señalado en rojo está el bloque de gestión de las operaciones asíncronas.
Por cada tarea se crea un hilo de ejecución paralelo.
Cuando todos los hilos terminan el comando \textit{wg.Wait()} desbloquea la ejecución del \textit{Looper}.


\phantom{blank}
\vspace{5mm}
\hrule
\begin{lstlisting}[language=Go,caption={DispatcherAdapter.go},breaklines=true,label={lst:DispatcherAdapter}]
package EventDispatcherAdapter

import (
	"github.com/Enrikerf/pfm/commandManager/app/Application/Port/In/Task/TaskEventHandler"
	"github.com/Enrikerf/pfm/commandManager/app/Domain/Event"
	TaskEvent "github.com/Enrikerf/pfm/commandManager/app/Domain/Task/Event"
)

type EventDispatcherAdapter interface {
	Dispatch(event Event.Event)
}

type eventDispatcherAdapter struct {
	taskEventHandler TaskEventHandler.UseCase
}

func New(
	taskEventHandler TaskEventHandler.UseCase,
) EventDispatcherAdapter {
	self := &eventDispatcherAdapter{
		taskEventHandler,
	}
	return self
}

func (e *eventDispatcherAdapter) Dispatch(event Event.Event) {
	switch event.GetName() {
	case TaskEvent.TaskCreatedEventName:
		go e.taskEventHandler.Handle(event)
	}
}


\end{lstlisting}
\hrule


El adaptador de salida \textit{Dispatcher}~\cref{lst:DispatcherAdapter} es a su vez un elemento que actuará como adaptador de entrada.
Ejecutará el caso de uso \textit{TaskEventHandlerUseCase}~\cref{lst:TaskEventHandlerUseCase} cuando el evento lanzado sea de tipo .\textit{TaskCreated}
Podemos ver en~\cref{lst:DispatcherAdapter} que el gestor de eventos está preparado para añadir más eventos y gestionar otros casos de uso en consecuencia.

\textit{TaskEventHandlerUseCase}~\cref{lst:TaskEventHandlerUseCase} Habilitará el \textit{Looper}~\cref{lst:Looper} si la tarea creada es de tipo AUTOMATIC y el status es PENDING\@.
El servicio de Dominio \textit{Looper} buscará todas las tareas automáticas pendientes y las ejecutará de forma asíncrona.

En el servicio \textit{Looper}~\cref{lst:Looper}, señalado en rojo está el bloque de gestión de las operaciones asíncronas.
Por cada tarea se crea un hilo de ejecución paralelo.
Cuando todos los hilos terminan el comando \textit{wg.Wait()} desbloquea la ejecución del \textit{Looper}.


\phantom{blank}
\vspace{5mm}
\hrule
\begin{lstlisting}[language=Go,caption={DispatcherAdapter.go},breaklines=true,label={lst:DispatcherAdapter}]
package EventDispatcherAdapter

import (
	"github.com/Enrikerf/pfm/commandManager/app/Application/Port/In/Task/TaskEventHandler"
	"github.com/Enrikerf/pfm/commandManager/app/Domain/Event"
	TaskEvent "github.com/Enrikerf/pfm/commandManager/app/Domain/Task/Event"
)

type EventDispatcherAdapter interface {
	Dispatch(event Event.Event)
}

type eventDispatcherAdapter struct {
	taskEventHandler TaskEventHandler.UseCase
}

func New(
	taskEventHandler TaskEventHandler.UseCase,
) EventDispatcherAdapter {
	self := &eventDispatcherAdapter{
		taskEventHandler,
	}
	return self
}

func (e *eventDispatcherAdapter) Dispatch(event Event.Event) {
	switch event.GetName() {
	case TaskEvent.TaskCreatedEventName:
		go e.taskEventHandler.Handle(event)
	}
}


\end{lstlisting}
\hrule


El adaptador de salida \textit{Dispatcher}~\cref{lst:DispatcherAdapter} es a su vez un elemento que actuará como adaptador de entrada.
Ejecutará el caso de uso \textit{TaskEventHandlerUseCase}~\cref{lst:TaskEventHandlerUseCase} cuando el evento lanzado sea de tipo .\textit{TaskCreated}
Podemos ver en~\cref{lst:DispatcherAdapter} que el gestor de eventos está preparado para añadir más eventos y gestionar otros casos de uso en consecuencia.

\textit{TaskEventHandlerUseCase}~\cref{lst:TaskEventHandlerUseCase} Habilitará el \textit{Looper}~\cref{lst:Looper} si la tarea creada es de tipo AUTOMATIC y el status es PENDING\@.
El servicio de Dominio \textit{Looper} buscará todas las tareas automáticas pendientes y las ejecutará de forma asíncrona.

En el servicio \textit{Looper}~\cref{lst:Looper}, señalado en rojo está el bloque de gestión de las operaciones asíncronas.
Por cada tarea se crea un hilo de ejecución paralelo.
Cuando todos los hilos terminan el comando \textit{wg.Wait()} desbloquea la ejecución del \textit{Looper}.

\input{./part/Ejecucion/Seguimiento/TaskEventHandler/codes/EventDispatcherAdapter}
\input{./part/Ejecucion/Seguimiento/TaskEventHandler/codes/TaskEventHandlerUseCase}
\input{./part/Ejecucion/Seguimiento/TaskEventHandler/codes/Looper}


\phantom{blank}
\vspace{5mm}
\hrule
\begin{lstlisting}[language=Go,caption={Looper.go},breaklines=true,label={lst:Looper}]
package Looper

import (
	"fmt"
	"github.com/Enrikerf/pfm/commandManager/app/Domain/Communication/Repository"
	ResultRepository "github.com/Enrikerf/pfm/commandManager/app/Domain/Result/Repository"
	"github.com/Enrikerf/pfm/commandManager/app/Domain/Task"
	"github.com/Enrikerf/pfm/commandManager/app/Domain/Task/ExecutionMode"
	TaskRepository "github.com/Enrikerf/pfm/commandManager/app/Domain/Task/Repository"
	"github.com/Enrikerf/pfm/commandManager/app/Domain/Task/Status"
	"sync"
)

var once sync.Once
var instance Looper

type Looper interface {
	IsEnabled() bool
	Enable()
}

func NewLooper(
	communicateRepository Repository.Communicate,
	findTasksByRepository TaskRepository.FindBy,
	saveTaskRepository TaskRepository.Save,
	saveBatchRepository ResultRepository.SaveBatch,
	saveResultRepository ResultRepository.Save,
) Looper {
	once.Do(func() {
		instance = &looper{
			communicateRepository: communicateRepository,
			findTasksByRepository: findTasksByRepository,
			saveTaskRepository:    saveTaskRepository,
			saveBatchRepository:   saveBatchRepository,
			saveResultRepository:  saveResultRepository,
			isLoopEnabled:         make(chan bool, 1),
		}
	})
	return instance
}

type looper struct {
	communicateRepository Repository.Communicate
	findTasksByRepository TaskRepository.FindBy
	saveTaskRepository    TaskRepository.Save
	saveBatchRepository   ResultRepository.SaveBatch
	saveResultRepository  ResultRepository.Save
	isLoopEnabled         chan bool
}

func (l *looper) IsEnabled() bool {
	return len(l.isLoopEnabled) != 0
}

func (l *looper) Enable() {
	l.isLoopEnabled <- true
	go l.loop()
}

func (l *looper) loop() {
	for l.IsEnabled() {
		tasks, err := l.findTasksByRepository.FindBy(map[string]interface{}{
			"status":         Status.Pending,
			"execution_mode": ExecutionMode.Automatic,
		})
		if err != nil {
			fmt.Printf(err.Error())
			l.stopLoop()
			return
		}
		if len(tasks) < 1 {
			l.stopLoop()
			return
		}
		<@\textcolor{red}{//---async waiting block---}@>
		var wg sync.WaitGroup
		for index := range tasks {
			wg.Add(1)
			go l.executeTask(&wg, tasks[index])
		}
		wg.Wait()
		<@\textcolor{red}{//------}@>
	}
}

func (l *looper) executeTask(wg *sync.WaitGroup, task Task.Task) {
	defer wg.Done()
	task.SetStatus(Status.New(Status.Running))
	l.saveTaskRepository.Persist(task)
	resultBatch := l.communicateRepository.Communicate(task)
	l.saveBatchRepository.Persist(resultBatch)
	task.SetStatus(Status.New(Status.Done))
	l.saveTaskRepository.Persist(task)
}

func (l *looper) stopLoop() {
	if len(l.isLoopEnabled) > 0 {
		fmt.Println("loop stopped")
		<-l.isLoopEnabled
	} else {
		fmt.Println("trying to stop loop but is not running")
	}
}


\end{lstlisting}
\hrule


\phantom{blank}
\vspace{5mm}
\hrule
\begin{lstlisting}[language=Go,caption={Looper.go},breaklines=true,label={lst:Looper}]
package Looper

import (
	"fmt"
	"github.com/Enrikerf/pfm/commandManager/app/Domain/Communication/Repository"
	ResultRepository "github.com/Enrikerf/pfm/commandManager/app/Domain/Result/Repository"
	"github.com/Enrikerf/pfm/commandManager/app/Domain/Task"
	"github.com/Enrikerf/pfm/commandManager/app/Domain/Task/ExecutionMode"
	TaskRepository "github.com/Enrikerf/pfm/commandManager/app/Domain/Task/Repository"
	"github.com/Enrikerf/pfm/commandManager/app/Domain/Task/Status"
	"sync"
)

var once sync.Once
var instance Looper

type Looper interface {
	IsEnabled() bool
	Enable()
}

func NewLooper(
	communicateRepository Repository.Communicate,
	findTasksByRepository TaskRepository.FindBy,
	saveTaskRepository TaskRepository.Save,
	saveBatchRepository ResultRepository.SaveBatch,
	saveResultRepository ResultRepository.Save,
) Looper {
	once.Do(func() {
		instance = &looper{
			communicateRepository: communicateRepository,
			findTasksByRepository: findTasksByRepository,
			saveTaskRepository:    saveTaskRepository,
			saveBatchRepository:   saveBatchRepository,
			saveResultRepository:  saveResultRepository,
			isLoopEnabled:         make(chan bool, 1),
		}
	})
	return instance
}

type looper struct {
	communicateRepository Repository.Communicate
	findTasksByRepository TaskRepository.FindBy
	saveTaskRepository    TaskRepository.Save
	saveBatchRepository   ResultRepository.SaveBatch
	saveResultRepository  ResultRepository.Save
	isLoopEnabled         chan bool
}

func (l *looper) IsEnabled() bool {
	return len(l.isLoopEnabled) != 0
}

func (l *looper) Enable() {
	l.isLoopEnabled <- true
	go l.loop()
}

func (l *looper) loop() {
	for l.IsEnabled() {
		tasks, err := l.findTasksByRepository.FindBy(map[string]interface{}{
			"status":         Status.Pending,
			"execution_mode": ExecutionMode.Automatic,
		})
		if err != nil {
			fmt.Printf(err.Error())
			l.stopLoop()
			return
		}
		if len(tasks) < 1 {
			l.stopLoop()
			return
		}
		<@\textcolor{red}{//---async waiting block---}@>
		var wg sync.WaitGroup
		for index := range tasks {
			wg.Add(1)
			go l.executeTask(&wg, tasks[index])
		}
		wg.Wait()
		<@\textcolor{red}{//------}@>
	}
}

func (l *looper) executeTask(wg *sync.WaitGroup, task Task.Task) {
	defer wg.Done()
	task.SetStatus(Status.New(Status.Running))
	l.saveTaskRepository.Persist(task)
	resultBatch := l.communicateRepository.Communicate(task)
	l.saveBatchRepository.Persist(resultBatch)
	task.SetStatus(Status.New(Status.Done))
	l.saveTaskRepository.Persist(task)
}

func (l *looper) stopLoop() {
	if len(l.isLoopEnabled) > 0 {
		fmt.Println("loop stopped")
		<-l.isLoopEnabled
	} else {
		fmt.Println("trying to stop loop but is not running")
	}
}


\end{lstlisting}
\hrule


\phantom{blank}
\vspace{5mm}
\hrule
\begin{lstlisting}[language=Go,caption={Looper.go},breaklines=true,label={lst:Looper}]
package Looper

import (
	"fmt"
	"github.com/Enrikerf/pfm/commandManager/app/Domain/Communication/Repository"
	ResultRepository "github.com/Enrikerf/pfm/commandManager/app/Domain/Result/Repository"
	"github.com/Enrikerf/pfm/commandManager/app/Domain/Task"
	"github.com/Enrikerf/pfm/commandManager/app/Domain/Task/ExecutionMode"
	TaskRepository "github.com/Enrikerf/pfm/commandManager/app/Domain/Task/Repository"
	"github.com/Enrikerf/pfm/commandManager/app/Domain/Task/Status"
	"sync"
)

var once sync.Once
var instance Looper

type Looper interface {
	IsEnabled() bool
	Enable()
}

func NewLooper(
	communicateRepository Repository.Communicate,
	findTasksByRepository TaskRepository.FindBy,
	saveTaskRepository TaskRepository.Save,
	saveBatchRepository ResultRepository.SaveBatch,
	saveResultRepository ResultRepository.Save,
) Looper {
	once.Do(func() {
		instance = &looper{
			communicateRepository: communicateRepository,
			findTasksByRepository: findTasksByRepository,
			saveTaskRepository:    saveTaskRepository,
			saveBatchRepository:   saveBatchRepository,
			saveResultRepository:  saveResultRepository,
			isLoopEnabled:         make(chan bool, 1),
		}
	})
	return instance
}

type looper struct {
	communicateRepository Repository.Communicate
	findTasksByRepository TaskRepository.FindBy
	saveTaskRepository    TaskRepository.Save
	saveBatchRepository   ResultRepository.SaveBatch
	saveResultRepository  ResultRepository.Save
	isLoopEnabled         chan bool
}

func (l *looper) IsEnabled() bool {
	return len(l.isLoopEnabled) != 0
}

func (l *looper) Enable() {
	l.isLoopEnabled <- true
	go l.loop()
}

func (l *looper) loop() {
	for l.IsEnabled() {
		tasks, err := l.findTasksByRepository.FindBy(map[string]interface{}{
			"status":         Status.Pending,
			"execution_mode": ExecutionMode.Automatic,
		})
		if err != nil {
			fmt.Printf(err.Error())
			l.stopLoop()
			return
		}
		if len(tasks) < 1 {
			l.stopLoop()
			return
		}
		<@\textcolor{red}{//---async waiting block---}@>
		var wg sync.WaitGroup
		for index := range tasks {
			wg.Add(1)
			go l.executeTask(&wg, tasks[index])
		}
		wg.Wait()
		<@\textcolor{red}{//------}@>
	}
}

func (l *looper) executeTask(wg *sync.WaitGroup, task Task.Task) {
	defer wg.Done()
	task.SetStatus(Status.New(Status.Running))
	l.saveTaskRepository.Persist(task)
	resultBatch := l.communicateRepository.Communicate(task)
	l.saveBatchRepository.Persist(resultBatch)
	task.SetStatus(Status.New(Status.Done))
	l.saveTaskRepository.Persist(task)
}

func (l *looper) stopLoop() {
	if len(l.isLoopEnabled) > 0 {
		fmt.Println("loop stopped")
		<-l.isLoopEnabled
	} else {
		fmt.Println("trying to stop loop but is not running")
	}
}


\end{lstlisting}
\hrule
