

El adaptador de salida \textit{Dispatcher}~\cref{lst:DispatcherAdapter} es a su vez un elemento que actuará como adaptador de entrada.
Ejecutará el caso de uso \textit{TaskEventHandlerUseCase}~\cref{lst:TaskEventHandlerUseCase} cuando el evento lanzado sea de tipo .\textit{TaskCreated}
Podemos ver en~\cref{lst:DispatcherAdapter} que el gestor de eventos está preparado para añadir más eventos y gestionar otros casos de uso en consecuencia.

\textit{TaskEventHandlerUseCase}~\cref{lst:TaskEventHandlerUseCase} Habilitará el \textit{Looper}~\cref{lst:Looper} si la tarea creada es de tipo AUTOMATIC y el status es PENDING\@.
El servicio de Dominio \textit{Looper} buscará todas las tareas automáticas pendientes y las ejecutará de forma asíncrona.

En el servicio \textit{Looper}~\cref{lst:Looper}, señalado en rojo está el bloque de gestión de las operaciones asíncronas.
Por cada tarea se crea un hilo de ejecución paralelo.
Cuando todos los hilos terminan el comando \textit{wg.Wait()} desbloquea la ejecución del \textit{Looper}.

\input{./part/Ejecucion/Seguimiento/TaskEventHandler/codes/EventDispatcherAdapter}


El adaptador de salida \textit{Dispatcher}~\cref{lst:DispatcherAdapter} es a su vez un elemento que actuará como adaptador de entrada.
Ejecutará el caso de uso \textit{TaskEventHandlerUseCase}~\cref{lst:TaskEventHandlerUseCase} cuando el evento lanzado sea de tipo .\textit{TaskCreated}
Podemos ver en~\cref{lst:DispatcherAdapter} que el gestor de eventos está preparado para añadir más eventos y gestionar otros casos de uso en consecuencia.

\textit{TaskEventHandlerUseCase}~\cref{lst:TaskEventHandlerUseCase} Habilitará el \textit{Looper}~\cref{lst:Looper} si la tarea creada es de tipo AUTOMATIC y el status es PENDING\@.
El servicio de Dominio \textit{Looper} buscará todas las tareas automáticas pendientes y las ejecutará de forma asíncrona.

En el servicio \textit{Looper}~\cref{lst:Looper}, señalado en rojo está el bloque de gestión de las operaciones asíncronas.
Por cada tarea se crea un hilo de ejecución paralelo.
Cuando todos los hilos terminan el comando \textit{wg.Wait()} desbloquea la ejecución del \textit{Looper}.

\input{./part/Ejecucion/Seguimiento/TaskEventHandler/codes/EventDispatcherAdapter}


El adaptador de salida \textit{Dispatcher}~\cref{lst:DispatcherAdapter} es a su vez un elemento que actuará como adaptador de entrada.
Ejecutará el caso de uso \textit{TaskEventHandlerUseCase}~\cref{lst:TaskEventHandlerUseCase} cuando el evento lanzado sea de tipo .\textit{TaskCreated}
Podemos ver en~\cref{lst:DispatcherAdapter} que el gestor de eventos está preparado para añadir más eventos y gestionar otros casos de uso en consecuencia.

\textit{TaskEventHandlerUseCase}~\cref{lst:TaskEventHandlerUseCase} Habilitará el \textit{Looper}~\cref{lst:Looper} si la tarea creada es de tipo AUTOMATIC y el status es PENDING\@.
El servicio de Dominio \textit{Looper} buscará todas las tareas automáticas pendientes y las ejecutará de forma asíncrona.

En el servicio \textit{Looper}~\cref{lst:Looper}, señalado en rojo está el bloque de gestión de las operaciones asíncronas.
Por cada tarea se crea un hilo de ejecución paralelo.
Cuando todos los hilos terminan el comando \textit{wg.Wait()} desbloquea la ejecución del \textit{Looper}.

\input{./part/Ejecucion/Seguimiento/TaskEventHandler/codes/EventDispatcherAdapter}


El adaptador de salida \textit{Dispatcher}~\cref{lst:DispatcherAdapter} es a su vez un elemento que actuará como adaptador de entrada.
Ejecutará el caso de uso \textit{TaskEventHandlerUseCase}~\cref{lst:TaskEventHandlerUseCase} cuando el evento lanzado sea de tipo .\textit{TaskCreated}
Podemos ver en~\cref{lst:DispatcherAdapter} que el gestor de eventos está preparado para añadir más eventos y gestionar otros casos de uso en consecuencia.

\textit{TaskEventHandlerUseCase}~\cref{lst:TaskEventHandlerUseCase} Habilitará el \textit{Looper}~\cref{lst:Looper} si la tarea creada es de tipo AUTOMATIC y el status es PENDING\@.
El servicio de Dominio \textit{Looper} buscará todas las tareas automáticas pendientes y las ejecutará de forma asíncrona.

En el servicio \textit{Looper}~\cref{lst:Looper}, señalado en rojo está el bloque de gestión de las operaciones asíncronas.
Por cada tarea se crea un hilo de ejecución paralelo.
Cuando todos los hilos terminan el comando \textit{wg.Wait()} desbloquea la ejecución del \textit{Looper}.

\input{./part/Ejecucion/Seguimiento/TaskEventHandler/codes/EventDispatcherAdapter}
\input{./part/Ejecucion/Seguimiento/TaskEventHandler/codes/TaskEventHandlerUseCase}
\input{./part/Ejecucion/Seguimiento/TaskEventHandler/codes/Looper}

\input{./part/Ejecucion/Seguimiento/TaskEventHandler/codes/Looper}

\input{./part/Ejecucion/Seguimiento/TaskEventHandler/codes/Looper}

\input{./part/Ejecucion/Seguimiento/TaskEventHandler/codes/Looper}
