
En esta sección vamos a exponer la ejecución del diseño realizado. Haciendo foco en los casos de uso más representativos de toda la metodología que este trabajo ha desarrollado en su fase de diseño. Se expondrá de una forma gráfica con diagramas que hagan comprensible la estructura y no tanto los detalles de implementación. El repositorio con el código al completo lo podemos encontrar en \href{https://github.com/Enrikerf/pfm}{github}

Una de las primeras características que cabe resaltar de golang es que no tiene clases ni herencia. Como estructura principal de datos tiene los structs. Si queremos controlar la instanciación de los elementos de dominio, por ejemplo, la declaración inicial de variables de dichos structs, debemos implementar patrones más allá del elemento de datos en sí. Para poder garantizar que dichos elementos se creen bajo una misma lógica que no se pueda evitar manteniendo la consistencia. Otro punto importante a entender desde el inicio es que Golang sólo expone elementos, ya sea structs, funciones o demás tipos fuera de un módulo si están escritos en mayúscula. Por último, en Golang un módulo son todos los archivos dentro de un mismo directorio.

Entendido estos tres puntos, una forma de crear una estructura equivalente a una clase es definir una interfaz pública y crear un struct privado que la implemente. Los struct pueden implementar metodos pasandose a si mismos por referencia en una funcion. Podemos ver una implementación a modo de ejemplo en la~\cref{fig:golang class equivalent}. En la interfaz \textbf{C}lassName la “C“ es mayúscula exponiendolo al exterior del módulo y \textbf{c}lassName struct en minuscula encapsulándolo. Podemos verlo más claramente en el diagrama UML~\cref{fig: uml Diagram Class Equivalent in Golang}

Los motivos por los que es importante concebir esta estructura en lugar de utilizar structs directamente son:

\begin{itemize}
    \item Evita la instanciación no consistente, lo cual sólo se garantiza a traves del método exportado NewClass.
    \item Evita accesos no deseados a la modificación de variables. Siendo publicas seria posible. Garantiza la inmutabilidad en la que se basan los \textit{Value Objects}.
    \item Limpieza en gestión errores en Go. Ante la respuesta múltiple, que contemple devolver un resultado o error, evitamos instanciar el objeto con contenido vacío creando un elemento inconsistente sólo por exigencias del lenguaje. De esta forma admite retornar null.
\end{itemize}

\begin{figure}[H]
    \centering
    \includegraphics[height=0.5\textheight]{./part/Ejecucion/Seguimiento/classExample}
    \caption{Golang class equivalent}\label{fig:golang class equivalent}
\end{figure}

\begin{figure}[H]
    \centering
    \includegraphics[height=0.25\textheight]{./part/Proyecto_ejecutivo/memoria_constructiva/ClassEquivalentInGolang}
    \caption{UML diagram: Class equivalent in Golang}\label{fig: uml Diagram Class Equivalent in Golang}
\end{figure}

Como contrapartida tenemos que escribir más codigo. Poniendo como comparación una clase Java o C++ no podemos decir que en cantidad de lineas escritas se aumente o disminuya, aunque estéticamente pueda resultar incómodo. En cuestión de conceptos aprendidos en Golang sólamente trabaja con interfaces y structs contra el concepto de clase y herencia.

Una vez presentado estos conceptos pasamos a exponer el código. Los casos de uso más complejos y completos para ver todo el diseño aplicado son: CreateTaskUseCase y TaskEventHandler.

\subsubsection{CreateTaskUseCase}\label{subsubsec:CreateTaskUseCase}
    
Se toma como referencia la~\cref{fig:hexagonalDiagram}, el diagrama más extendido de ejemplo para una arquitectura hexagonal, para una explicación de alto nivel para la funcionalidad de crear tareas \textit{CreateTaskUseCase} que se aprecia en la~\cref{fig:CreateTaskHexagonalDiagram}.
Los componentes que intervienen son:
\begin{itemize}
    \item \textit{WebAdapter}: contiene la lógica relacionada con la recepción de llamadas RPC está ligada a la Infraestructura necesaria para ello, es decir importa las librerías para ello.
    A través de una interfaz se le ha de inyectar un puerto de entrada que le permita ejecutar la lógica del caso de uso.
    \item UseCase: contiene la lógica necesaria para garantizar la correcta llamada a la lógica de Dominio para la creación de una Tarea.
    Contiene las verificaciones, mediante la instanciación, entre los tipos primitivos y los objetos de Dominio.
    \item \textit{Creator}: servicio de creación de tareas.
    Contiene la lógica que obligatoriamente ha de ejecutarse de forma simultanea cuando se crea una tarea.
    La persistencia en la base de datos a través de un puerto de salida; una interfaz que se le inyecta y le permite ejecutar dicha lógica sin conocerla.
    También de emite un evento de Dominio comunicando la creación mediante el acceso a un puerto de salida.
    \item \textit{PersistenceAdapter}: contiene la lógica relacionada con la persistencia de datos.
    está ligado a la Infraestructura de la base de datos y contiene por tanto las importaciones de las librerías necesarias.
    \item \textit{DispatcherAdapter}: este puerto de salida contiene la lógica para gestionar los eventos emitidos por el Dominio.
    Para ello, necesita que le sean inyectados los casos de uso, eventHandler en este caso al no ser ejecutados directamente por un usuario.
    De esta forma, a la vez que un puerto de salida se convierte en un Adaptador de entrada que ejecuta puertos de entrada de nuevo a la Aplicación.
    \item \textit{Looper}: Contiene la lógica necesaria para obtener de base de datos, a través de un puerto de salida, todas las tareas sobre las que hay que iterar para ejecutarlas.
    \item \textit{Entity}: en este caso de uso serán las Tareas (\textit{Tasks})
\end{itemize}

Encontramos en la figura la numeración en orden de ejecución.
\begin{enumerate}
    \item se recibe una llamada RPC de tipo \textit{CreateTask} mediante el WebAdapter y es atendida
    \item A través de puerto de entrada se ejecuta el caso de uso
    \item El caso de uso instancia los elementos de Dominio necesarios y ejecuta el servicio \textit{Creator}
    \item El servicio \textit{Creator} comunica la información a persistir a través del puerto de salida
    \item La interfaz del el Adaptador de persistencia que ha de ser inyectada en el \textit{Creator}
    \item El componente inyectado en el \textit{Creator} que cumple con la interfaz 5 persiste la información en base de datos
    \item El servicio \textit{Creator} comunica el evento de creación a través del puerto de salida 7
    \item El servicio \textit{Dispatcher} inyectado en el creator que cumple con la interfaz 7 ejecuta el caso de uso correspondiente a ese evento a través de la interfaz
    \item La interfaz del caso de uso, en este caso \textit{eventHandler} de la creación de tareas
    \item El caso de uso que gestiona los eventos de creación de tareas.
    Cumple con la interfaz 9 y se encarga de comprobar que si el tipo de tarea creada requiere la activación del servicio de ejecución.
    \item El servicio de ejecución de tareas automáticas o \textit{Looper}.
    Se encarga de obtener a través de el puerto de salida 5, de la base de datos, y ejecutar todas las tareas definidas como automáticas.
\end{enumerate}

\begin{figure}[H]
    \centering
    \includegraphics[height=0.3\textheight]{./part/Ejecucion/Seguimiento/CreateTaskUseCase/img/CreateTaskHexagonalDiagram}
    \caption{Hexagonal architecture diagram}\label{fig:CreateTaskHexagonalDiagram}
\end{figure}

Se puede apreciar que el UL ayuda a la comprensión de los pasos y las funcionalidades.
El diseño ayuda a diferenciar la resolución del problema: representado por la creación de tareas y su ejecución.
Separando el porqué y dónde se ejecuta: se activa la resolución del problema por una llamada RPC, en una Base de datos y en un sistema remoto.
Si cualquiera de los elementos que acceden a el programa que resuelve el caso de uso quiere ser intercambiado, la única exigencia es implementar las interfaces requeridas por la capa de Aplicación.

La implementación concreta para componer la capa de Aplicación correspondiente a este caso de uso la encontramos en el~\cref{lst:bootstrapingExample}.
El orden de instanciación e inyección de cada uno de los componentes necesarios para el armado de la aplicación se vuelve un poco más confuso a la hora de llevarlo a la práctica, es por esto que los ejemplos gráficos son de gran ayuda.

\phantom{blank}
\vspace{10mm}
\hrule
\begin{lstlisting}[language=Go,caption={Ejemplo de \textit{Bootstraping} del sistema },breaklines=true,label={lst:bootstrapingExample}]

package Bootstrap

func Bootstrap() {
    //infra
    var persistenceAdapter := PersistenceAdapter.New(<infra dependencies>)
    //Domain
    var looper := Looper.New(persistenceAdapter)
    var createdEventHandler := CreatedEventHandler(looper)
    //Application
    var dispatcherAdapter := DispatcherAdapter.New(createdEventHandler)
    //Domain
    var creator := creator.New(dispatcherAdapter)
    //Application
    var useCase := UseCase.New(creator)
    //Infra
    var webAdapter := WebAdapter.New(<infra dependencies>,useCase)
}

\end{lstlisting}
\hrule

Para una explicación conceptual de la arquitectura como técnica de diseño los diagramas hexagonales pueden tener un valor didáctico, sin embargo, para un caso de uso más completo como el que desarrolla esta sección pierde el sentido.
No logra ser explicativo, ya que hay puertos de salida que se convierten en puertos de entrada como es el caso del Dispatcher, y al quitar lógica de negocio del caso de uso mediante servicios de Dominio que impidan el acceso directo a los repositorios de persistencia dicho diagrama se va quedando pequeño.
Es preferible diagramas UML completos como el desarrollado en la~\cref{fig:createTaskUseCaseArchitecture}.
En el diagrama~\cref{fig:createTaskUseCaseArchitectureFolderStructure} se muestra interacción de componentes atendiendo a su distribución en nuestro diseño organizativo del código en ficheros.

Como estrategia de mapeo entre capas o \textit{mapping} Go casi obliga al uso general de interfaces entre todos los elementos, incluidos las clases.
Al necesitar el equivalente de la implementación de clase para garantizar la cohesión ya se hace uso de  interfaces para todos los elementos.
En el desarrollo de la aplicación se ha optado por una estrategia \textit{Full Mapping} con adaptaciones.
Dentro de la estrategia \textit{Full Mapping} se debe implementar un modelo tanto de entrada, como ya se contempla, como de salida.
Es decir, la respuesta que hay entre cada capa también debe ser mapeada mediante un DTO, el patrón adaptado se muestra en la figura~\cref{fig:GetHandMapping}

Esto aumenta la burocracia y finalmente la opción implementada ha sido una adaptación personalizada que se muestra en la figura~\cref{fig:CreateTaskUseCaseMapping}.
En rojo se han marcado los atajos que se han tomado, es decir, el mapeo que no se ha implementado.
El objetivo es aislar bien de los puertos de entrada, es decir el GRPC, y no tanto de los puertos de salida;
ya que la persistencia está bien aislada mediante las interfaces y somos propietarios y consumidores internos de ellas.
Al ser los únicos consumidores de los puertos de salida se dispone de más libertad de cambio sin afectar a terceros.
Dentro de los adaptadores se hace el trabajo de mapeo entre las entidades de Dominio y los modelos de persistencia; traduciendo de nuevo a Dominio para responder.

\begin{figure}[H]
    \centering
    \includegraphics[angle=90,height=1\textheight]{./part/Ejecucion/Seguimiento/CreateTaskUseCase/img/createTaskUseCaseArchitecture}
    \caption{CreateTaskUseCase hexagonal architecture diagram}\label{fig:createTaskUseCaseArchitecture}
\end{figure}

\begin{figure}[H]
    \centering
    \includegraphics[height=0.5\textheight]{./part/Ejecucion/Seguimiento/CreateTaskUseCase/img/PFM - CreateUseCaseFolderStructure}
    \caption{CreateTaskUseCase folder Structure}\label{fig:createTaskUseCaseArchitectureFolderStructure}
\end{figure}

\begin{figure}[H]
    \centering
    \includegraphics[height=0.2\textheight]{./part/Ejecucion/Seguimiento/CreateTaskUseCase/img/PFM - GetHandMapping}
    \caption{\textit{Full Mapping} con DTO de salida y de entrada\cite{TomHombergs2019GYHD}}\label{fig:GetHandMapping}
\end{figure}

\begin{figure}[H]
    \centering
    \includegraphics[height=0.2\textheight]{./part/Ejecucion/Seguimiento/CreateTaskUseCase/img/PFM - FinalMapping}
    \caption{\textit{CreateTaskUseCase Mapping} o Mapeo}\label{fig:CreateTaskUseCaseMapping}
\end{figure}

El código concreto para este caso de uso más relevante se compone de:
\begin{itemize}
    \item TaskController~\cref{lst:TaskControler}
    \item CreateTaskUseCase~\cref{lst:CreateTaskUseCaseCode}
    \item Creator~\cref{lst:Creator}
    \item SavePort~\cref{lst:SavePort}
    \item PersistAdapter~\cref{lst:SaveAdapter}
    \item DispatcherPort~\cref{lst:Dispatcher}
\end{itemize}

En el controller del~\cref{lst:TaskControler} se hace uso de la request entrante y se compone el comando que corresponde para el caso de uso.
Mediante la interfaz inyectada en el controller en el \textit{bootstrapping} se ejecuta el caso de uso resaltado en rojo.
También es interesante resaltar que una vez ejecutado el caso de uso se utiliza la respuesta, que contiene el id de la nueva Task creada para devolverlo.
Sin embargo, debido la estrategia utilizada de \textit{mapping}, definida en el diagrama~\cref{fig:CreateTaskUseCaseMapping}, no hacemos uso directo de esta respuesta si no que componemos una nueva para el protocolo de comunicación que está utilizando el cliente, en este caso RPC.
En los imports de este componente se aprecia que sólo hay dependencias de la Infraestructura, gRPC en este caso, y de la capa de Aplicación haciendo uso del comando y la interfaz del caso de uso.

La implementación del caso de uso se muestra en el~\cref{lst:CreateTaskUseCaseCode}.
Los imports son una forma rápida de comprobar la independencia de capas, de que la arquitectura, y de que los límites se están respetando.
En este caso todos corresponden a Dominio, no tiene imports de librerías de terceros, es decir, Infraestructura.
La interfaz, utilizada en el controlador, \textit{UseCase} se encuentra escrita con la primera letra en mayúsculas, exponiéndola hacia afuera, mientras que el struct \textit{useCase} está en minúscula no pudiendo ser utilizado fuera del package.
El caso de uso realiza la instanciación de los \textit{ValueObjects} necesarios para la creación de una \textit{Task}.
Asegurando que no dan ningún error los datos de entrada y luego, haciendo uso del servicio se procede a crearla.
Aparece marcado en rojo el uso del servicio \textit{Creator} de Dominio.

En este caso se junta en un mismo fichero la interfaz y la implementación, al pertenecer a la misma capa.
No ocurre lo mismo dentro de \textit{Creator}~\cref{lst:Creator} Donde las interfaces de \textit{SaveRepository.Persist}~\cref{lst:SavePort} y \textit{Dispatcher.Dispatch}~\cref{lst:Dispatcher} se encuentra dentro del Dominio, pero la implementación\textit{SaveAdapter}~\cref{lst:SaveAdapter} se encuentra en la capa de Infraestructura.

El servicio de Dominio Creator se encarga de encapsular y atomizar los procesos que tienen que ocurrir en bloque siempre que se desee crear una Task.
Entre los cuales está: instanciar la tarea con los ValueObjects que recibe como parámetros;
Persistir la misma en la base de datos a través del puerto de salida, interfaz que implementará un adaptador;
y emitir el evento de creación a través del otro puerto de salida, el Dispatcher interfaz que implementará otro adaptador.
En la sección de \textit{imports} se aprecia que no hay elementos de Infraestructura o Aplicación.
Se limita al uso de Dominio.
Aislados del exterior.


\phantom{blank}
\vspace{5mm}
\hrule
\begin{lstlisting}[language=Go,caption={TaskController.go},breaklines=true,label={lst:TaskControler}]

package Controller

import (
	"context"
	"fmt"
	taskProto "github.com/Enrikerf/pfm/commandManager/app/Adapter/In/ApiGrcp/gen/task"
	"github.com/Enrikerf/pfm/commandManager/app/Application/Port/In/Task/CreateTask"
	"google.golang.org/grpc/codes"
	"google.golang.org/grpc/status"
)

type TaskController struct {
	SaveTaskUseCase   CreateTask.UseCase
}

func (controller TaskController) CreateTask(
	ctx context.Context,
	request *taskProto.CreateTaskRequest,
) (*taskProto.CreateTaskResponse, error) {
	protoTask := request.GetTaskParams()
	var command CreateTask.Command
	command.Host = protoTask.GetHost()
	command.Port = protoTask.GetPort()
	command.CommandSentences = protoTask.GetCommands()
	command.CommunicationMode = protoTask.GetMode().String()
	command.ExecutionMode = protoTask.GetExecutionMode().String()

    <@\textcolor{red}{task, err := controller.SaveTaskUseCase.Create(command)}@>

	if err != nil {
		return nil, fmt.Errorf("error")
	}
	var commandNames []string
	for _, command := range task.GetSteps() {
		commandNames = append(commandNames, command.GetSentence())
	}
	newTask := taskProto.Task{
		Uuid:          task.GetId().GetUuidString(),
		Host:          task.GetHost().GetValue(),
		Port:          task.GetPort().GetValue(),
		Commands:      commandNames,
		Mode:          string(task.GetCommunicationMode()),
		Status:        string(task.GetStatus().Value()),
		ExecutionMode: string(task.GetExecutionMode()),
	}
	return &taskProto.CreateTaskResponse{Task: &newTask}, nil
}

\end{lstlisting}
\hrule
\input{./part/Ejecucion/Seguimiento/CreateTaskUseCase/codes/createTaskUseCaseCode}
\input{./part/Ejecucion/Seguimiento/CreateTaskUseCase/codes/Creator}

\phantom{blank}
\vspace{5mm}
\hrule
\begin{lstlisting}[language=Go,caption={SavePort.go},breaklines=true,label={lst:SavePort}]
package Repository

import "github.com/Enrikerf/pfm/commandManager/app/Domain/Task"

type Save interface {
	Persist(task Task.Task)
}


\end{lstlisting}
\hrule

\phantom{blank}
\vspace{5mm}
\hrule
\begin{lstlisting}[language=Go,caption={Dispatcher.go},breaklines=true,label={lst:Dispatcher}]
package Event

type Dispatcher interface {
	Dispatch(event Event)
}


\end{lstlisting}
\hrule
\input{./part/Ejecucion/Seguimiento/CreateTaskUseCase/codes/SaveAdapter}


\subsubsection{TaskEventHandler}
    

El adaptador de salida \textit{Dispatcher}~\cref{lst:DispatcherAdapter} es a su vez un elemento que actuará como adaptador de entrada.
Ejecutará el caso de uso \textit{TaskEventHandlerUseCase}~\cref{lst:TaskEventHandlerUseCase} cuando el evento lanzado sea de tipo .\textit{TaskCreated}
Podemos ver en~\cref{lst:DispatcherAdapter} que el gestor de eventos está preparado para añadir más eventos y gestionar otros casos de uso en consecuencia.

\textit{TaskEventHandlerUseCase}~\cref{lst:TaskEventHandlerUseCase} Habilitará el \textit{Looper}~\cref{lst:Looper} si la tarea creada es de tipo AUTOMATIC y el status es PENDING\@.
El servicio de Dominio \textit{Looper} buscará todas las tareas automáticas pendientes y las ejecutará de forma asíncrona.

En el servicio \textit{Looper}~\cref{lst:Looper}, señalado en rojo está el bloque de gestión de las operaciones asíncronas.
Por cada tarea se crea un hilo de ejecución paralelo.
Cuando todos los hilos terminan el comando \textit{wg.Wait()} desbloquea la ejecución del \textit{Looper}.

\input{./part/Ejecucion/Seguimiento/TaskEventHandler/codes/EventDispatcherAdapter}


El adaptador de salida \textit{Dispatcher}~\cref{lst:DispatcherAdapter} es a su vez un elemento que actuará como adaptador de entrada.
Ejecutará el caso de uso \textit{TaskEventHandlerUseCase}~\cref{lst:TaskEventHandlerUseCase} cuando el evento lanzado sea de tipo .\textit{TaskCreated}
Podemos ver en~\cref{lst:DispatcherAdapter} que el gestor de eventos está preparado para añadir más eventos y gestionar otros casos de uso en consecuencia.

\textit{TaskEventHandlerUseCase}~\cref{lst:TaskEventHandlerUseCase} Habilitará el \textit{Looper}~\cref{lst:Looper} si la tarea creada es de tipo AUTOMATIC y el status es PENDING\@.
El servicio de Dominio \textit{Looper} buscará todas las tareas automáticas pendientes y las ejecutará de forma asíncrona.

En el servicio \textit{Looper}~\cref{lst:Looper}, señalado en rojo está el bloque de gestión de las operaciones asíncronas.
Por cada tarea se crea un hilo de ejecución paralelo.
Cuando todos los hilos terminan el comando \textit{wg.Wait()} desbloquea la ejecución del \textit{Looper}.

\input{./part/Ejecucion/Seguimiento/TaskEventHandler/codes/EventDispatcherAdapter}


El adaptador de salida \textit{Dispatcher}~\cref{lst:DispatcherAdapter} es a su vez un elemento que actuará como adaptador de entrada.
Ejecutará el caso de uso \textit{TaskEventHandlerUseCase}~\cref{lst:TaskEventHandlerUseCase} cuando el evento lanzado sea de tipo .\textit{TaskCreated}
Podemos ver en~\cref{lst:DispatcherAdapter} que el gestor de eventos está preparado para añadir más eventos y gestionar otros casos de uso en consecuencia.

\textit{TaskEventHandlerUseCase}~\cref{lst:TaskEventHandlerUseCase} Habilitará el \textit{Looper}~\cref{lst:Looper} si la tarea creada es de tipo AUTOMATIC y el status es PENDING\@.
El servicio de Dominio \textit{Looper} buscará todas las tareas automáticas pendientes y las ejecutará de forma asíncrona.

En el servicio \textit{Looper}~\cref{lst:Looper}, señalado en rojo está el bloque de gestión de las operaciones asíncronas.
Por cada tarea se crea un hilo de ejecución paralelo.
Cuando todos los hilos terminan el comando \textit{wg.Wait()} desbloquea la ejecución del \textit{Looper}.

\input{./part/Ejecucion/Seguimiento/TaskEventHandler/codes/EventDispatcherAdapter}
\input{./part/Ejecucion/Seguimiento/TaskEventHandler/codes/TaskEventHandlerUseCase}
\input{./part/Ejecucion/Seguimiento/TaskEventHandler/codes/Looper}

\input{./part/Ejecucion/Seguimiento/TaskEventHandler/codes/Looper}

\input{./part/Ejecucion/Seguimiento/TaskEventHandler/codes/Looper}

\subsubsection{PID Control}\label{subsubsec:pidControl}
    
En el código correspondiente del algoritmo de control \textit{Control Algorithm}~\cref{lst:Looper} el aspecto más relevante se encuentra en el cálculo de la variable controlada.
Para ello Se hace uso del Encoder; dentro del Dominio del controlador automático se ha desarrollado un modelo que lo representa.
El \textit{Encoder}~\cref{lst:Encoder} es una interfaz que actúa como puerto de salida hacia la implementación.
Define función de observación que llamaremos \textit{watchdog} de las lectura de dichos pines y una función a la que llamaremos para obtener la posición en el momento que deseemos, \textit{getPosition}.
También atañe al Dominio saber que hay que restablecer los parámetros del encoder cuando se requiera y que hay que desactivar el hardware cuando se deje de utilizar para que no haya inconsistencias a la hora de volver a ejecutar el programa.
Para ello dispone de las otras funciones de la interfaz.

El adaptador\textit{Encoder Model}~\cref{lst:EncoderAdapter} hace uso de una librería en Go para Raspberry que permite la interacción con los pines de lectura con los que está conectado el encoder y realiza las operaciones pertinentes para obtener la posición.
El dispositivo particular en el que se va a ejecutar la lógica de control, el encoder físicamente conectado y todos los detalles finales del hardware no son más que un detalle de implementación secundario como lo es una base de datos.
El punto de más relevancia se encuentra enmarcado en rojo en el~\cref{lst:EncoderAdapter}.
Cuando se ejecuta la función de \textit{watchdog} se ponen en marcha dos \textit{Gorutines}; una para vigilar cada lectura de los dos pines del Encoder.

Se queda en un bucle infinito la primera sentencia de dicho bucle es quedarse bloqueado hasta que haya un cambio en el pin de lectura, ya sea de 0 a 1 o de 1 a 0.
Si esto ocurre lo primero que hace es bloquear todas las demás \textit{Gorutines}, porque va a hacer cambios en memoria de forma atómica, escribe la información de dicha lectura en un array de lecturas y desbloquea las \textit{Gorutines}, se termina y vuelve a esperar.
Independientemente, el watchdog, una vez lanzadas las \textit{Gorutines} de lectura se queda en bucle infinito que lo que hace es bloquear las \textit{Gorutines}.
Extraer un elemento del array de lecturas y procesarlo.

Tal y como están diseñadas la \textit{Gorutines}, se van encolando y esperan su momento en la CPU para ejecutarse.
Si hay bloqueos también esperan su turno.
De esta forma, aunque el procesamiento de una lectura fuera bloqueado por el sistema operativo, se irían encolando las \textit{Gorutines} a la espera de poder añadirse en el array de lecturas, pero no se perderían.
De esta forma se consigue leer exactamente los 360 pulsos por vuelta del encoder sin perder ninguno.

\input{./part/Ejecucion/Seguimiento/PidControl/code/pid}

\phantom{blank}
\vspace{5mm}
\hrule
\begin{lstlisting}[language=Go,caption={Encoder.go},breaklines=true,label={lst:Encoder}]
package Entity

type Encoder interface {
	Watchdog()
	ResetPosition()
	GetPosition() int64
	TearDown()
}
\end{lstlisting}
\hrule
\input{./part/Ejecucion/Seguimiento/PidControl/code/encoderAdapter}
\subsubsection{Testing}
    
Proteger el sistema frente al error humano es esencial para garantizar la entrega de código con seguridad y calidad. Para ello se emplearán herramientas de \gls{CI/CD} que se encargarán de realizar comprobaciones antes de permitir la llegada a servicio del nuevo código y el despliegue automático de dicho código en el entorno final. Este proyecto se centra en los primeros, CI o Continuous integration, debido a que no disponemos de servidores sobre los que desplegar de forma final.

En términos técnicos, CI/CD implica el uso de sistemas de control de versiones, servidores de automatización de compilación y pruebas, y herramientas de automatización de implementación y entrega. Estas herramientas permiten que los desarrolladores integren sus cambios de código en un repositorio central, donde se desencadenan automáticamente pruebas y compilaciones para detectar cualquier error y garantizar que el código se pueda implementar sin problemas.

Una vez que el código ha pasado todas las pruebas y ha sido aprobado por los revisores, el proceso de entrega y/o implementación se activa automáticamente. Esto implica la creación de un paquete de implementación, la realización de pruebas de aceptación automatizadas y el despliegue en producción. Este proceso ayuda a mejorar la eficiencia del equipo de desarrollo y a reducir los errores y riesgos en el proceso de entrega de software, permitiendo una implementación más rápida y frecuente de nuevas características y mejoras en el software.

Los tests son un apartado importante en la garantía de entrega de código libre de errores. En este proyecto se va a centrar el control de calidad en el dominio. Es decir, va a existir una suite de test unitarios por cada elemento de dominio que exista. Los test unitarios de aplicación e infraestructura quedan como un elemento deseable a tener, pero dependerá de los tiempos. los test de integración quedan fuera de esta primera versión del sistema. Las técnicas utilizadas para el desarrollo de los tests encuentran su origen en las siguentes condiciones de contorno:

\begin{itemize}
    \item si el código de pruebas no ahorra más esfuerzo de lo que cuesta desarrollarlo entonces se está perdiendo dinero.
    \item no se puede entregar un producto que no tiene ningúna clase de garantía de su correcto funcionamiento o acotado su margen de error.
\end{itemize}

Debemos trabajar dentro de esos dos límites: tener tests que garanticen y especifíquen el grado de calidad acotado al presupuesto que se dispone. Para poder expresar de una forma cuantitativa que los tests están correctamente diseñados y desarrollados primero definamos ineficaz e ineficiente. definición segun RAE:

\begin{itemize}
    \item eficaz: Capacidad de lograr el efecto que se desea o se espera.
    \item eficiente (2 accepción): Capacidad de lograr los resultados deseados con el mínimo posible de recursos
\end{itemize}

atendamos a la siguiente frase: si existiese un numero exacto de pruebas optimas, para cubrir todas la posibilidades con tests, si haces menos estas siendo ineficaz y si haces más ineficiente. Viendo estas definiciones queda claro que, cuando nos enfrentamos a la realidad, el trabajo de ingeniería constará en tomar una solución de compromiso entre eficacia y eficiencia. No podemos no probar y no tendremos recursos para probar todo.

En un ejemplo muy sencillo: para un método que recibe la información de un formulario de 8 campos con 10 posibles valores diferentes cada uno se obtendría un conjunto de \[ 8^{10} = 1,073,741,824 \] casos de prueba. Esta táctica de pruebas siendo la más obvia la mayor parte de las veces será inasumible.

El estado del arte y la técnica en diseño de test es amplia y variada. Este proyecto utilizará pruebas de caja negra. Las pruebas de caja negra se basan en diseñar un test sin suponer cómo va a estar desarrollado el código. Se diseña el caso de uso al que va a tener que enfretarse y el resultado esperado. Para evitar la prueba exhaustiva existen las siguientes técnicas:

\begin{itemize}
    \item variables independientes
    \subitem clases de equivalencia.
    \subitem metodo de valores límite
    \item variables dependientes
    \subitem vector de pares
\end{itemize}

\textbf{Variables independientes}

Su enfoque es reducir el conjunto de los valores posibles de cada entrada a un subconjunto representativo y reducido del conjunto original que asegure una cierta garantía de cobertura. Entonces la combinación de todos estos valores representativos de todas las entradas reducirá drásticamente el conjunto de casos de prueba final. Estas técnicas son Clases de Equivalencia y Análisis de Valores Límite; la segunda se aplica sobre la anterior.

Las clases de equivalencia son un subconjunto de valores de la entrada que el sistema debería manejar de forma equivalente en el ejercicio del \gls{SUT}(System Under Test). Para cada entrada deben repartirse todos sus posibles valores en un número finito de clases de equivalencia que cumplan la siguiente propiedad: la prueba de un valor representativo de una clase de equivalencia permite suponer “razonablemente“ que el resultado obtenido será un proceso similar que el obtenido probando cualquier otro valor de esa clase de equivalencia.

Es un proceso heurístico y se obtiene analizando:
\begin{itemize}
    \item la especificación (caja negra) que describe las características del proceso que debe cumplir la implementación.
    \item las salidas del SUT, que pueden aportar criterios para la partición en clases de equivalencia.
    \item las precondiciones del SUT, que aportan clases de equivalencia de error.
\end{itemize}

los pasos para diseñar el test son:
\begin{itemize}
    \item encontrar las Clases de Equivalencia de cada factor
    \item escoger un valor cualquiera de cada clase de equivalencia de cada factor
    \item si existen varios factores, generar todas las combinaciones de todos los valores anteriores de cada factor
    \item eliminar aquella combinaciones que no son factibles por la combinación de los valores de entrada
    \item añadir la salida correspondiente a cada combinación de valores de entrada
\end{itemize}

Para escoger dentro de una clase de equivalencia un valor se aplica la técnica de los valores límites. Esta técnica es aplicable a la partición de clases de equivalencia cuando sus valores tiene un orden total, o sea, que para toda pareja de valores distintos se puede comprobar si uno es mayor que otro o viceversa. Por ejemplo, enteros, reales, caracteres por su código, cadenas de caracteres por su orden lexicográfico, enumerados por su ordinal, fechas o horas. Son valores dentro de esa clase de equivalencia para el que cambia el comportamiento del SUT respecto del valor anterior.

La justificación se basa en la evidencia experimental de que “los errores se esconden en los rincones y se aglomeran en los límites” [Beizer] y por tanto se aumenta el conjunto de casos de prueba para mejorar la eficacia de encontrar errores. Es decir, que dentro de una clase de equivalencia puede requerirse el testeo de varios valores.

Vamos a exponer dos ejemplos que muestran la diferencia entre: parámetro de entrada y factor; y entre factor y clase de equivalencia.

Supongamos una función que acepta como parámetro de entrada un texto.

\begin{verbatim}
    funcion de nextDay(string day){...}
\end{verbatim}

Como estamos diseñando el test, no podemos saber la implementación. La entrada es un texto cualquiera. pero si quisieramos testear todos los textos posibles que pueden introducirse las pruebas necesarias serían infinitas. Aquí entra el concepto de factor. El factor es que para esa entrada existen 3 posibilidades: que la entrada sea invalida, o que sea válida. Para esos dos factores el conjunto que define la clases de equivalencia son

\begin{itemize}
    \item invalido: cualquier texto invalido
    \item válido: “lunes“,“martes“,“miércoles“,“jueves“,“viernes“,“sábado“ o “domingo“.
\end{itemize}

por lo tanto podríamos coger cualquier valor dentro de esas dos clases, por ejemplo:

\begin{itemize}
    \item invalido: “cualquierCadenaInvalida“ \(\longrightarrow\) resultado esperado: ERROR
    \item válido: “lunes“ \(\longrightarrow\) resultado esperado: “martes“
\end{itemize}

Vemos como podemos siguiendo esta metodología podemos dar una garantía. Explicar metodológicamente porqué se han hecho los tests y los valores escogidos y que covertura se está ofreciendo. Ahora si aplicamos la técnica de los valores límites sabemos que el “domingo“ es el último día y se debe volver al “lunes“. Es decir, es el valor límite dentro de nuestra clase de equivalencia. Por lo tanto si queremos garantizar y elegir con critero dentro de los casos de nuestra clase de equivalencia los tests quedarían:

\begin{itemize}
    \item invalido: “cualquierCadenaInvalida“ \(\longrightarrow\) resultado esperado: ERROR
    \item válido limite inicial: “lunes“ \(\longrightarrow\) resultado esperado: “martes“
    \item válido limite final: “domingo“ \(\longrightarrow\) resultado esperado: “lunes“
\end{itemize}

\textbf{Variables dependientes}

El método de pairwise, también conocido como pruebas de combinaciones de pares, es una técnica de diseño de pruebas que permite reducir el número de casos de prueba necesarios para lograr una cobertura exhaustiva de las combinaciones posibles de parámetros en un sistema o aplicación. En lugar de probar todas las combinaciones posibles de los parámetros, el método de pairwise identifica las combinaciones de pares que tienen el potencial de causar problemas o errores y las incluye en los casos de prueba. Estas combinaciones de pares se seleccionan utilizando un algoritmo que busca minimizar el número de casos de prueba necesarios sin sacrificar la calidad de la cobertura de pruebas.

El método de pairwise fue introducido por David C. Kuhn en un artículo titulado~\cite{37051} “Practical Combinatorial Testing“ publicado en 1997. Desde entonces, ha sido ampliamente utilizado en la industria de software y ha sido objeto de numerosos estudios y mejoras por parte de investigadores y practicantes.

Entre los beneficios del método de pairwise se encuentra la reducción del número de casos de prueba necesarios para una cobertura exhaustiva, lo que puede ahorrar tiempo y recursos. Sin embargo, es importante tener en cuenta que el método de pairwise no es una técnica infalible y que puede no detectar ciertos tipos de errores o problemas en el sistema o aplicación bajo prueba.

En conclusión, el método de pairwise es una técnica efectiva para la reducción del número de casos de prueba necesarios para lograr una cobertura exhaustiva de las combinaciones posibles de parámetros en un sistema o aplicación. Su aplicación puede mejorar la eficiencia y efectividad del proceso de pruebas, aunque debe ser utilizado en conjunto con otras técnicas de diseño de pruebas para lograr una cobertura completa.

un ejemplo extraido de~\cite{Bach04pairwisetesting} lo ejemplifica perfectamente. En un software con un menú que tenga doce botones para activar o desactivar tendremos 4096 casos diferentes que testear, para un software comercial, semejante coste de calidad es simplemente inasumible. y tal y como se menciona ¨Pairwise testing normally begins by selecting values for the system’s input variables. These individual values are often selected using domain partitioning. The values are then permuted to achieve coverage of all the pairings. This is very tedious to do by hand. Practical techniques used to create pairwise test sets include Orthogonal Arrays¨\cite{Bach04pairwisetesting} En este mismo paper reducen el caso tipico de menú de un programa de 12,288 a 10 casos con este método. Es una metodología para escoger los tests que más aportan valor de cara a la seguridad sin perder pragmatismo y economía. Para el cálculo de los pares hay varios software disponible como páginas web que permiten el cálculo de forma rápida.

Con estos métodos vamos a realizar pruebas que cumplan con las características de:

\begin{itemize}
    \item inocuas
    \item Automatizadas
    \item autoverificables
    \item repetibles
    \item independientes
    \item rápidas
    \item vector de pares
\end{itemize}

Con respecto al punto de inocuas significa que no introducen nuevos errores: para desarrollar el test tenemos que modificar el código no estamos siendo inocuos. Se requiere un diseño correcto y no es trivial. Un código se demuestra que es correcto cuando es fácil de testear de ahí que el enfoque TDD sea diseñar primero el test y luego el código, para conseguir esto se requiere de muchísima experiencia. En el desarrollo de software las veces que se puede hacer un TDD puro son escasas y por lo tanto por lo menos tenemos que tener claro que si a la hora de diseñar un test se complica el aislar la característica a testear, definir los factores y las clases de equivalencia, o nos salen un número inabarcable de tests estamos ante un código mejorable.


