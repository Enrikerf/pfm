\textbf{Dominio}

\begin{figure}[H]
    \setlength{\DTbaselineskip}{10pt}
    \DTsetlength{0.2em}{1em}{0.2em}{0.4pt}{1.6pt}
    \dirtree{%
        .1 Domain.
            .2 Step.
                .3 StepVo.
                .3 ResultVo.
                .3 Repository.
                    .4 consoleWrite.
                .3 Services.
                    .4 UnaryExecutor.
                    .4 ClientStreamExecutor.
                    .4 ServerStreamExecutor.
                    .4 BidirectionalExecutor.
            .2 EnginePidController.
                .3 EnginePidController.
                .3 Service.
                    .4 Configurator.
                    .4 RPMControllerActivator.
                    .4 PositionControllerActivator.
                    .4 ControlDisabler.
            .2 Engine.
                .3 EngineVo.
                .3 Repository.
                    .4 Find.
            .2 Encoder.
                .3 Encoder.
            .2 Pin.
                .3 EncoderPin.
                    .4 EncoderPinInterface: actuan como repositorios.
                .3 OutPin.
                    .4 OutPinInterface: actuan como repositorios.
                .3 PWMPin.
                    .4 PWMPinInterface: actuan como repositorios.
            .2 Physic.
                .3 Angle.
                .3 LinearSpeed.
                .3 AngularSpeed.
                .3 Frequency.
                .3 Duty.
    }
    \caption{Client+Control-Estructura de carpetas de Dominio}\label{fig:Client+Control-Estructura de carpetas de Dominio}
\end{figure}

\textbf{Aplicación}

\begin{figure}[H]
    \setlength{\DTbaselineskip}{10pt}
    \DTsetlength{0.2em}{1em}{0.2em}{0.4pt}{1.6pt}
    \dirtree{%
        .1 Application.
            .2 Port.
                .3 in.
                    .4 Step.
                        .5 ExecuteUnary.
                            .6 Command.
                            .6 UseCase.
                        .5 ExecuteServerStream (Command).
                        .5 ExecuteClientStream (Command).
                        .5 ExecuteBidi (Command).
                    .4 Engine.
                        .5 Configure.
                            .6 Command.
                            .6 UseCase.
                        .5 EnableRpmControl (Command).
                        .5 EnablePositionControl (Command).
                        .5 Disable (Command).
    }
    \caption{Client+Control-Estructura de carpetas de Aplicación}\label{fig:Client+Control-Estructura de carpetas de Aplicación}
\end{figure}

\textbf{Adapters}

\begin{figure}[H]
    \setlength{\DTbaselineskip}{10pt}
    \DTsetlength{0.2em}{1em}{0.2em}{0.4pt}{1.6pt}
    \dirtree{%
        .1 Adapter.
            .2 in.
                .3 GRPC.
                    .4 Harán uso de los useCases de aplicación cuando llegue una request RPC.
                .3 Console.
                    .4 Por ejemplo si quisieramos ejecutar los casos de uso mediante terminal.
            .2 out.
                .3 console.
                    .4 implementacion de consoleWrite para ejecutar los comandos en el sistema y obtener los resultados.
                .3 pin.
                    .4 implementación con librerías para actuar sobre los pines, Se buscará una librería que pueda ser configurada para actuar sobre pines de varias plataformas.
    }
    \caption{Client+Control-Estructura de carpetas de Infraestructura}\label{fig:Client+Control-Estructura de carpetas de Infraestructura}
\end{figure}