
Se ha diseñado y desarrollado un sistema capaz de gestionar y ejecutar tareas en sistemas remotos.
Como ejemplo de aplicación remota, el sistema consta también de un programa de control PID para un motor de corriente continua.
El objetivo principal es obtener un ejemplo práctico de la aplicación de una metodología que abarca el proceso de desarrollo de software desde su diseño hasta su entrega.
La metodología se basa en un diseño estructurado en capas, separando la solución tecnológica concreta de la lógica que resuelve el problema.
Se hace un esfuerzo de diseño para utilizar un lenguaje expresivo que transmita el problema que resuelve, de tal forma que haya una comunicación efectiva en todos los participantes del proyecto.

El diseño va enfocado a manejar de forma efectiva el principal problema de desarrollo del software: la evolución.
La evolución y el cambio del software requiere de la interacción entre personas con diferentes conocimientos: técnicos, gente de negocio, clientes y todo aquel al que le afecte. Personas que comienzan a interactuar con el sistema sin conocimientos previos o gente que deja de interactuar con el mismo eliminando conocimiento del mismo.
También intervienen cambios tecnológicos que no deberían afectar al problema que resuelve. Por ejemplo, un cambio en el sistema de comunicación, cambiar el uso del correo electrónico a mensajería instantanea no debiera suponer un peligro o incertidumbre a la estabilidad de ejecución del sistema.

Se ha adaptado un documento de proyecto de ejecución de tipo industrial para documentar el primer paso del proceso: el diseño.
Para el segundo paso se adapta un documento de ejecución donde ha descrito el proceso de desarrollo.
Como resultado la estructura del presente documento consta de los siguientes puntos:

\begin{itemize}
    \item Proyecto ejecutivo
    \begin{itemize}
        \item Memoria descriptiva
        \begin{itemize}
            \item Información previa
            \item Descripción del proyecto
            \item Prestaciones
        \end{itemize}
    \end{itemize}
    \begin{itemize}
              \item Memoria constructiva
              \begin{itemize}
                  \item Trabajos previos
                  \item Infraestructura
              \end{itemize}
    \end{itemize}
    \item Ejecución
    \begin{itemize}
        \item Desarrollo y seguimiento
        \item Final de obra
        \item Memoria explicativa de cambios
    \end{itemize}
    \item Conclusiones
\end{itemize}

El presente trabajo evalua el uso de Golang como lenguaje de programación enfocado a la gestión de proyectos comerciales. En particular enfocado a soluciones que requieran de la interacción con actuadores o hardware para el control automático.

Se ha llevado a cabo el diseño y ejecución de un proyecto que abarca las complejidades a las que se enfrenta un proyecto de desarrollo de software desde su concepción hasta su entrega. Si bien no es un objetivo tratar toda la profundidad que puede encarar cualquiera de los puntos, si cumple su objetivo de presentar y describir cada uno de ellos. Permitiendo evaluar los aspectos positivos y negativos de Golang para dicho fin. Se han puesto a prueba los conceptos clave que el lenguaje promueve: asincronía y gestión de concurrencia y simplicidad.

El sistema diseñado consta de un servidor que gestiona tareas que hemos llamado Manager. Dichas tareas son comandos que pueden ser ejecutados en servidores remotos que llamamos Clients. En particular uno de los comandos desarrollados ejecutará un programa de control PID que actua sobre un motor de corriente continua.

\begin{itemize}
    \item En el Manager se ha puesto a prueba la facilidad de implementar arquitecturas estándar para la gestión de datos, así como la comunicación con sistemas clientes a través de APIs. Se centra en la implementación de una arquitectura hexagonal, con el fin de evaluar los conceptos generales del lenguaje: gestión del protocolos de comunicación en tiempo real, enrutamiento, inyección de dependencias y estruccturación del código.
    \item En el software cliente, se ha puesto a prueba la versatilidad de Golang para ser compilado en diferentes plataformas sin la necesidad de una máquina virtual. Arquitecturas típicas de PC personal, como Intel, AMD o ARM para ejecutarlo en una Raspberry.
    \item  En la implementación del programa de control automático, se ha puesto a prueba la versatilidad del lenguaje para desarrollar algoritmos de forma limpia y mantenible.
\end{itemize}

El resultado final es un sistema que contiene los elementos típicos de una solución comercial: un sistema web API y servicios intercomunicados a través del mismo, un frontal y servicios clientes.


