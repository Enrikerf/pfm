
Se ha diseñado y desarrollado un sistema capaz de gestionar y ejecutar tareas en sistemas remotos.
Como ejemplo de aplicación remota, el sistema consta también de un programa de control PID para un motor de corriente continua.
Respondiendo al objetivo principal se ha obtenido un ejemplo práctico de aplicación de una metodología que abarca el proceso de desarrollo de software, desde su diseño hasta su entrega.
La metodología se basa en un diseño estructurado en capas, separando la solución tecnológica concreta de la lógica que resuelve el problema, facilitando la evolución iterativa del software.
Se ha realizado un esfuerzo de diseño para utilizar un lenguaje expresivo que transmite el problema resuelto.
Dicho lenguaje facilita la comunicación efectiva en todos los participantes del proyecto.
El sistema resultante consta de tres programas independientes y una plataforma hardware sobre la que actúan:

\begin{itemize}
    \item Un programa que gestiona la interacción entre los distintos sistemas y los usuarios.
    \item Un programa cliente que responde a las ordenes de ejecución de las tareas definidas en el gestor.
    Ejecutable en paralelo en tantas máquinas como se requiera.
    \item Un programa de control PID para ser ejecutado por el cliente.
    \item Un montaje sobre una placa de pruebas que consta de un puente H y un motor de corriente continua para ser controlado por el control PID\@.
\end{itemize}

El diseño está enfocado en manejar de forma efectiva el principal problema de desarrollo del software: la evolución.
La evolución y el cambio del software requiere de la interacción entre personas con diferentes conocimientos: técnicos, comerciales, financieros, clientes y todo aquel afectado.
Personas que comienzan a interactuar con el sistema sin conocimientos previos o gente que deja de interactuar con el mismo eliminando conocimiento del mismo.
También intervienen cambios tecnológicos que no deberían afectar al problema que resuelve.
Por ejemplo, un cambio en el sistema de comunicación, cambiar el uso del correo electrónico a mensajería instantánea no debe suponer un peligro o incertidumbre a la estabilidad de ejecución del sistema.

El presente trabajo evalúa cómo responde Golang ante este tipo de proyectos y metodologías.
En particular enfocado a soluciones que requieren de la interacción con actuadores o hardware para el control automático.
Se han puesto a prueba los conceptos clave que el lenguaje promueve: asincronía y gestión de concurrencia y simplicidad.

Se ha realizado el ejercicio de adaptación de los documentos utilizados en un proyecto de ejecución de tipo industrial a un proyecto de software para evaluar su encaje.
Se ha documentado el primer paso del proceso, el diseño, en un documento de proyecto ejecutivo.
Para el segundo paso, el desarrollo, se adapta un documento de ejecución de obra.


