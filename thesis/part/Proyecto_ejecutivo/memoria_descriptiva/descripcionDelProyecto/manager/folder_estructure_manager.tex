En el proyecto constará de cuatro carpetas principales. Donde podemos ver las 3 capas y Bootstrap que será donde hagamos la composición de la aplicación, es decir la inyección de dependencias.

\dirtree{%
    .1 Project .
        .2 Domain.
        .2 Application.
        .2 Adapter.
        .2 Bootstrap.
}

\textbf{Dominio}

Cada \textit{Agreggate Root} tiene asociada una carpeta como podemos ver en la ~\cref{fig:ManagerEstructura de carpetas de Dominio}. En core tenemos la base de la cual extienden determinadas cosas que son comunes en toda a aplicación como los Errores o los eventos. La lógica común.

\begin{figure}[H]
    \setlength{\DTbaselineskip}{10pt}
    \DTsetlength{0.2em}{1em}{0.2em}{0.4pt}{1.6pt}
%    \DTsetlength{1em}{3em}{0.1em}{1pt}{4pt}
\dirtree{%
    .1 Domain.
        .2 Core.
            .3 \textcolor{blue}{Id.go}.
            .3 \textcolor{blue}{Event.go}.
            .3 \textcolor{blue}{Error.go}.
        .2 Task.
            .3 \textcolor{blue}{Id.go}.
            .3 \textcolor{blue}{Task.go}.
            .3 Host.
                .4 \textcolor{blue}{InvalidHostError.go}.
                .4 \textcolor{blue}{Host.go}.
            .3 Port.
                .4 \textcolor{blue}{InvalidPortError.go}.
                .4 \textcolor{blue}{Port.go}.
            .3 CommunicationMode.
                .4 \textcolor{blue}{InvalidCommunicationModeError.go}.
                .4 \textcolor{blue}{Port.go}.
            .3 ExecutionMode.
                .4 \textcolor{blue}{InvalidExecutionModeError.go}.
                .4 \textcolor{blue}{ExecutionMode.go}.
            .3 Status.
                .4 \textcolor{blue}{InvalidStatusError.go}.
                .4 \textcolor{blue}{Status.go}.
            .3 Step.
                .4 \textcolor{blue}{Id.go}.
                .4 \textcolor{blue}{StepEntity.go}.
                .4 \textcolor{blue}{StepVo.go}.
                .4 Repository.
                    .5 \textcolor{blue}{Find.go}.
                    .5 \textcolor{blue}{Save.go}.
                    .5 \textcolor{blue}{Search.go}.
                    .5 \textcolor{blue}{Delete.go}.
                    .5 \textcolor{blue}{Update.go}.
                .4 Service.
                    .5 \textcolor{blue}{Finder.go}.
                    .5 \textcolor{blue}{Creator.go}.
                    .5 \textcolor{blue}{Updater.go}.
                    .5 \textcolor{blue}{Eraser.go}.
                    .5 \textcolor{blue}{Searcher.go}.
            .3 Repository.
                .4 \textcolor{blue}{Find.go}.
                .4 \textcolor{blue}{Save.go}.
                .4 \textcolor{blue}{Search.go}.
                .4 \textcolor{blue}{Delete.go}.
                .4 \textcolor{blue}{Update.go}.
            .3 Service.
                .4 \textcolor{blue}{Finder.go}.
                .4 \textcolor{blue}{Creator.go}.
                .4 \textcolor{blue}{Updater.go}.
                .4 \textcolor{blue}{Eraser.go}.
                .4 \textcolor{blue}{Searcher.go}.
        .2 Result.
            .3 \textcolor{blue}{ResultId.go}.
            .3 \textcolor{blue}{BatchId.go}.
            .3 \textcolor{blue}{Result.go}.
            .3 \textcolor{blue}{Batch.go}.
            .3 Repository.
                .4 "...".
            .3 Service.
                .4 "...".
}
    \caption{Manager: Estructura de carpetas de Dominio}\label{fig:ManagerEstructura de carpetas de Dominio}
\end{figure}

\textbf{Aplicación}

\begin{figure}[H]
\setlength{\DTbaselineskip}{10pt}
\DTsetlength{0.2em}{1em}{0.2em}{0.4pt}{1.6pt}
\dirtree{%
.1 Application.
    .2 Port.
        .3 in.
            .4 Task.
                .5 LoopExecutorCommand.
                    .6 \textcolor{blue}{Command.go}.
                    .6 \textcolor{blue}{UseCase.go}.
                .5 ManuallyExecutorCommand.
                    .6 \textcolor{blue}{Command.go}.
                    .6 \textcolor{blue}{UseCase.go}.
                .5 CreateCommand.
                    .6 \textcolor{blue}{Command.go}.
                    .6 \textcolor{blue}{UseCase.go}.
                .5 UpdateCommand.
                    .6 \textcolor{blue}{Command.go}.
                    .6 \textcolor{blue}{UseCase.go}.
                .5 DeleteCommand.
                    .6 \textcolor{blue}{Command.go}.
                    .6 \textcolor{blue}{UseCase.go}.
                .5 GetQuery.
                    .6 \textcolor{blue}{Query.go}.
                    .6 \textcolor{blue}{UseCase.go}.
                .5 ListQuery.
                    .6 \textcolor{blue}{Query.go}.
                    .6 \textcolor{blue}{UseCase.go}.
                .5 TaskEvent.
                    .6 \textcolor{blue}{Handler.go}.
            .4 Batch.
                .5 CreateCommand.
                .5 DeleteCommand.
                .5 GetQuery.
                .5 ListBatchesOfATaskQuery.
                .5 ListResultsOfABatchQuery.
        .3 out.
            .4 GRPC.
                .5 Unary .
                .5 ServerStream.
                .5 ClientStream.
                .5 Bidirectional.
}
    \caption{Manager: Estructura de carpetas de Aplicación}\label{fig:ManagerEstructura de carpetas de Aplicación}
\end{figure}


\textbf{Adapters}

En este punto tenemos el mayor grado de incertidumbre, no conocemos el contenido que vamos a tener que desarrollar, pero si podemos expresar la organización básica de lo que se va a requerir.

\begin{figure}[H]
    \setlength{\DTbaselineskip}{10pt}
    \DTsetlength{0.2em}{1em}{0.2em}{0.4pt}{1.6pt}
\dirtree{%
    .1 Adapter.
        .2 in.
            .3 GRPC.
                .4 Harán uso de los useCases de aplicación cuando llegue una request RPC.
        .2 out.
            .3 Mysql.
                .4 Implementación de los repository de cada entidad.
            .3 GRPC.
                .4 implementación de los repository de llamada a los servidores clientes.
}
    \caption{Manager: Estructura de carpetas de Infraestructura}\label{fig:ManagerEstructura de carpetas de Infraestructura}
\end{figure}
