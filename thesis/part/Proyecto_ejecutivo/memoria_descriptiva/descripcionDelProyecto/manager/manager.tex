\begin{figure}[H]
    \centering
    \includegraphics[height=0.3\textheight]{./part/Proyecto_ejecutivo/memoria_descriptiva/descripcionDelProyecto/manager/uml/managerServerConcept}
    \caption{Diagrama UML de despliegue del programa Manager}\label{fig:diagrama_uml_de_despliegue_del_manager}
\end{figure}

El diagrama UML de despliegue de la ~\cref{fig:diagrama_uml_de_despliegue_del_manager} muestra que el programa Manager será el programa con el que interactúe directamente el usuario.
Podrá insertar los comandos a ejecutar en una dirección de servidor determinada que habrá de contener el programa cliente \textit{Client}.
Dispondrá de una base de datos para almacenar de forma persistente la información.

\subparagraph{Dominio}

El diagrama UML de la~\cref{fig:Diagrama UML de el Dominio de manager} muestra que el Dominio de Manager se compone de tres módulos: uno para las tareas, Task;
otro para los resultados, Result;
y otro Core, que contiene los elementos básicos del sistema que comparte todo el Dominio:

\begin{itemize}
    \item CoreDomain
    \begin{itemize}
        \item \textbf{Id}: todos los ids extienden de el, para no reescribir código.
        contiene un \textit{UUID}, el paquete para generarlos habrá que seleccionarlo según los que ofrece el ecosistema.
        Es la única dependencia externas existente dentro del Dominio por ello se encapsula en el  paquete core para no dispersar a lo largo de todo el Domino dicha dependencia, por si hubiese que cambiarla.
        \item \textbf{Event} todos los eventos del sistema extenderán de él.
    \end{itemize}
    \item TaskDomain
    \begin{itemize}
        \item \textbf{Task}: es el \textit{Aggregate Root} que engloba todos los componentes que definen una tarea.
        \item \textbf{TaskId}: cada Entidad tiene su Id tipo, para evitar errores.
        Evitando el uso directo del Id de core.
        En la búsqueda de una Task mediante un Id que corresponde a un Result estaría correctamente tipado pero sería incorrecto.
        No se encontraría y el sistema no advertiría del error.
        \item \textbf{Host}: es un \textit{Value Object} compuesto por el valor del host, es un string pero el \textit{Value Object} garantiza que es un valor válido, si no devuelve un error
        \item \textbf{Port}: es un \textit{Value Object} compuesto por el valor del puerto, es un string pero el \textit{Value Object} garantiza que es un valor válido, si no devuelve un error
        \item \textbf{CommunicationMode}: es un Enum para expresar esa tarea de las formas de comunicación posibles que hay entre dos servidores cual sera la utilizada
        \begin{itemize}
            \item UNARY
            \item SERVER\_STREAM
            \item CLIENT\_STREAM
            \item BIDIRECTIONAL
        \end{itemize}
        \item \textbf{ExecutionMode}: es un \textit{Enum} o enumeración para expresar los modos de ejecución.
        \begin{itemize}
            \item MANUAL
            \item AUTOMATIC
        \end{itemize}
        \item \textbf{Status}: es un \textit{Enum} que define los estados en los que se puede encontrar una tarea.
        \begin{itemize}
            \item PENDING
            \item RUNNING
            \item SUCCESSFUL
            \item FAILED
        \end{itemize}
        \item \textbf{Step}: cada tarea puede componerse en distintos pasos.
        Por ejemplo poner en marcha el motor durante 15 segundos y luego llevarlo a una posición de inicio.
        \begin{itemize}
            \item \textbf{StepId}
            \item \textbf{sentence}: string de contenido libre que el servidor que contiene el programa cliente ejecutará.
            Depende del gestor del servidor de él dicho programa que el programa cliente quiere ejecutar se encuentre instalado
        \end{itemize}
        \item \textbf{TaskCreatedEvent}: cuando se cree una tarea se emitirá un evento, habrá un manejador de eventos que actuará en consecuencia.
        Si es una tarea automatizada y el loop de ejecución está parado lo pondrá en marcha.
        \item \textbf{TaskModifiedEvent}: evento emitido cuando una tarea es modificada habrá un manejador de eventos que actuará en consecuencia.
        Si la tarea vuelve a ser puesta a pending, es automatizada y el loop de ejecución está parado lo pondrá en marcha.
    \end{itemize}
    \item \textbf{ResultDomain}
    \begin{itemize}
        \item \textbf{BatchId}
        \item \textbf{Batch}: \textit{Aggregate Root} para referencial el conjunto de resultados correspondiente a la ejecución de una tarea.
        Una tarea no sólo se puede ejecutar varias veces, cada vez con distintos resultados, si no que además para una ejecución pueden obtenerse también varios resultados.
        Para identificarlos y agruparlos se hará uso de este \textit{Aggregate Root}.
        \item \textbf{ResultId}
        \item \textbf{Result}: String con el resultado en plano.
        Cada tarea y proceso tendrá a su disposición este strings para crear su propio formato de respuesta.
        Será tarea de cada cliente interpretarlos.
    \end{itemize}
\end{itemize}

\begin{figure}[H]
    \centering
    \includegraphics[height=0.5\textheight]{./part/Proyecto_ejecutivo/memoria_descriptiva/descripcionDelProyecto/manager/uml/managerDomain}
    \caption{Diagrama UML de el Dominio del programa Manager}\label{fig:Diagrama UML de el Dominio de manager}
\end{figure}

\subparagraph{Casos de uso}

En el Manager existirán dos \textit{Entities} y dos \textit{Aggregate Root}.
Las interacciones que se definan sobre ellos serán los casos de uso disponibles para el usuario.
Se dispondrá del \gls{CRUD} completo de \textit{Task}.
Dentro de éste, se contempla la edición de los \textit{Steps} a través de Task ya que es su \textit{Aggregate Root}.
Por definición no se debe poder acceder a sus componentes si no es a través del \textit{Aggregate Root}.
De \textit{Batch} y \textit{Result} no estará permitida la actualización por lógica de Dominio.
Este caso de uso no debe existir, ya que sus variables son auto-generadas en el momento de creación.

\begin{itemize}
    \item  \textbf{Create Task}: La lógica de crear tareas se basará en la comprobación de que la información de entrada es correcta y suficiente, es decir, no hay parámetros opcionales.
    Por lo tanto deberán estar todos y ser validos.
    Esto se hará en la instanciación de los \textit{Value Object}.
    Si alguno de los datos no es válido el flujo se detendrá, devolviendo el error correspondiente al usuario para que pueda subsanarlo y enviar una nueva petición.

    \begin{figure}[H]
        \centering
        \includegraphics[height=0.2\textheight]{./part/Proyecto_ejecutivo/memoria_descriptiva/descripcionDelProyecto/manager/uml/createTaskUseCase}
        \caption{Use Case: Create Task}\label{fig:Use Case-Create Task}
    \end{figure}

    \item \textbf{Read Task}: La obtención de una Task se hará a través de su identificador \textit{UUID}.
    Se comprobará que efectivamente existe una Task con dicho identificador y de se devolverá el resultado de dicha búsqueda.

    \begin{figure}[H]
        \centering
        \includegraphics[height=0.15\textheight]{./part/Proyecto_ejecutivo/memoria_descriptiva/descripcionDelProyecto/manager/uml/getTaskUseCase}
        \caption{Use Case: Read Task}\label{fig:Use Case-Read Task}
    \end{figure}

    \item \textbf{Read Tasks}: Uno de los puntos más amplios en una API CRUD es el filtrado de datos.
    No entra dentro del ámbito de este proyecto crear un sistema de filtrado que incluya la paginación.
    En este caso habría que crear una nomenclatura de filtros de cara al usuario y un sistema que los procese, devolviendo error ante un filtro erróneo o el listado de tareas que responda a dicho filtro.
    En este sistema se devolverán todas las tareas existentes.

    \item \textbf{Update Task}: A diferencia del flujo de creación en este caso si ha parámetros opcionales.
    Ya que este flujo equivale a lo que en \gls{REST} se entiende como PATCH, es decir una actualización parcial, los parámetros que no se reciban en la petición quedarán intactos.
    En un PUT el parámetro que no se recibe se entiende que ha de ser eliminado.

    \begin{figure}[H]
        \centering
        \includegraphics[height=0.4\textheight]{./part/Proyecto_ejecutivo/memoria_descriptiva/descripcionDelProyecto/manager/uml/updateTaskUseCase}
        \caption{Use Case: Update Task}\label{fig:Use Case-Update Task}
    \end{figure}

    \item \textbf{Delete Task}: En el flujo de eliminación de tareas, además de confirmar que la tarea que se desea eliminar existe, se añadirán más validaciones para garantizar al consistencia del sistema.
    No se permitirán eliminar tareas que estén en status RUNNING\@.

    \begin{figure}[H]
        \centering
        \includegraphics[height=0.2\textheight]{./part/Proyecto_ejecutivo/memoria_descriptiva/descripcionDelProyecto/manager/uml/deleteTaskUseCase}
        \caption{Use Case: Delete Task}\label{fig:Use Case-Delete Task}
    \end{figure}

    \item \textbf{TaskEventHandler}: Tanto en \textit{TaskCreatedEvent} como \textit{TaskUpdatedEvent} el handler va a ser el mismo.
    El gestor de eventos~\cref{fig:Use Case-TaskEventHandler} comprobará si la tarea es de tipo AUTOMATIC, o se ha modificado a ese tipo.
    Comprobará si se acaba de crear, y por lo tanto el status es PENDING, o se ha modificado a dicho tipo.
    Si las dos primeras condiciones se cumplen entonces comprobará que el loop de gestión de tareas está habilitado;
    de no ser así se habilitará.

    \begin{figure}[H]
        \centering
        \includegraphics[height=0.2\textheight]{./part/Proyecto_ejecutivo/memoria_descriptiva/descripcionDelProyecto/manager/uml/taskEventHandlerUseCase}
        \caption{Use Case: TaskEventHandler}\label{fig:Use Case-TaskEventHandler}
    \end{figure}

    \item \textbf{TaskLoop: ejecutar tareas automáticas}\label{ref:X}: Cuando \textit{TaskEventHandler} pone en marcha el \textit{TaskLoop} este obtendrá de base de datos todas las tareas automáticas pendientes y iterando sobre ellas lanzará un hilo de ejecución para ejecutarlas.
    Una vez terminadas volverá a intentar obtener tareas pendientes ya que en el periodo que haya estado ejecutándolas pueden haberse introducido nuevas.
    Si no hay más tareas pendientes el loop se desactivará.

    Este caso de uso es el core de la aplicación, donde se pondrá en juego la asincronía y todo el potencial del lenguaje.
    Es por tanto una primera versión del diseño.
    Igual que los otros casos de uso son lógica rutinaria que no debiera diferir mucho del diseño original, en este caso sí se pueden encontrar con lógicas que no estén previstas y por lo tanto haya que modificar.
    Todo quedará reflejado en la memoria explicativa de cambios~\cref{subsec:memoria explicativa de cambios}.

    \begin{figure}[H]
        \centering
        \includegraphics[height=0.4\textheight]{./part/Proyecto_ejecutivo/memoria_descriptiva/descripcionDelProyecto/manager/uml/executeTaskLoop}
        \caption{Use Case: TaskLoop}\label{fig:Use Case-TaskLoop}
    \end{figure}

    Atendiendo por tanto a esta posibilidad de cambio se hará uso de la arquitectura para desacoplar dicha incertidumbre \textbf{TaskProcessor} que aparece en la figura~\cref{fig:Use Case-TaskLoop} en color verde es el proceso asíncrono donde se hará uso del las Gorutines que permiten la gestión de tareas asíncronas de forma sencilla.
    Se definirá como puerto de salida y se implementará un adaptador para encapsular dicha incertidumbre.

    Es de preveer que dentro de ese adaptador se implemente lógica que habrá de ser de negocio.
    Sin embargo, a no poder vislumbrarla con claridad, se deja pendiente de una refactorización.
    La refactorización posterior podrá ir introduciendo en Dominio y limpiando dicho adaptador de el posible mal código resultado de la investigación mientras se construye, en vez de ensuciar el Dominio y acoplar nuestro código base a esta incertidumbre.

    A pesar de dicho desconocimiento inicial que impide el diseño con exactitud del flujo, El objetivo de este proyecto ejecutivo es hacer siempre una primera iteración en la medida que se pueda.
    Por lo tanto una primera iteración se puede ver en la figura~\cref{fig:Use Case-Task Processor}.
    Se puede apreciar que está muy acoplado a la interacción de los flujos de comunicación RPC. Esto es el origen de la incertidumbre al no haber trabajado todavía con dicha tecnología.

    \begin{figure}[H]
        \centering
        \includegraphics[height=0.55\textheight]{./part/Proyecto_ejecutivo/memoria_descriptiva/descripcionDelProyecto/manager/uml/1-TaskProcessor}
        \caption{Use Case: Task Processor}\label{fig:Use Case-Task Processor}
    \end{figure}

    \item \textbf{Create Batch}

    \begin{figure}[H]
        \centering
        \includegraphics[height=0.2\textheight]{./part/Proyecto_ejecutivo/memoria_descriptiva/descripcionDelProyecto/manager/uml/createBatchUseCase}
        \caption{Use Case: Create Batch}\label{fig:Use Case: Create Batch}
    \end{figure}

    \item \textbf{Delete Batch}

    \begin{figure}[H]
        \centering
        \includegraphics[height=0.2\textheight]{./part/Proyecto_ejecutivo/memoria_descriptiva/descripcionDelProyecto/manager/uml/deleteBatchUseCase}
        \caption{Use Case: Delete Batch}\label{fig:Use Case: Delete Batch}
    \end{figure}

    \item \textbf{Read batch}

    \begin{figure}[H]
        \centering
        \includegraphics[height=0.2\textheight]{./part/Proyecto_ejecutivo/memoria_descriptiva/descripcionDelProyecto/manager/uml/getBatchUseCase}
        \caption{Use Case: Read Batch}\label{fig:Use Case: Read Batch}
    \end{figure}

    \item \textbf{Read batches of a Task}

    \begin{figure}[H]
        \centering
        \includegraphics[height=0.2\textheight]{./part/Proyecto_ejecutivo/memoria_descriptiva/descripcionDelProyecto/manager/uml/getTaskBatchUseCase}
        \caption{Use Case: Read Batches of Task}\label{fig:Use Case-Read Batches of Task}
    \end{figure}

    \item \textbf{Obtener los results de un batch}

    \begin{figure}[H]
        \centering
        \includegraphics[height=0.2\textheight]{./part/Proyecto_ejecutivo/memoria_descriptiva/descripcionDelProyecto/manager/uml/getBatchResultsUseCase}
        \caption{Read Results of a Batch Use Case}\label{fig:Read Results of a Batch Use Case}
    \end{figure}

    \item \textbf{Execute task manually}

    Solo se puede ejecutar una tarea manual por vez, en este primer diseño.
    Se quiere monitorizar en tiempo real la respuesta, por ejemplo para monitorizar la velocidad del motor en tiempo real, pero no se quiere exponer al sistema a la complejidad de monitorizar varios resultados a la vez.

    \begin{figure}[H]
        \centering
        \includegraphics[height=0.2\textheight]{./part/Proyecto_ejecutivo/memoria_descriptiva/descripcionDelProyecto/manager/uml/1-executeTaskManual}
        \caption{Use Case: Execute Task Manually}\label{fig:Use Case-Execute Task Manually}
    \end{figure}

    El diagrama de secuencia para el caso de que una tarea sea de tipo \textit{BIDIRECTIONAL} se puede apreciar en el diagrama~\cref{fig:executeTaskManualInteraction}.
    Se hará uso de esta lógica por ejemplo para el control en velocidad y posición del motor de corriente continua.
    Se enviará una consigna y se recibirán resultados hasta que el cliente envíe la orden de detener dicho control.
    Para otros tipos más simples, por ejemplo, ejecutar manualmente la tarea de configurar el PID, simplemente será de tipo \textit{UNARY}, se enviará una request y se obtendrá una respuesta.

    \begin{figure}[H]
        \centering
        \includegraphics[height=0.2\textheight]{./part/Proyecto_ejecutivo/memoria_descriptiva/descripcionDelProyecto/manager/uml/1-ExecuteTaskManuallyInteraction}
        \caption{Execute task manually interaction}\label{fig:executeTaskManualInteraction}
    \end{figure}
\end{itemize}


\subparagraph{Estructura de carpetas}

La definición de una estructura de carpetas que refleje el diseño del software constituye un aspecto fundamental de buenas prácticas en el desarrollo de proyectos informáticos.
Del mismo modo, los nombres de los archivos deben expresar su contenido para facilitar la documentación y depuración del proyecto.
Cada lenguaje, framework o comunidad tienen sus propios estándares.
En concreto, este trabajo adopta la propuestas por~\cite{TomHombergs2019GYHD}, que permite un desarrollo flexible, extensible y adaptable del software.
Este diseño expresa con claridad una arquitectura hexagonal, o de puertos y adaptadores y que el Dominio se encuentra estructurado mediante un diseño \textit{DDD} expresando el \textit{UL} en los propios nombres de los archivos.

En el proyecto constará de 4 carpetas principales
\dirtree{%
    .1 Project .
        .2 Adapter.
        .2 Application.
        .2 Domain.
        .2 Bootstrap.
}

Bootstrap será donde hagamos la composición de la aplicación, es decir la inyección de dependencias.

Vamos a desplegar la estructura capa por capa

\textbf{Adapters}

\dirtree{%
.1 Adapter.
    .2 in.
        .3 GRPC.
            .4 CreateEntityGrpcCall.
        .3 Console.
            .4 CreateEntityTerminalCommand.
    .2 out.
        .3 Email.
            .4 SendEmailOnCreationImplementation (*1).
        .3 Mysql.
            .4 SaveEntityImplementation (*2).
}

\textbf{Aplicación}

\dirtree{%
.1 Application.
    .2 Port.
        .3 in.
            .4 Entity.
                .5 CreateUseCase.
                    .6 CreateCommand.
                        .7 CreateCommand (using sendEmail param in this example).
                        .7 CreateUseCase.
                    .6 SomeEventHandler.
                        .7 CreationEvent.
                        .7 CreationEventUseCase.
        .3 out.
            .4 Email.
                .5 SendEmailConCreation (interface for *1).
}

\textbf{Dominio}

\dirtree{%
.1 Domain.
    .2 Core.
        .3 Id.
        .3 Event.
        .3 Error.
    .2 Task.
        .3 Id.
        .3 Task.
        .3 Host.
        .3 Port.
        .3 CommunicationMode.
        .3 ExecutionMode.
        .3 Status.
        .3 Step.
            .4 Id.
            .4 StepEntity.
            .4 StepVo.
            .4 Repository.
                .5 Find.
                .5 Save.
                .5 Search.
                .5 Delete.
                .5 Update.
            .4 Service.
                .5 Finder.
                .5 Creator.
                .5 Updater.
                .5 Eraser.
                .5 Searcher.
        .3 Repository.
            .4 Find.
            .4 Save.
            .4 Search.
            .4 Delete.
            .4 Update.
        .3 Service.
            .4 Finder.
            .4 Creator.
            .4 Updater.
            .4 Eraser.
            .4 Searcher.
    .2 Result.
}

