En la~\cref{fig:Control-Diagrama UML de despliegue} podemos ver que el programa de control tiene acceso directo al hardware. Ejecutará los comandos recibidos por el programa cliente y actuará en consecuencia sobre el motor de corriente continua.

%Casos de uso todas las funciones del engine y diseñarlo para que la infra sea a través de consola de comandos (preparando para que luego debido a la comunicación rpc devolver los datos sea muy complicado si lo dividimos en otro comando y decidimos absorver este dominio)
%
%El sistema está pensado para que el programa a ejecutar sea de libre decisión del cliente. pero vamos a aprovechar la oportunidad para crear un programa de control y profundizar tanto en el uso del lenguaje como en los conocimientos relacionados con este master. En concreto el control automático. Uno do los programas que podrá ejecutar será un control PID.

\begin{figure}[H]
    \centering
    \includegraphics[height=0.2\textheight]{./part/Proyecto_ejecutivo/memoria_descriptiva/descripcionDelProyecto/control/uml/controlConcept}
    \caption{Control: Diagrama UML de despliegue}\label{fig:Control-Diagrama UML de despliegue}
\end{figure}

\subparagraph{Dominio}

\begin{figure}[H]
    \centering
    \includegraphics[height=0.5\textheight]{./part/Proyecto_ejecutivo/memoria_descriptiva/descripcionDelProyecto/control/uml/controlDomain}
    \caption{Control: Diagrama UML de el dominio}\label{fig:Control-Diagrama UML de el dominio}
\end{figure}

\begin{itemize}
    \item PhysicModule: contendrá todas las variables físicas, son interfaces de int y floats que usaremos para garantizar la congruencia en las operaciones. Aunque los tipos sean float hay variables que no pueden operar entre sí sin atender a su realidad física, es decir sus unidades.
    \item PinModule: para abstraernos del sistema donde se ejecute el código crearemos la abstracción de el hardware necesario para atender a interfaz física de un motor:
    \begin{itemize}
        \item EncoderPinModule: sabemos que vamos a necesitar un pin de entrada tipo pull de entrada y esperar a los flancos de subida que entrarán del encoder. Ya sea en los hardwares comerciales más conocidos como arduino o raspberry como cualquier otro.
        \item OutPinModule: para activar o desactivar frenos y triggers de dirección necesitaremos pin de salida
        \item PWMPinModule: para tener un control PWM necesitaremos un pin que nos permita salida pero con una frecuencia y un ciclo de trabajo.
    \end{itemize}
    \item EncoderModule: estará compuesto por dos pines de tipo Encoder para leer los pulsos y calcular si está en una dirección
    \item EngineModule: este modulo contendrá tanto la interfaz del motor como la implementación, donde haciendo uso de los pines calculará y contendrá las variables intrínsecas del motor.
    \item EnginePIDControllerModule: contendrá todo lo relativo al controlador PId. El controlador usará una instancia de Engine y una instancia de PIDAlgorithm
\end{itemize}

\subparagraph{Casos de uso}

Los casos de usos actuarán sobre El objeto de dominio EnginePIDController, que en este caso al no haber base de datos ni identificadores serán ValueObjects. Particularmente en nuestro programa solo instanciaremos uno. No implementaremos un patrón singleton sobre ellos debido a que podría desearse controlar varios motores a la vez si el hardware lo permitiera.

\textbf{Configure PID}

El primer paso a ejecutar antes de habilitar el control PID será sintonizarlo.

\begin{figure}[H]
    \centering
    \includegraphics[height=0.2\textheight]{./part/Proyecto_ejecutivo/memoria_descriptiva/descripcionDelProyecto/control/uml/configurePID}
    \caption{Use Case: Configure PID}\label{fig:Use Case-Configure PID}
\end{figure}

\newpage
\textbf{Enable RPM control}

Cuando se habilita el control rpm se genera una goRutine, es decir, si el hardware no devuelve error se responde OK pero un nuevo proceso que se quedará ejecutando hasta que se ejecute el caso de uso Disable Control. Este nuevo proceso ejecutará el control PID para mantener la consigna en el valor indicado. En este caso la consigna será la velocidad angular requerida.

\begin{figure}[H]
    \centering
    \includegraphics[height=0.6\textheight]{./part/Proyecto_ejecutivo/memoria_descriptiva/descripcionDelProyecto/control/uml/enableRPMControl}
    \caption{Use Case: Enable RPM control}\label{fig:Use Case- Enable RPM control}
\end{figure}

\newpage
\textbf{Enable Position control}

Para el control en posición se actuará análogamente al control de velocidad pero la consigna será el ángulo requerido.

\begin{figure}[H]
    \centering
    \includegraphics[height=0.4\textheight]{./part/Proyecto_ejecutivo/memoria_descriptiva/descripcionDelProyecto/control/uml/enablePositionControl}
    \caption{Use Case: Enable Position control}\label{fig:Use Case- Enable Position control}
\end{figure}

\textbf{Disable control}

\begin{figure}[H]
    \centering
    \includegraphics[height=0.2\textheight]{./part/Proyecto_ejecutivo/memoria_descriptiva/descripcionDelProyecto/control/uml/disableControl}
    \caption{Use Case: Disable control}\label{fig:Use Case- Disable control}
\end{figure}

\newpage
\subparagraph{Estructura de carpetas}


\begin{figure}[H]
    \setlength{\DTbaselineskip}{10pt}
    \DTsetlength{0.2em}{1em}{0.2em}{0.4pt}{1.6pt}
    \dirtree{%
        .1 Project .
            .2 Domain.
            .2 Application.
            .2 Adapter.
            .2 Bootstrap.
    }
    \caption{Control: Estructura de carpetas de proyecto}\label{fig:Control- Estructura de carpetas de proyecto}
\end{figure}

\textbf{Dominio}

La estructura de organización para el dominio de este programa mostrado en la~\cref{fig:Control- Estructura de carpetas de Dominio} muestra que en este caso se utiliza otro lenguaje para nombrar a los repositorios.
Siguiendo las guías de diseño de DDD y el UL se utiliza un lenguaje que exprese el problema que ocupa.
En el Manager y el Cliente enfrentan un paradigma de servicio web API\@.
Es por esto que nombrar repositorios a los servicios de infraestructura de salida tiene sentido.
En este caso la infraestructura de salida se compone de actuadores físicos.
Se usan por tanto los nombres que los definen como el Encoder y los pines.

\begin{figure}[H]
    \setlength{\DTbaselineskip}{10pt}
    \DTsetlength{0.2em}{1em}{0.2em}{0.4pt}{1.6pt}
    \dirtree{%
        .1 Domain.
            .2 EnginePidController.
                .3 \textcolor{blue}{EnginePidController.go}.
                .3 Service.
                    .4 \textcolor{blue}{Configurator.go}.
                    .4 \textcolor{blue}{RPMControllerActivator.go}.
                    .4 \textcolor{blue}{PositionControllerActivator.go}.
                    .4 \textcolor{blue}{ControlDisabler.go}.
            .2 Engine.
                .3 \textcolor{blue}{EngineVo.go}.
                .3 Repository.
                    .3 \textcolor{blue}{Find.go}.
            .2 Encoder.
                .3 \textcolor{blue}{Encoder.go}.
            .2 Pin.
                .3 EncoderPin.
                    .4 \textcolor{blue}{EncoderPinInterface.go}.
                .3 OutPin.
                    .4 \textcolor{blue}{OutPinInterface.go}.
                .3 PWMPin.
                    .4 \textcolor{blue}{PWMPinInterface.go}.
            .2 Physic.
                .3 \textcolor{blue}{Angle.go}.
                .3 \textcolor{blue}{LinearSpeed.go}.
                .3 \textcolor{blue}{AngularSpeed.go}.
                .3 \textcolor{blue}{Frequency.go}.
                .3 \textcolor{blue}{Duty.go}.
    }
    \caption{Control: Estructura de carpetas de Dominio}\label{fig:Control- Estructura de carpetas de Dominio}
\end{figure}

\textbf{Aplicación}

Para los casos de uso definidos en la capa de aplicación se se siguen el paradigma CQRS de la misma forma que en todos los programas.
La interacción con el entorno físico consta en este caso únicamente de comandos para actuar sobre el motor.

\begin{figure}[H]
    \setlength{\DTbaselineskip}{10pt}
    \DTsetlength{0.2em}{1em}{0.2em}{0.4pt}{1.6pt}
    \dirtree{%
        .1 Application.
            .2 Port.
            .3 in.
                .4 EnginePidController.
                    .5 ConfigureCommand.
                        .6 \textcolor{blue}{Command.go}.
                        .6 \textcolor{blue}{UseCase.go}.
                    .5 EnableRpmControlCommand.
                        .6 \textcolor{blue}{Command.go}.
                        .6 \textcolor{blue}{UseCase.go}.
                    .5 EnablePositionControlCommand.
                        .6 \textcolor{blue}{Command.go}.
                        .6 \textcolor{blue}{UseCase.go}.
                    .5 DisableCommand.
                        .6 \textcolor{blue}{Command.go}.
                        .6 \textcolor{blue}{UseCase.go}.
    }
    \caption{Control: Estructura de carpetas de Aplicación}\label{fig:Control-Estructura de carpetas de Aplicación}
\end{figure}

\textbf{Adapters}

\begin{figure}[H]
    \setlength{\DTbaselineskip}{10pt}
    \DTsetlength{0.2em}{1em}{0.2em}{0.4pt}{1.6pt}
    \dirtree{%
        .1 Adapter.
            .2 in.
                .3 GRPC.
                    .4 Harán uso de los useCases de aplicación cuando llegue una request RPC.
                .3 Console.
                    .4 Servirán para ejecutar los casos de uso mediante terminal.
            .2 out.
                .3 pin.
                    .4 implementación con librerías para actuar sobre los pines, Se investigará para hacer uso de una librería que pueda ser configurada para actuar sobre pines de varias plataformas.
    }
    \caption{Control: Estructura de carpetas de Infraestructura}\label{fig:Control-Estructura de carpetas de Infraestructura}
\end{figure}





