\begin{figure}[H]
    \setlength{\DTbaselineskip}{10pt}
    \DTsetlength{0.2em}{1em}{0.2em}{0.4pt}{1.6pt}
    \dirtree{%
        .1 Project .
            .2 Domain.
            .2 Application.
            .2 Adapter.
            .2 Bootstrap.
    }
    \caption{Control: Estructura de carpetas de proyecto}\label{fig:Control- Estructura de carpetas de proyecto}
\end{figure}

\textbf{Dominio}

\begin{figure}[H]
    \setlength{\DTbaselineskip}{10pt}
    \DTsetlength{0.2em}{1em}{0.2em}{0.4pt}{1.6pt}
    \dirtree{%
        .1 Domain.
            .2 EnginePidController.
                .3 \textcolor{blue}{EnginePidController.go}.
                .3 Service.
                    .4 \textcolor{blue}{Configurator.go}.
                    .4 \textcolor{blue}{RPMControllerActivator.go}.
                    .4 \textcolor{blue}{PositionControllerActivator.go}.
                    .4 \textcolor{blue}{ControlDisabler.go}.
            .2 Engine.
                .3 \textcolor{blue}{EngineVo.go}.
                .3 Repository.
                    .3 \textcolor{blue}{Find.go}.
            .2 Encoder.
                .3 \textcolor{blue}{Encoder.go}.
            .2 Pin.
                .3 EncoderPin.
                    .4 \textcolor{blue}{EncoderPinInterface.go}.
                .3 OutPin.
                    .4 \textcolor{blue}{OutPinInterface.go}.
                .3 PWMPin.
                    .4 \textcolor{blue}{PWMPinInterface.go}.
            .2 Physic.
                .3 \textcolor{blue}{Angle.go}.
                .3 \textcolor{blue}{LinearSpeed.go}.
                .3 \textcolor{blue}{AngularSpeed.go}.
                .3 \textcolor{blue}{Frequency.go}.
                .3 \textcolor{blue}{Duty.go}.
    }
    \caption{Control: Estructura de carpetas de Dominio}\label{fig:Control- Estructura de carpetas de Dominio}
\end{figure}

\textbf{Aplicación}

\begin{figure}[H]
    \setlength{\DTbaselineskip}{10pt}
    \DTsetlength{0.2em}{1em}{0.2em}{0.4pt}{1.6pt}
    \dirtree{%
        .1 Application.
            .2 Port.
            .3 in.
                .4 EnginePidController.
                    .5 ConfigureCommand.
                        .6 \textcolor{blue}{Command.go}.
                        .6 \textcolor{blue}{UseCase.go}.
                    .5 EnableRpmControlCommand.
                        .6 \textcolor{blue}{Command.go}.
                        .6 \textcolor{blue}{UseCase.go}.
                    .5 EnablePositionControlCommand.
                        .6 \textcolor{blue}{Command.go}.
                        .6 \textcolor{blue}{UseCase.go}.
                    .5 DisableCommand.
                        .6 \textcolor{blue}{Command.go}.
                        .6 \textcolor{blue}{UseCase.go}.
    }
    \caption{Control: Estructura de carpetas de Aplicación}\label{fig:Control-Estructura de carpetas de Aplicación}
\end{figure}

\textbf{Adapters}

\begin{figure}[H]
    \setlength{\DTbaselineskip}{10pt}
    \DTsetlength{0.2em}{1em}{0.2em}{0.4pt}{1.6pt}
    \dirtree{%
        .1 Adapter.
            .2 in.
                .3 GRPC.
                    .4 Harán uso de los useCases de aplicación cuando llegue una request RPC.
                .3 Console.
                    .4 Por ejemplo si quisieramos ejecutar los casos de uso mediante terminal.
            .2 out.
                .3 pin.
                    .4 implementación con librerías para actuar sobre los pines, Se buscará una librería que pueda ser configurada para actuar sobre pines de varias plataformas.
    }
    \caption{Control: Estructura de carpetas de Infraestructura}\label{fig:Control-Estructura de carpetas de Infraestructura}
\end{figure}