
\begin{figure}[H]
    \setlength{\DTbaselineskip}{10pt}
    \DTsetlength{0.2em}{1em}{0.2em}{0.4pt}{1.6pt}
    \dirtree{%
        .1 Project .
            .2 Domain.
            .2 Application.
            .2 Adapter.
            .2 Bootstrap.
    }
    \caption{Control: Estructura de carpetas de proyecto}\label{fig:Control- Estructura de carpetas de proyecto}
\end{figure}

\textbf{Dominio}

La estructura de organización para el dominio de este programa mostrado en la~\cref{fig:Control- Estructura de carpetas de Dominio} muestra que en este caso se utiliza otro lenguaje para nombrar a los repositorios.
Siguiendo las guías de diseño de DDD y el UL se utiliza un lenguaje que exprese el problema que ocupa.
En el Manager y el Cliente enfrentan un paradigma de servicio web API\@.
Es por esto que nombrar repositorios a los servicios de infraestructura de salida tiene sentido.
En este caso la infraestructura de salida se compone de actuadores físicos.
Se usan por tanto los nombres que los definen como el Encoder y los pines.

\begin{figure}[H]
    \setlength{\DTbaselineskip}{10pt}
    \DTsetlength{0.2em}{1em}{0.2em}{0.4pt}{1.6pt}
    \dirtree{%
        .1 Domain.
            .2 EnginePidController.
                .3 \textcolor{blue}{EnginePidController.go}.
                .3 Service.
                    .4 \textcolor{blue}{Configurator.go}.
                    .4 \textcolor{blue}{RPMControllerActivator.go}.
                    .4 \textcolor{blue}{PositionControllerActivator.go}.
                    .4 \textcolor{blue}{ControlDisabler.go}.
            .2 Engine.
                .3 \textcolor{blue}{EngineVo.go}.
                .3 Repository.
                    .3 \textcolor{blue}{Find.go}.
            .2 Encoder.
                .3 \textcolor{blue}{Encoder.go}.
            .2 Pin.
                .3 EncoderPin.
                    .4 \textcolor{blue}{EncoderPinInterface.go}.
                .3 OutPin.
                    .4 \textcolor{blue}{OutPinInterface.go}.
                .3 PWMPin.
                    .4 \textcolor{blue}{PWMPinInterface.go}.
            .2 Physic.
                .3 \textcolor{blue}{Angle.go}.
                .3 \textcolor{blue}{LinearSpeed.go}.
                .3 \textcolor{blue}{AngularSpeed.go}.
                .3 \textcolor{blue}{Frequency.go}.
                .3 \textcolor{blue}{Duty.go}.
    }
    \caption{Control: Estructura de carpetas de Dominio}\label{fig:Control- Estructura de carpetas de Dominio}
\end{figure}

\textbf{Aplicación}

Para los casos de uso definidos en la capa de aplicación se se siguen el paradigma CQRS de la misma forma que en todos los programas.
La interacción con el entorno físico consta en este caso únicamente de comandos para actuar sobre el motor.

\begin{figure}[H]
    \setlength{\DTbaselineskip}{10pt}
    \DTsetlength{0.2em}{1em}{0.2em}{0.4pt}{1.6pt}
    \dirtree{%
        .1 Application.
            .2 Port.
            .3 in.
                .4 EnginePidController.
                    .5 ConfigureCommand.
                        .6 \textcolor{blue}{Command.go}.
                        .6 \textcolor{blue}{UseCase.go}.
                    .5 EnableRpmControlCommand.
                        .6 \textcolor{blue}{Command.go}.
                        .6 \textcolor{blue}{UseCase.go}.
                    .5 EnablePositionControlCommand.
                        .6 \textcolor{blue}{Command.go}.
                        .6 \textcolor{blue}{UseCase.go}.
                    .5 DisableCommand.
                        .6 \textcolor{blue}{Command.go}.
                        .6 \textcolor{blue}{UseCase.go}.
    }
    \caption{Control: Estructura de carpetas de Aplicación}\label{fig:Control-Estructura de carpetas de Aplicación}
\end{figure}

\textbf{Adapters}

\begin{figure}[H]
    \setlength{\DTbaselineskip}{10pt}
    \DTsetlength{0.2em}{1em}{0.2em}{0.4pt}{1.6pt}
    \dirtree{%
        .1 Adapter.
            .2 in.
                .3 GRPC.
                    .4 Harán uso de los useCases de aplicación cuando llegue una request RPC.
                .3 Console.
                    .4 Servirán para ejecutar los casos de uso mediante terminal.
            .2 out.
                .3 pin.
                    .4 implementación con librerías para actuar sobre los pines, Se investigará para hacer uso de una librería que pueda ser configurada para actuar sobre pines de varias plataformas.
    }
    \caption{Control: Estructura de carpetas de Infraestructura}\label{fig:Control-Estructura de carpetas de Infraestructura}
\end{figure}