En el proyecto constará de 4 carpetas principales

\tiny
\dirtree{%
    .1 Project .
        .2 Domain.
        .2 Application.
        .2 Adapter.
        .2 Bootstrap.
}
\normalsize

\textbf{Dominio}

\begin{figure}[H]
    \tiny
\dirtree{%
    .1 Domain.
        .2 Step.
            .3 StepVo.
            .3 Repository.
                .4 consoleWrite.
            .3 Services.
                .4 Executor.
        .2 Result.
            .3 ResultVo.
}
\normalsize
    \caption[Diagrama de objetos de dominio]{}\label{fig:1-controlDomain}
\end{figure}

\textbf{Aplicación}

\tiny
\dirtree{%
.1 Application.
    .2 Port.
        .3 in.
            .4 Step.
                .5 Execute.
                    .6 Command.
                    .6 UseCase.
}
\normalsize

\textbf{Adapters}

\tiny
\dirtree{%
    .1 Adapter.
        .2 in.
            .3 GRPC.
                .4 Harán uso de los useCases de aplicación cuando llegue una request RPC.
            .3 Console.
                .4 Por ejemplo si quisieramos ejecutar los casos de uso mediante terminal.
        .2 out.
            .3 console.
                .4 implementación de los repository de llamada a los servidores clientes.
}
\normalsize