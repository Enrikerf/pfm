
\tiny
\dirtree{%
    .1 Project .
        .2 Domain.
        .2 Application.
        .2 Adapter.
        .2 Bootstrap.
}
\normalsize

\textbf{Dominio}

\begin{figure}[H]
    \tiny
\dirtree{%
    .1 Domain.
        .2 EnginePidController.
            .3 EnginePidController.
            .3 Service.
                .4 Configurator.
                .4 RPMControllerActivator.
                .4 PositionControllerActivator.
                .4 ControlDisabler.
        .2 Engine.
            .3 EngineVo.
            .3 Repository.
                .4 Find.
        .2 Encoder.
            .3 Encoder.
        .2 Pin.
            .3 EncoderPin.
                .4 EncoderPinInterface: actuan como repositorios.
            .3 OutPin.
                .4 OutPinInterface: actuan como repositorios.
            .3 PWMPin.
                .4 PWMPinInterface: actuan como repositorios.
        .2 Physic.
            .3 Angle.
            .3 LinearSpeed.
            .3 AngularSpeed.
            .3 Frequency.
            .3 Duty.
}
\normalsize
    \caption[Diagrama de objetos de dominio]{}\label{fig:1-controlDomain}
\end{figure}

\textbf{Aplicación}

\tiny
\dirtree{%
.1 Application.
    .2 Port.
        .3 in.
            .4 EnginePidController.
                .5 Configure.
                    .6 Command.
                    .6 UseCase.
                .5 EnableRpmControl (Command).
                .5 EnablePositionControl (Command).
                .5 Disable (Command).
}
\normalsize

\textbf{Adapters}

\tiny
\dirtree{%
    .1 Adapter.
        .2 in.
            .3 GRPC.
                .4 Harán uso de los useCases de aplicación cuando llegue una request RPC.
            .3 Console.
                .4 Por ejemplo si quisieramos ejecutar los casos de uso mediante terminal.
        .2 out.
            .3 pin.
                .4 implementación con librerías para actuar sobre los pines, Se buscará una librería que pueda ser configurada para actuar sobre pines de varias plataformas.
}
\normalsize