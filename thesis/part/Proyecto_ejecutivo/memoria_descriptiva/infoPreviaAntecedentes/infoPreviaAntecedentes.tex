
\paragraph{Hexagonal Architecture}

Dentro de las arquitecturas de capas se va a realizar un diseño de puertos y adaptadores;
más conocido como arquitectura hexagonal.
El diagrama básico más utilizado en la teoría para representarlo se muestra en la ~\cref{fig:hexagonalDiagram}.
Este diagrama tiene un enfoque didáctico ya que el hexágono es una simple licencia estética.
En el caso de existir más puertos de salida o entrada que los representados, el hexágono pierde el sentido.
%Cuando se enfrenta por primera vez este diagrama se tiende a intentar descifrar el sentido oculto detrás de la elección de la forma poligonal, no existe.

\begin{figure}[H]
    \centering
    \includegraphics[height=0.22\textheight]{./part/Ejecucion/Seguimiento/CreateTaskUseCase/img/HexagonalDiagram}
    \caption{Ejemplo básico de un Diagrama de arquitectura hexagonal\cite{TomHombergs2019GYHD}}\label{fig:hexagonalDiagram}
\end{figure}

En este diagrama, el Dominio está representado por una única entidad, o \textit{Entity}, que se encuentra aislada de todo y no depende de ningún elemento.
La Aplicación está representada por los casos de uso, o \textit{Use Case}, y utiliza el Dominio dependiendo de él.
La Aplicación se aísla del exterior, la Infraestructura, obligando a utilizar sus interfaces a los elementos que acceden a ella.
Las interfaces son los elementos que bordean el hexágono nombrados como \textit{Input/Output Port}.
Utilizando estas interfaces, nombradas puertos, se encuentran los adaptadores o \textit{Adapters}.
Los Adaptadores de entrada, a la izquierda, utilizan directamente las interfaces, mientras que los de salida los implementan.
Representado por fechas con cabeza negra o blanca respectivamente.

En el diagrama~\cref{fig:layers} se aprecia este concepto simplificado.
La arquitectura hexagonal no es más que una arquitectura de capas que decide dividir el concepto de Infraestructura en dos tipos: la que hace uso de la Aplicación y la que la Aplicación utiliza.
El objetivo es expresar que la dependencia de las capas, expresada por las flechas, sea siempre de fuera hacia adentro.
Se preserva del cambio el interior y exponer al cambio el exterior.
Separar lo propenso al cambio de lo que no.
Separar el lenguaje propio del Dominio, \textit{Domain}, de los detalles de implementación, que tienen su propio lenguaje.

\begin{figure}[H]
    \centering
    \includegraphics[height=0.22\textheight]{./part/Proyecto_ejecutivo/memoria_descriptiva/infoPreviaAntecedentes/img/PFM - Layer}
    \caption{Ejemplo básico de un Diagrama de arquitectura de capas}\label{fig:layers}
\end{figure}

\paragraph{Domain Driven Design}

\gls{DDD} (Domain-Driven Design o Diseño Dirigido por el Dominio) es una metodología de diseño enfocada a desarrollar un lenguaje común para todos los partícipes en un problema y su solución por ejemplo: clientes, vendedores, técnicos y financieros en un programa de gestión de venta de productos.
Para desarrollar ese lenguaje se divide el problema en contextos delimitados.

Por ejemplo en una empresa que comercialice actuadores automáticos: los actuadores para el equipo comercial significarán números de referencia y precios;
Para el departamento financiero significarán facturas y devoluciones;
Para los técnicos serán características físicas que los definen.
Si Los comerciales les llaman productos, los financieros les llaman concepto y los técnicos dispositivos, se encontrarán un problema de comunicación.
Realizar un software para la gestión de la empresa sin intentar resolver este problema de comunicación añade otra capa que además queda por escrita en el código.

%afiliado en un club deportivo significa facturas, números de identificación fiscal para los financieros y para la gente de operaciones significa reserva de pistas y cancelaciones.
%Para los de ventas significan descuentos, promociones, los demás actores tendrán otro significado para el mismo concepto.
%Todos utilizan la misma palabra para definir conceptos que difieren, pero cuando hablan de afiliados deben entenderse.
%Si en un departamento lo llaman clientes y en otro afiliados surgen problemas de comunicación.

Tener un lenguaje común donde todos puedan expresarse y hablar de la misma solución es el reto de este proceso.
Es lo que se define como \gls{UL} (Ubiquitous Language o lenguaje ubicuo) Intentar expresar en un mismo contexto todos esos significados termina en lo que se conoce como \textit{Big Ball of Mud} o Gran bola de barro.
Los componentes y las relaciones de este UL se pueden apreciar en la figura~\cref{fig:DomainDrivenDesignReference}.

Se describen los elementos que surgen del componente \textit{Model Driven Design}.
Que son los elementos dentro de ese lenguaje que afectan al diseño del software a su nivel más elemental:

\begin{itemize}
    \item \textit{Entity}: Elemento que contiene atributos definido por un identificador
    \item \textit{Value Object}: Elemento que tiene atributos pero no identificador
    \item Domain Event: Elemento que define una suceso inducido por la interacción entre los componentes del Dominio.
    \item Aggregate: Cluster de elementos tratado como una unidad.
    Las referencias o acciones externas sobre sus elementos siempre se hacen a través de un único elemento de este cluster conocido como \textit{Agreggate Root}.
    Tiene reglas definidas de consistencia dentro de su delimitación.
    \item Repository: Es un mecanismo de interacción para encapsular el acceso a tecnologías, como el almacenamiento en base de datos, para interactuar con ellas.
    La implementación no concierne al Dominio.
    \item Service: Es una funcionalidad de interacción entre elementos de Dominio.
    Encapsula lógica compleja que garantiza un comportamiento consistente.
\end{itemize}

Este paradigma de diseño compone el elemento central en el paradigma de la arquitectura de capas o \gls{LayerArchitecture}.
Se aíslan esos contextos que definen el Dominio de implementaciones concretas;
ya sea para acceder al Dominio o a las que accede el Dominio.
Por ejemplo, aislarlo de que se ejecute un servicio mediante una consola de comandos o desde una llamada http y que se guarde en una base de datos la información o se guarde en un archivo.

\begin{figure}[H]
    \centering
    \includegraphics[height=0.6\textheight]{./part/Proyecto_ejecutivo/memoria_descriptiva/infoPreviaAntecedentes/img/DomainDrivenDesignReference}
    \caption{Diagrama conceptual de los elementos que componen DDD\cite{EricEvans2003DDTC}}\label{fig:DomainDrivenDesignReference}
\end{figure}

\paragraph{CQRS}

Junto con el concepto de arquitectura hexagonal y el \textit{DDD} se aplica el paradigma de diseño conocido como \gls{CQRS} (Command Query Responsibility Segregation).
Es una técnica de diseño de arquitectura de software que separa la lógica de escritura (comandos) de la lógica de lectura (consultas) en sistemas de información.
La idea es que las operaciones de escritura (comandos) y las operaciones de lectura (consultas) se manejen por separado, ya que tienen necesidades y características distintas.
Mientras que las operaciones de escritura son responsables de modificar el estado de la aplicación, las operaciones de lectura son responsables de devolver información sobre ese estado sin modificarlo.
Al separar estas dos responsabilidades, se pueden optimizar las operaciones de lectura para que sean más rápidas y escalables.
Además, se puede diseñar una arquitectura de software más flexible, permitiendo una mayor adaptabilidad y evolución del sistema a medida que cambian los requisitos de la aplicación.

El resumen hasta ahora es que el diseño sigue una arquitectura hexagonal, con un enfoque \textit{DDD} en el Dominio y un enfoque CQRS en los casos de uso.
Es decir se diseñará un Dominio rico y con un UL y se accederá a su funcionalidad a través de casos de uso que sigan el criterio de ser comandos y consultas.

\paragraph{\textit{Mapping} o Mapeo}

Otro paradigma de diseño para garantizar la separación efectiva de las capas en su uso de los elementos básicos con los que interactúan es el mapeo o el \textit{Mapping}.
Si un elemento de Infraestructura utiliza directamente una \textit{Entity} de Dominio estaría saltando la capa de Aplicación en su diagrama de dependencia.
Bien es cierto que se mantendría la dependencia de fuera hacia adentro, pero se ha de tomar una decisión de hasta qué punto se quiere desacoplar una capa de otra.
A esta decisión de diseño se le conoce como Estrategia de \textit{Mapping} entre capas.
Estrictamente cada capa requiere sus objetos de trabajo para estar desacoplada de las demás, pero como en todo aspecto de diseño está sometido a discusión acerca de seguir la teoría al pié de la letra y el pragmatismo de no verse envuelto en redundancias y sobredimensionar las soluciones.

Hay tantas estrategias como atajos dentro de este paradigma se desee asumir.
Habrá de enfrentarse a la decisión de diseño entre la teoría y a los atajos que se pueden tomar.
Evaluar los beneficios e inconvenientes y tomar una decisión con la que se ha de ser consecuente, y más importante en desarrollo, consistente.
Esto quiere decir que una vez tomada una decisión debe ser una decisión en equipo que todos sigan.
Es más eficiente un diseño imperfecto que sea consistente que un diseño perfecto en unos puntos e imperfecto en otros.
Esto lleva al desorden a la hora de escribir código y complica la entrada de nuevos compañeros, el entendimiento del código existente con todas las consecuencias negativas que esto conlleva.

Se pueden clasificar los tipos de \textit{mapping} o mapeo en los siguientes grupos\cite{TomHombergs2019GYHD}:
\begin{itemize}
    \item Sin mapeo o \textit{The No Mapping Strategy}~\cref{fig:nomapping}
    \item Mapeo bidireccional o \textit{The Two-Way Mapping Strategy}~\cref{fig:twowaymapping}
    \item Mapeo completo o \textit{The Full Mapping Strategy}~\cref{fig:fullmapping}
    \item Mapeo unidireccional o \textit{The One-Way Mapping Strategy}~\cref{fig:onWaymapping}
\end{itemize}

Se van exponer cada tipo por separado con un ejemplo;
definiendo los elementos que intervienen en cada capa.
Para realizar una comparación, en todos los casos se simplifica el Dominio a una única entidad, \textit{Account}.
Se separa dicho Dominio de la Aplicación, en estos ejemplos un caso de uso \textit{SendMoney}, que contendrá la lógica necesaria para efectuar el envío de dinero a una cuenta.
Que a su vez se encontrará separada de la Infraestructura que interactuará con el sistema físico para efectuar ese envío y actualizará la información de las cuentas.

\begin{figure}[H]
    \centering
    \includegraphics[height=0.1\textheight]{./part/Ejecucion/Seguimiento/CreateTaskUseCase/img/nomapping}
    \caption{No mapping strategy~\cite{TomHombergs2019GYHD}}\label{fig:nomapping}
\end{figure}

En la estrategia de \textit{No Mapping} ilustrada en la figura~\cref{fig:nomapping} se aprecia que tanto Aplicación como Infraestructura dependen de Dominio.
Esto evita todo código redundante, los \gls{DTO} (Data Transfer Object), pero resta flexibilidad.
Si por detalles técnicos de una Infraestructura concreta se requiere más o menos parámetros que los definidos en el Dominio, o se deben guardar en otro formato, este diseño se enfrenta a un problema.

\begin{figure}[H]
    \centering
    \includegraphics[height=0.1\textheight]{./part/Ejecucion/Seguimiento/CreateTaskUseCase/img/twowaymapping}
    \caption{Two way mapping strategy~\cite{TomHombergs2019GYHD}}\label{fig:twowaymapping}
\end{figure}

La estrategia de \textit{Two-Way Mapping Strategy}~\cref{fig:twowaymapping} evita este problema y crea un modelo DTO que sirva para transportar la información de una capa a otra necesaria para conformar la Entidad.
De esta forma puede evolucionar por separado.
El problema persiste entre la Aplicación, los casos de uso, y el Dominio.
Se hace necesario escribir más código para crear las entidades a través de los DTO y viceversa.

\begin{figure}[H]
    \centering
    \includegraphics[height=0.1\textheight]{./part/Ejecucion/Seguimiento/CreateTaskUseCase/img/onWaymapping}
    \caption{One way mapping strategy~\cite{TomHombergs2019GYHD}}\label{fig:onWaymapping}
\end{figure}

En la estrategia de \textit{The One-Way Mapping Strategy}~\cref{fig:onWaymapping} se ilustra la creación de una interfaz para la Entidad y hace depender de nuevo todas las capas de dicha interfaz.
Se encuentran las capas más separadas y preparadas para el caso en el que se enfrente la necesidad de tener que crear distintas implementaciones aunque no se creen en un primer momento.

\begin{figure}[H]
    \centering
    \includegraphics[height=0.1\textheight]{./part/Ejecucion/Seguimiento/CreateTaskUseCase/img/fullmapping}
    \caption{Full mapping strategy~\cite{TomHombergs2019GYHD}}\label{fig:fullmapping}
\end{figure}

En la figura de \textit{The Full Mapping Strategy}~\cref{fig:twowaymapping} se opta por crear DTOs entre todas las capas.
Es la solución más pura, pero el coste es evidente: la cantidad de código a realizar es considerable y tiene que estar justificada con una necesidad de aislar hasta este punto.

No hay una regla de oro para elegir una estrategia que valga para todos los casos.
Debe ser una decisión de equipo, seguir la decisión y re-evaluar con cada inconveniente que enfrente la decisión si se tiene que cambiar la estrategia.

Se puede ver el primer ejemplo en el un proyecto de ejecución que intentara resolver todas las decisiones y describir al detalle la solución a desarrollar no aportaría valor.
Tomar una decisión de este tipo y documentarla carece de utilidad en este punto del diseño;
no se dispone de información suficiente para ello.
Al enfrentar el desarrollo del código se podrá valorar qué estrategia encaja mejor.

\paragraph{\textit{Testing} o Testeo}
    
Proteger el sistema frente al error humano es esencial para garantizar la entrega de código con seguridad y calidad. Para ello se emplearán herramientas de \gls{CI/CD} que se encargarán de realizar comprobaciones antes de permitir la llegada a servicio del nuevo código y el despliegue automático de dicho código en el entorno final. Este proyecto se centra en los primeros, CI o Continuous integration, debido a que no disponemos de servidores sobre los que desplegar de forma final.

En términos técnicos, CI/CD implica el uso de sistemas de control de versiones, servidores de automatización de compilación y pruebas, y herramientas de automatización de implementación y entrega. Estas herramientas permiten que los desarrolladores integren sus cambios de código en un repositorio central, donde se desencadenan automáticamente pruebas y compilaciones para detectar cualquier error y garantizar que el código se pueda implementar sin problemas.

Una vez que el código ha pasado todas las pruebas y ha sido aprobado por los revisores, el proceso de entrega y/o implementación se activa automáticamente. Esto implica la creación de un paquete de implementación, la realización de pruebas de aceptación automatizadas y el despliegue en producción. Este proceso ayuda a mejorar la eficiencia del equipo de desarrollo y a reducir los errores y riesgos en el proceso de entrega de software, permitiendo una implementación más rápida y frecuente de nuevas características y mejoras en el software.

Los tests son un apartado importante en la garantía de entrega de código libre de errores. En este proyecto se va a centrar el control de calidad en el dominio. Es decir, va a existir una suite de test unitarios por cada elemento de dominio que exista. Los test unitarios de aplicación e infraestructura quedan como un elemento deseable a tener, pero dependerá de los tiempos. los test de integración quedan fuera de esta primera versión del sistema. Las técnicas utilizadas para el desarrollo de los tests encuentran su origen en las siguentes condiciones de contorno:

\begin{itemize}
    \item si el código de pruebas no ahorra más esfuerzo de lo que cuesta desarrollarlo entonces se está perdiendo dinero.
    \item no se puede entregar un producto que no tiene ningúna clase de garantía de su correcto funcionamiento o acotado su margen de error.
\end{itemize}

Debemos trabajar dentro de esos dos límites: tener tests que garanticen y especifíquen el grado de calidad acotado al presupuesto que se dispone. Para poder expresar de una forma cuantitativa que los tests están correctamente diseñados y desarrollados primero definamos ineficaz e ineficiente. definición segun RAE:

\begin{itemize}
    \item eficaz: Capacidad de lograr el efecto que se desea o se espera.
    \item eficiente (2 accepción): Capacidad de lograr los resultados deseados con el mínimo posible de recursos
\end{itemize}

atendamos a la siguiente frase: si existiese un numero exacto de pruebas optimas, para cubrir todas la posibilidades con tests, si haces menos estas siendo ineficaz y si haces más ineficiente. Viendo estas definiciones queda claro que, cuando nos enfrentamos a la realidad, el trabajo de ingeniería constará en tomar una solución de compromiso entre eficacia y eficiencia. No podemos no probar y no tendremos recursos para probar todo.

En un ejemplo muy sencillo: para un método que recibe la información de un formulario de 8 campos con 10 posibles valores diferentes cada uno se obtendría un conjunto de \[ 8^{10} = 1,073,741,824 \] casos de prueba. Esta táctica de pruebas siendo la más obvia la mayor parte de las veces será inasumible.

El estado del arte y la técnica en diseño de test es amplia y variada. Este proyecto utilizará pruebas de caja negra. Las pruebas de caja negra se basan en diseñar un test sin suponer cómo va a estar desarrollado el código. Se diseña el caso de uso al que va a tener que enfretarse y el resultado esperado. Para evitar la prueba exhaustiva existen las siguientes técnicas:

\begin{itemize}
    \item variables independientes
    \subitem clases de equivalencia.
    \subitem metodo de valores límite
    \item variables dependientes
    \subitem vector de pares
\end{itemize}

\textbf{Variables independientes}

Su enfoque es reducir el conjunto de los valores posibles de cada entrada a un subconjunto representativo y reducido del conjunto original que asegure una cierta garantía de cobertura. Entonces la combinación de todos estos valores representativos de todas las entradas reducirá drásticamente el conjunto de casos de prueba final. Estas técnicas son Clases de Equivalencia y Análisis de Valores Límite; la segunda se aplica sobre la anterior.

Las clases de equivalencia son un subconjunto de valores de la entrada que el sistema debería manejar de forma equivalente en el ejercicio del \gls{SUT}(System Under Test). Para cada entrada deben repartirse todos sus posibles valores en un número finito de clases de equivalencia que cumplan la siguiente propiedad: la prueba de un valor representativo de una clase de equivalencia permite suponer “razonablemente“ que el resultado obtenido será un proceso similar que el obtenido probando cualquier otro valor de esa clase de equivalencia.

Es un proceso heurístico y se obtiene analizando:
\begin{itemize}
    \item la especificación (caja negra) que describe las características del proceso que debe cumplir la implementación.
    \item las salidas del SUT, que pueden aportar criterios para la partición en clases de equivalencia.
    \item las precondiciones del SUT, que aportan clases de equivalencia de error.
\end{itemize}

los pasos para diseñar el test son:
\begin{itemize}
    \item encontrar las Clases de Equivalencia de cada factor
    \item escoger un valor cualquiera de cada clase de equivalencia de cada factor
    \item si existen varios factores, generar todas las combinaciones de todos los valores anteriores de cada factor
    \item eliminar aquella combinaciones que no son factibles por la combinación de los valores de entrada
    \item añadir la salida correspondiente a cada combinación de valores de entrada
\end{itemize}

Para escoger dentro de una clase de equivalencia un valor se aplica la técnica de los valores límites. Esta técnica es aplicable a la partición de clases de equivalencia cuando sus valores tiene un orden total, o sea, que para toda pareja de valores distintos se puede comprobar si uno es mayor que otro o viceversa. Por ejemplo, enteros, reales, caracteres por su código, cadenas de caracteres por su orden lexicográfico, enumerados por su ordinal, fechas o horas. Son valores dentro de esa clase de equivalencia para el que cambia el comportamiento del SUT respecto del valor anterior.

La justificación se basa en la evidencia experimental de que “los errores se esconden en los rincones y se aglomeran en los límites” [Beizer] y por tanto se aumenta el conjunto de casos de prueba para mejorar la eficacia de encontrar errores. Es decir, que dentro de una clase de equivalencia puede requerirse el testeo de varios valores.

Vamos a exponer dos ejemplos que muestran la diferencia entre: parámetro de entrada y factor; y entre factor y clase de equivalencia.

Supongamos una función que acepta como parámetro de entrada un texto.

\begin{verbatim}
    funcion de nextDay(string day){...}
\end{verbatim}

Como estamos diseñando el test, no podemos saber la implementación. La entrada es un texto cualquiera. pero si quisieramos testear todos los textos posibles que pueden introducirse las pruebas necesarias serían infinitas. Aquí entra el concepto de factor. El factor es que para esa entrada existen 3 posibilidades: que la entrada sea invalida, o que sea válida. Para esos dos factores el conjunto que define la clases de equivalencia son

\begin{itemize}
    \item invalido: cualquier texto invalido
    \item válido: “lunes“,“martes“,“miércoles“,“jueves“,“viernes“,“sábado“ o “domingo“.
\end{itemize}

por lo tanto podríamos coger cualquier valor dentro de esas dos clases, por ejemplo:

\begin{itemize}
    \item invalido: “cualquierCadenaInvalida“ \(\longrightarrow\) resultado esperado: ERROR
    \item válido: “lunes“ \(\longrightarrow\) resultado esperado: “martes“
\end{itemize}

Vemos como podemos siguiendo esta metodología podemos dar una garantía. Explicar metodológicamente porqué se han hecho los tests y los valores escogidos y que covertura se está ofreciendo. Ahora si aplicamos la técnica de los valores límites sabemos que el “domingo“ es el último día y se debe volver al “lunes“. Es decir, es el valor límite dentro de nuestra clase de equivalencia. Por lo tanto si queremos garantizar y elegir con critero dentro de los casos de nuestra clase de equivalencia los tests quedarían:

\begin{itemize}
    \item invalido: “cualquierCadenaInvalida“ \(\longrightarrow\) resultado esperado: ERROR
    \item válido limite inicial: “lunes“ \(\longrightarrow\) resultado esperado: “martes“
    \item válido limite final: “domingo“ \(\longrightarrow\) resultado esperado: “lunes“
\end{itemize}

\textbf{Variables dependientes}

El método de pairwise, también conocido como pruebas de combinaciones de pares, es una técnica de diseño de pruebas que permite reducir el número de casos de prueba necesarios para lograr una cobertura exhaustiva de las combinaciones posibles de parámetros en un sistema o aplicación. En lugar de probar todas las combinaciones posibles de los parámetros, el método de pairwise identifica las combinaciones de pares que tienen el potencial de causar problemas o errores y las incluye en los casos de prueba. Estas combinaciones de pares se seleccionan utilizando un algoritmo que busca minimizar el número de casos de prueba necesarios sin sacrificar la calidad de la cobertura de pruebas.

El método de pairwise fue introducido por David C. Kuhn en un artículo titulado~\cite{37051} “Practical Combinatorial Testing“ publicado en 1997. Desde entonces, ha sido ampliamente utilizado en la industria de software y ha sido objeto de numerosos estudios y mejoras por parte de investigadores y practicantes.

Entre los beneficios del método de pairwise se encuentra la reducción del número de casos de prueba necesarios para una cobertura exhaustiva, lo que puede ahorrar tiempo y recursos. Sin embargo, es importante tener en cuenta que el método de pairwise no es una técnica infalible y que puede no detectar ciertos tipos de errores o problemas en el sistema o aplicación bajo prueba.

En conclusión, el método de pairwise es una técnica efectiva para la reducción del número de casos de prueba necesarios para lograr una cobertura exhaustiva de las combinaciones posibles de parámetros en un sistema o aplicación. Su aplicación puede mejorar la eficiencia y efectividad del proceso de pruebas, aunque debe ser utilizado en conjunto con otras técnicas de diseño de pruebas para lograr una cobertura completa.

un ejemplo extraido de~\cite{Bach04pairwisetesting} lo ejemplifica perfectamente. En un software con un menú que tenga doce botones para activar o desactivar tendremos 4096 casos diferentes que testear, para un software comercial, semejante coste de calidad es simplemente inasumible. y tal y como se menciona ¨Pairwise testing normally begins by selecting values for the system’s input variables. These individual values are often selected using domain partitioning. The values are then permuted to achieve coverage of all the pairings. This is very tedious to do by hand. Practical techniques used to create pairwise test sets include Orthogonal Arrays¨\cite{Bach04pairwisetesting} En este mismo paper reducen el caso tipico de menú de un programa de 12,288 a 10 casos con este método. Es una metodología para escoger los tests que más aportan valor de cara a la seguridad sin perder pragmatismo y economía. Para el cálculo de los pares hay varios software disponible como páginas web que permiten el cálculo de forma rápida.

Con estos métodos vamos a realizar pruebas que cumplan con las características de:

\begin{itemize}
    \item inocuas
    \item Automatizadas
    \item autoverificables
    \item repetibles
    \item independientes
    \item rápidas
    \item vector de pares
\end{itemize}

Con respecto al punto de inocuas significa que no introducen nuevos errores: para desarrollar el test tenemos que modificar el código no estamos siendo inocuos. Se requiere un diseño correcto y no es trivial. Un código se demuestra que es correcto cuando es fácil de testear de ahí que el enfoque TDD sea diseñar primero el test y luego el código, para conseguir esto se requiere de muchísima experiencia. En el desarrollo de software las veces que se puede hacer un TDD puro son escasas y por lo tanto por lo menos tenemos que tener claro que si a la hora de diseñar un test se complica el aislar la característica a testear, definir los factores y las clases de equivalencia, o nos salen un número inabarcable de tests estamos ante un código mejorable.

\paragraph{Docker}

Docker es el sistema de gestión de contenedores más utilizado en la industria a día de hoy.
Según la misma página oficial de Docker un contenedor se define como: Una unidad estándar de software que empaqueta el código y todas sus dependencias de tal modo que la aplicación pueda ejecutarse de forma rápida y fiable de un entorno de ejecución a otro.
La imagen de un contenedor es un paquete de software ligero, independiente y ejecutable que incluye todo lo que necesita para ser ejecutado: el código, el sistema, las herramientas, librerías y configuración~\cite{docker}.

El sistema se basa en la definición de imágenes que no son más que las instrucciones para la construcción de esos contenedores mediante el sistema de Docker Engine.
Difieren de las máquinas virtuales tal y como se muestra en la comparativa~\cref{fig:Docker vs VM}.

Esto permite correr multiples aplicaciones con multiples requerimientos en cualquier máquina sin tener que instalar dichas dependencias de forma local, evitando incompatibilidades e interacciones no deseadas.
Permite ofrecer un producto consistentes a los clientes de forma que el software y todo aquello que necesita para ejecutarse se entregan de forma conjunta, evitando problemas de instalación.

\begin{figure}[H]
    \centering
    \includegraphics[height=0.3\textheight]{./part/Proyecto_ejecutivo/memoria_descriptiva/prestaciones/docker/img/dockerVsVM}
    \caption{Comparación: Docker vs Máquinas virtuales.\cite{docker}}\label{fig:Docker vs VM}
\end{figure}

En la~\cref{fig:Docker vs VM} se muestra una comparación entre las capas que virtualiza cada sistema.
Cada contenedor de Docker es una abstracción en la capa de aplicación que empaqueta el código y las dependencias juntas.
Múltiples contenedores pueden ejecutarse en la misma máquina y compartir el \textit{Kernel} del sistema operativo cada uno ejecutándose en un entorno completamente aislado.
Los contenedores utilizan menos espacio que las máquinas virtuales, pueden manejar más aplicaciones y requieren menos virtualización y requerimientos del sistema operativo que otros sistemas.

Las máquinas virtuales son una abstracción completa del sistema físico que emulan, el hardware.
Sirven para transformar una máquina en, virtualmente, múltiples máquinas.
Cada máquina virtual incluye una copia completa de un sistema operativo, la aplicación, los ejecutables y librerías;
llegando a ocupar Gigas y sus arranques son lentos.

Los principales beneficios del sistema Docker son que:
\begin{itemize}
    \item Permite trabajar en el desarrollo de distintos proyectos con distintos requerimientos en una misma máquina sin la complejidad de mantener las dependencias de cada proyecto.
    O trasladarse a otra máquina y seguir trabajando de forma rápida al no tener que ponerse a instalar dichas dependencias de nuevo.
    \item Permite trabajar en equipo ya que cada cambio en las dependencias queda expresado en los cambios que se realizan de las imágenes de los contenedores que están disponibles para todo el equipo.
    \item Permite configurar el sistema para desarrollar o para producción de forma consistente con sus distintas diferencias en las dependencias necesarias, evitando problemas en el cambio de entorno de desarrollo y producción.
\end{itemize}