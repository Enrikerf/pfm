En un proyecto de desarrollo de un producto mediante software, hay un diseño inicial, pero también hay un proceso iterativo para descubrir la solución final que se desea. En este punto no se asemeja a la construcción de un bien inmobiliario en el que hay unos requisitos una fase de diseño y de construcción. Después se puede seguir iterando, construyendo más, eliminando partes o sustituyendolas. Es por esto que la arquitectura con la que se desarrolla tiene que estar enfocada siempre a enfrentar ese proceso iterativo con las mayores garantías posibles de que el diseño del programa no entorpece dicho proceso iterativo.

Para exponer el primer diseño de una solución de forma que sea útil a la hora de tomar decisiones de cara a invertir recursos en un proyecto tiene que exponerse de forma que se entienda lo que se va a desarrollar y el valor que aporta. Para combinarlo con una metodología ágil que suponga la entrega de valor de forma iterativa, evitar la parálisis por análisis, no centrarse en el imposible de intentar describir detalles técnicos que han de ser investigados, analizados o puestos a prueba con su propia implementación. Se va ha realizar este proyecto ejecutivo a forma de primer entregable explicativo del diseño a implementar. El objetivo es encontrar un balance entre detalle explicativo que elimine incertidumbre en el diseño y pragmatísmo de no definir en exceso y encontrarse con mucha documentación que luego tiende a cambiar por el propio devenir del método iterativo.

\sout{En este aspecto El proyecto ejecutivo difiere En uno industrial en que hay que dejar cerrado lo que se puede dejar cerrado y señalar los puntos de incertidumbre claramente. Estimando el coste de solventar dicha incertidumbre. En ese aspecto el presupuesto marca hasta que punto esa incertidumbre se resuelve o se solventa. el resultado puede ser que afecte a una funcionalidad basica del proyecto, no se consiga solventar por falta de presupuesto pero siempre quede una funcionalidad entregable unicamente pendiente de resolver ese punto. Es decir el programa debe ser valido si eso se resuelve para no emplear recursos a fondo perdido}

\subsubsection{Información previa: antecedentes y condiciones}
    
Para poder empezar la descripción primero hay que presentar una explicación del vocabulario básico. Lo cierto es que acerca de DDD y arquitecturas de software hay ingente documentación. Realizar un exposición de toda la teoría sería extenso. Como referencia fundamentar para iniciarse en el tema tomamos como referencia la devopedia\cite{devopediaDDD}. Intentar resumirlo mejor carece de valor. Este trabajo lo que intenta no es ser un manual teórico más de DDD, si no una aplicación práctica de dichos conceptos para entender en un caso real el alcance de dicha teoría. Las referencias básicas que se van a tomar son:

\begin{itemize}
    \item Domain-Driven Design: Tackling Complexity in the Heart of Software. De Eric Evans\cite{EricEvans2003DDTC}
    \item Implementing Domain-Driven Design. De Vaughn Vernon\cite{VaughnVernon2013IDD}
    \item Get Your Hands Dirty on Clean Architecture de Tom Hombergs \cite{TomHombergs2019GYHD}
\end{itemize}

Toda teoría del DDD va enfocada a desarrollar un lenguaje común a todos los interesados que intervienen en un problema y su solución: clientes, vendedores, técnicos, financieros, etc. para desarrollar ese lenguaje se divide el problema en contextos delimintados. En un ejemplo rápido afiliado en un club deportivo significa facturas, números de identificación fiscal para los financieros y para la gente de operaciones significa reserva de pistas y cancelaciones. y para los de ventas significan descuentos, promociones. Ect. El tener un lenguaje común donde todos puedan expresarse y hablar de la misma solución es el reto de este proceso. Es lo que se define como \textbf{Ubiquitous Language} o \textbf{UL}

Intentar expresar en un mismo contexto todos esos significados termina en lo que se conoce como ¨Big Ball of Mud¨ o Gran bola de barro. Los componentes de este UL se pueden apreciar en la figura  \ref{fig:DomainDrivenDesignReference}, se puede apreciar también la relación entre ellos

\begin{figure}[H]
    \centering
    \includegraphics[height=0.5\textheight]{./part/Proyecto_ejecutivo/memoria_descriptiva/infoPreviaAntecedentes/img/DomainDrivenDesignReference}
    \caption{DomainDrivenDesignReference\cite{EricEvans2003DDTC}}\label{fig:DomainDrivenDesignReference}
\end{figure}

De todo este diagrama los conceptos en los que nos vamos a centrar

\begin{itemize}
    \item Entity: Un Objeto que tiene atributos pero principalmente definido por un identificador
    \item Value Object: Un Objeto que tiene atributos pero no identificador
    \item Domain Event: Un Objeto que define una suceso inducido por la interacción entre los componentes del dominio.
    \item Aggregate: Es un cluster de objetos tratado como una unidad. Las referencias o acciones externas sobre sus elementos siempre se hacen a través de un único elemento de este cluster conocido como Aggregate root. Tiene reglas definidas de consistencia dentro de su delimitación. External references are restricted to only one member, called the Aggregate Root.
    \item Repository: Es un mecanismo de interaccion para encapsular el acceso a tecnologías, normalmente de almacenamiento para interactuar con ellas, cuya implementación no concierne al dominio.
    \item Service: Es una funcionalidad de interacción con el dominio que garantiza la interacción con el mismo de una forma consistente
\end{itemize}

Esta definición de dominio se implementa normalmente con un paradigma de diseño conocido como arquitectura de capa. Lo que se busca es aislar esos contextos que definen nuestro dominio de implementaciones concretas ya sea para acceder al mismo o a las que accede el dominio. Por ejemplo, aislarlo de que se ejecute un servicio mediante una consola de comandos o desde una llamada http y que se guarde en una base de datos la información o se guarde en un archivo.

El tipico diagráma cuando se habla de arquitectura hexagonal es tal y como se muestra en la figura\ref{fig:hexagonalDiagram} Si bien consideramos que no tiene mucho sentido y de cara a la parte didáctica confunde, ya que el hexágono es una simple licencia estética. en el caso de existir más puertos de salida y entrada que los representados el hexágono pierde todo el sentido y cuando se enfrenta por primera vez este diagrama se tiende a intentar descifrar el sentído del hexagono.

\begin{figure}[H]
    \centering
    \includegraphics[height=0.3\textheight]{./part/Ejecucion/Seguimiento/CreateTaskUseCase/img/HexagonalDiagram}
    \caption{Hexagonal architecture diagram\cite{TomHombergs2019GYHD}}\label{fig:hexagonalDiagram}
\end{figure}

Vemos en este diagrama que el dominio, representado por la Entity no accede a elementos exteriores. La aplicación respresentada por los UseCase utiliza el dominio y depende de él. pero se aisla del exterior, la infraestructura obligando a utilizar interfaces a los elementos que acceden a el y obligando a implementar las interfaces definidas por la aplicación para la infraestructura que sirve para acceder al exterior de la aplicación.

En el diagrama \ref{fig:layers} podemos ver simplificado que el objetivo es que la dependencia de las capas, expresada por las flechas, sea siempre de fuera hacia adentro. Queremos preservar del cambio el interior y exponer al cambio el exterior. Separar lo propenso al cambio de lo que no. El UL de los detalles de implementación que tienen su propio lenguaje.

\begin{figure}[H]
    \centering
    \includegraphics[height=0.3\textheight]{./part/Proyecto_ejecutivo/memoria_descriptiva/infoPreviaAntecedentes/img/PFM - Layer}
    \caption{layered architecture}\label{fig:layers}
\end{figure}

Junto con el concepto de arquitectura hexagonal y el DDD vamos a aplicar el paradigma de diseño conocido como CQRS

CQRS significa Command Query Responsibility Segregation (Segregación de responsabilidades de comandos y consultas, en español). Es una técnica de diseño de arquitectura de software que separa la lógica de escritura (comandos) de la lógica de lectura (consultas) en sistemas de información.

La idea es que las operaciones de escritura (comandos) y las operaciones de lectura (consultas) se manejen por separado, ya que tienen necesidades y características distintas. Mientras que las operaciones de escritura son responsables de modificar el estado de la aplicación, las operaciones de lectura son responsables de devolver información sobre ese estado.

Al separar estas dos responsabilidades, se pueden optimizar las operaciones de lectura para que sean más rápidas y escalables. Además, se puede diseñar una arquitectura de software más flexible, permitiendo una mayor adaptabilidad y evolución del sistema a medida que cambian los requisitos de la aplicación.

Tenemos entonces que vamos a diseñar una arquitectura hexagonal con un enfoque DDD en el dominio y un enfoque CQRS en los casos de uso. Es decir se diseñará un dominio rico y con un UL y se accederá a su funcionalidad a través de casos de uso que sigan el criterio de ser comandos o queries.

Una vez definidas las tres capas y resumido el concepto. Un punto importante a comentar es el concepto de la estrategia de mapping que hay entre capas. Cada capa requiere sus objetos de trabajo para estar desacoplada de las demás. Pero como en todo aspecto está sometido a discusión acerca de seguir la teoría a rajatabla y el pragmatismo de no verse envuelto en redundancias y sobredimensionar las soluciones.

En un extracto del libro Get Your Hands Dirty on Clean Architecture\cite{TomHombergs2019GYHD} podemos leer:
"\textit{ The argument might have gone something like this:}

\begin{itemize}
    \item \textit{Pro-Mapping Developer:}
    \subitem  \textit{ If we don’t map between layers, we have to use the same model in both layers which means that the layers will be tightly coupled!}
    \item \textit{Contra-Mapping Developer:}
    \subitem \textit{ But if we do map between layers, we produce a lot of boilerplate code which is overkill for many use cases, since they’re only doing CRUD and have the same model across layers anyways!}
\end{itemize}
\textit{As is often the case in discussions like this, there’s truth to both sides of the argument. Let’s discuss some mapping strategies with their pros and cons and see if we can help those developers make a decision.}"

Hay tantas estrategias como atajos dentro de este paradigma queramos asumir. Los tipos de mapping que se documentan en este libro:

\begin{itemize}
    \item The NoMapping Strategy \ref{fig:nomapping}
    \item The Two-Way MappingStrategy \ref{fig:twowaymapping}
    \item The Full MappingStrategy \ref{fig:fullmapping}
    \item The One-Way MappingStrategy \ref{fig:onWaymapping}
\end{itemize}

En todos estos casos vemos simplificado el dominio a una única entidad Account y vemos como separa dicho dominio del acceso al proceso que envía dinero a una cuenta y de dónde se guarda la información del envío de ese dinero.

\begin{figure}[H]
    \centering
    \includegraphics[height=0.1\textheight]{./part/Ejecucion/Seguimiento/CreateTaskUseCase/img/nomapping}
    \caption{No mapping strategy \cite{TomHombergs2019GYHD}}\label{fig:nomapping}
\end{figure}

\begin{figure}[H]
    \centering
    \includegraphics[height=0.1\textheight]{./part/Ejecucion/Seguimiento/CreateTaskUseCase/img/twowaymapping}
    \caption{two way mapping strategy \cite{TomHombergs2019GYHD}}\label{fig:twowaymapping}
\end{figure}

\begin{figure}[H]
    \centering
    \includegraphics[height=0.1\textheight]{./part/Ejecucion/Seguimiento/CreateTaskUseCase/img/onWaymapping}
    \caption{One way mapping strategy \cite{TomHombergs2019GYHD}}\label{fig:onWaymapping}
\end{figure}

\begin{figure}[H]
    \centering
    \includegraphics[height=0.1\textheight]{./part/Ejecucion/Seguimiento/CreateTaskUseCase/img/fullmapping}
    \caption{Full mapping strategy \cite{TomHombergs2019GYHD}}\label{fig:fullmapping}
\end{figure}

Todo programa creado en este proyecto: gestor de tareas, cliente y controlador va a estar dividido por tanto en dichas capas y seguir los paradigmas de diseño aquí definidos. En cuanto a la estrategia de mapping tomaremos una decisión cuando nos enfrentemos al problema.
\subsubsection{Descripción del proyecto}
    \begin{figure}[H]
    \centering
    \includegraphics[height=0.4\textheight]{./part/Proyecto_ejecutivo/memoria_descriptiva/descripcionDelProyecto/manager/uml/systemConcept}
    \caption{Diagrama UML de despliegue del sistema}\label{fig:Diagrama UML de despliegue del sistema}
\end{figure}

Se puede ver un diagrama conceptual del sistema en la~\cref{fig:Diagrama UML de despliegue del sistema}. Los actores en este sistema diseñado serán:
\begin{itemize}
    \item el servidor que contenta el programa \gls{Manager} guardará las tareas, las ejecutará y guardará el resultado de dicha ejecución.
    \item los servidores que contengan una copia del programa \gls{Client} recepcionarán las llamadas del manager con el comando y las ejecutará devolviendo los resultados al Manager.
    \item los mismos servidores clientes contendrán una copia del programa a ejecutar, pudiendo ser tantos como se deseen, en nuestro caso implementaremos el programa de control PID par aun motor de corriente continua.
\end{itemize}

La memoria descriptiva de cada uno de los componentes estará compuesta de los siguientes elementos:

\begin{itemize}
    \item diagrama de los despliegue: interacción entre componentes del subsistema
    \item diagrama de objetos: diagrama de su diseño\textit{DDD}
    \item casos de uso: descripción y diagrama de actividad
    \item estructura de carpetas para la organización del código
\end{itemize}

\paragraph{Manager Server}
\begin{figure}[H]
    \centering
    \includegraphics[height=0.4\textheight]{./part/Proyecto_ejecutivo/memoria_descriptiva/descripcionDelProyecto/manager/uml/managerServerConcept}
    \caption[Diagrama componentes]{}\label{fig:managerServerConcept}
\end{figure}

\subparagraph{Dominio}

En el diagrama UML~\ref{fig:managerDomain} vemos que el dominio de Manager se compone de dos modulos: uno para las tareas y otro para los resultados. Vamos a ir explicando uno por uno:

\begin{figure}[H]
    \centering
    \includegraphics[height=0.4\textheight]{./part/Proyecto_ejecutivo/memoria_descriptiva/descripcionDelProyecto/manager/uml/managerDomain}
    \caption[Diagrama de objetos de dominio]{}\label{fig:managerDomain}
\end{figure}

\begin{itemize}
    \item CoreDomain
    \begin{itemize}
        \item Id todos los ids extenderan de este id, contiene un uuid, no sabemos que paquete usaremos para generarlos, es una de las pocas dependencias externas que vamos a tener dentro del dominio y queremo encapsularla lo máximo posible por si hubera que cambiarla. Además de esta forma los ids de las entities no se confunden en su tipo. si por ejemplo buscas una task mediante un id que corresponde a un result, si fueran del mismo tipo daría lugar a confusión porque no lo encontraríamos pero no nos advertiría de nuestro error
        \item Event todos los eventos del sistema extenderan del evento este
    \end{itemize}
    \item TaskDomain
    \begin{itemize}
        \item Task: AggregateRoot
        \item TaskId
        \item Host: es un Value Object compuesto por el valor del host, es un string pero el Value Object garantiza que es un valor válido, si no devuelve un error
        \item Port: es un Value Object compuesto por el valor del puerto, es un string pero el Value Object garantiza que es un valor válido, si no devuelve un error
        \item CommunicationMode: es un enum para expresar esa tarea de las formas de comunicación posibles que hay entre dos servidores cual sera la utilizada
        \begin{itemize}
            \item UNARY
            \item SERVER\_STREAM
            \item CLIENT\_STREAM
            \item BIDIRECTIONAL
        \end{itemize}
        \item ExecutionMode: es un enum
        \begin{itemize}
            \item MANUAL
            \item AUTOMATIC
        \end{itemize}
        \item Status: es un enum
        \begin{itemize}
            \item PENDING
            \item RUNNING
            \item SUCCESSFUL
            \item FAILED
        \end{itemize}
        \item Step: cada tarea puede componerse en distintos pasos. Por ejemplo queremos poder poner en marcha el motor durante 15 segudos y luego llevarlo a una posición de inicio
        \begin{itemize}
            \item StepId
            \item sentence: será un string de contenido libre que el servidor cliente ejecutará en su sistema, dependerá de él tener instalado dicho programa y corroborar que la sintaxis es la correcta
        \end{itemize}
        \item TaskCreatedEvent. Cuando se cree una tarea se emitirá un evento, habrá un manejador de eventos que actuará en consecuencia. Si es una tarea automatizada y el loop de ejecución está parado lo pondrá en marcha.
        \item TaskModifiedEvent: si una tarea es modificada se emitirá un evento, habrá un manejador de eventos que actuará en consecuencia. Si la tarea vuelve a ser puesta a pending, es automatizada y el loop de ejecución está parado lo pondrá en marcha
        \item TaskDeletedEvent: si una tarea es modificada se emitirá un evento, habrá un manejador de eventos que actuará en consecuencia.
    \end{itemize}
    \item ResultDomain
    \begin{itemize}
        \item BatchId
        \item Batch: aggregate root para referencial el conjunto de resultados correspondiente a la ejecución de una tarea. Una tarea se puede ejecutar varias veces, cada vez con distintos resultados. Para agruparlos necesitaremos este aggregate root
        \item ResultId
        \item Result: contendrá un string con el resultado en plano. Cada tarea y proceso tendrá a su disposición este strings para crear su propio formato de respuesta. Será tarea de cada cliente interpretarlos.
    \end{itemize}
\end{itemize}

\subparagraph{casos de uso}

Vamos a describir los casos de uso que podrán ejecutarse en el programa manager. Tenemos 2 Entities y dos aggregate root.

\begin{itemize}
    \item Task
    \subitem Step
    \item Batch
    \subitem Result
\end{itemize}

Vamos a hacer el CRUD completo de Task; este contempla la edición de los steps a través de Task ya que es su aggregate root y por definición no se debe poder acceder a sus componentes si no es a través del aggregate. De Batch no vamos a hacer el update porque no tiene sentido a nivel de caso de uso poder actualizar un Batch, teniendo en cuenta que sus variables están ligadas al Task con el que se encuentra ligado y las fechas no dependen del usuario. y lo mismo pasa con result


\textbf{crear tarea}

\begin{figure}[H]
    \centering
    \includegraphics[height=0.2\textheight]{./part/Proyecto_ejecutivo/memoria_descriptiva/descripcionDelProyecto/manager/uml/createTaskUseCase}
    \caption[Diagrama de objetos de dominio]{}\label{fig:createTaskUseCase}
\end{figure}

\textbf{obtener tarea}

\begin{figure}[H]
    \centering
    \includegraphics[height=0.2\textheight]{./part/Proyecto_ejecutivo/memoria_descriptiva/descripcionDelProyecto/manager/uml/getTaskUseCase}
    \caption[Diagrama de objetos de dominio]{}\label{fig:getTaskUseCase}
\end{figure}

\textbf{listar tareas}

Uno de los puntos más amplios en una API CRUD es el filtrado de datos. No entra dentro del ámbito de este proyecto crear un sistema de filtrado que incluya la paginación. En este caso habría que crear una nomenclatura de filtros de cara al usuario y un sistema que los procese, devolviendo error ante un filtro erroneo o el listado de tareas que responda a dicho filtro. En nuestro caso devolveremos todas las tareas.

\textbf{actualizar tareas}

\begin{figure}[H]
    \centering
    \includegraphics[height=0.5\textheight]{./part/Proyecto_ejecutivo/memoria_descriptiva/descripcionDelProyecto/manager/uml/updateTaskUseCase}
    \caption[Diagrama de objetos de dominio]{}\label{fig:updateTaskUseCase}
\end{figure}

\textbf{eliminar tareas}

\begin{figure}[H]
    \centering
    \includegraphics[height=0.2\textheight]{./part/Proyecto_ejecutivo/memoria_descriptiva/descripcionDelProyecto/manager/uml/deleteTaskUseCase}
    \caption[Diagrama de objetos de dominio]{}\label{fig:deleteTaskUseCase}
\end{figure}

\textbf{TaskEventHandler}

tanto en TaskCreatedEvent como TaskUpdatedEvent el handler va a ser el mismo. TaskDeletedEvent lo dejaremos en este diseño inicial.

\begin{figure}[H]
    \centering
    \includegraphics[height=0.2\textheight]{./part/Proyecto_ejecutivo/memoria_descriptiva/descripcionDelProyecto/manager/uml/taskEventHandlerUseCase}
    \caption[Diagrama de objetos de dominio]{}\label{fig:taskCreatedEventHandtaskEventHandlerUseCaselerUseCase}
\end{figure}

\textbf{TaskLoop: ejecutar tareas automáticas pendientes}

Una vez que el TaskEventHandler ponga en marcha el TaskLoop este obtendrá de base de datos todas las tareas automáticas pendientes y iterando sobre ellas lanzará un hilo de ejecución para ejecutarlas. Una vez terminadas todas volverá a intentar obtener tareas pendientes ya que en el periodo que haya estado ejecutandolas pueden haberse introducido nuevas. Si no hay más tareas pendientes el loop se desactivará.

Un punto a mencionar de este Task loop es que se implementará en aplicación porque aquí intervendrán todavía puntos de la aplicación y la tecnología que no están investigados y son complejos. Es donde esta arquitectura mostrará toda su flexibilidad y fiabilidad de cara a proteger los elementos básicos de la aplicación de los cambios.

Una vez desarrollado, o puede que incluso durante el mismo desarrollo habrá en este proceso muchos conceptos que descubriremos core de la aplicación y debieran ser dominio, pero como tenemos que descubrirlos vamos a implementarlos en aplicación haciendo uso de interfaces de infraestructura y de elementos de dominio. Si vemos claramente que algo puede convertirse en dominio lo introduciremos.

\sout{Será el único punto en donde una interfaz de dominio será implementada por un elemento de aplicación en vez de infraestructura.} \textcolor{red}{El diseño de esto es complejo y todavía no lo tengo. Event handler necesita todo para crearse?}

\begin{figure}[H]
    \centering
    \includegraphics[height=0.3\textheight]{./part/Proyecto_ejecutivo/memoria_descriptiva/descripcionDelProyecto/manager/uml/executeTaskLoop}
    \caption[Diagrama de objetos de dominio]{}\label{fig:executeTaskLoop}
\end{figure}

\textbf{TaskProcessor} que aparece en la figura\ref{fig:executeTaskLoop} en color verde es el proceso asíncrono donde se hará uso del las gorutines que permiten la gestión de tareas asíncronas de forma sencilla. Una de las bondades que vende el lenguaje y donde se pondrá a prueba. se lanzarán todas las tareas pendientes en paralelo y se hará un bloqueo del loop mientras todas terminan. Como en el tiempo que esto ocurre han podido entrar nuevas tareas pendientes se repetirá la lógica hasta que no haya más.

Se puede ver el flujo lógico de esta lógica en la figura\ref{fig:1-TaskProcessor}

\begin{figure}[H]
    \centering
    \includegraphics[height=0.5\textheight]{./part/Proyecto_ejecutivo/memoria_descriptiva/descripcionDelProyecto/manager/uml/1-TaskProcessor}
    \caption[Diagrama de objetos de dominio]{}\label{fig:1-TaskProcessor}
\end{figure}


\textbf{Create Batch}

\begin{figure}[H]
    \centering
    \includegraphics[height=0.2\textheight]{./part/Proyecto_ejecutivo/memoria_descriptiva/descripcionDelProyecto/manager/uml/createBatchUseCase}
    \caption[Diagrama de objetos de dominio]{}\label{fig:createBatchUseCase}
\end{figure}

\textbf{Delete Batch}

\begin{figure}[H]
    \centering
    \includegraphics[height=0.2\textheight]{./part/Proyecto_ejecutivo/memoria_descriptiva/descripcionDelProyecto/manager/uml/deleteBatchUseCase}
    \caption[Diagrama de objetos de dominio]{}\label{fig:deleteBatchUseCase}
\end{figure}

\textbf{obtener batch}

\begin{figure}[H]
    \centering
    \includegraphics[height=0.2\textheight]{./part/Proyecto_ejecutivo/memoria_descriptiva/descripcionDelProyecto/manager/uml/getBatchUseCase}
    \caption[Diagrama de objetos de dominio]{}\label{fig:getBatchUseCase}
\end{figure}

\textbf{listar batches}

Uno de los puntos más amplios en una API CRUD es el filtrado de datos. No entra dentro del ámbito de este proyecto crear un sistema de filtrado que incluya la paginación. En este caso habría que crear una nomenclatura de filtros de cara al usuario y un sistema que los procese, devolviendo error ante un filtro erroneo o el listado de tareas que responda a dicho filtro. En nuestro caso devolveremos todas las tareas.

\textbf{Obtener los results de un batch}

\begin{figure}[H]
    \centering
    \includegraphics[height=0.2\textheight]{./part/Proyecto_ejecutivo/memoria_descriptiva/descripcionDelProyecto/manager/uml/getBatchResultsUseCase}
    \caption[Diagrama de objetos de dominio]{}\label{fig:getBatchResultsUseCase}
\end{figure}

\textbf{Execute task manually}

\textcolor{red}{Ahora mismo es el stream result}

solo se puede ejecutar una tarea manual por vez, en este primer diseño. El motivo es que como se quiere graficar en tiempo real la respuesta, por ejemplo para monitorizar la velocidad del motor en tiempo real, no queremos meternos en la complejidad de graficar varios resultados a la vez.

Podemos apreciar que cuando se ejecuta manualmente una tarea se pone en estado running y entra enun bucle infinito. Hasta que un usuario no pona esa tarea en DONE no para de ejecutar el comando. O Hasta que uno de los steps para el proceso

\textcolor{red}{Decir que esto es el primer esbozo. Se deja claro que depende mucho de como funcione el RPC o las limitaciones tecnicas que nos encontremos. esto no esta implementado asi. el diagrama esta mal porque ejecuta una y otra vez los steps, en lo que hay ahora solo ejecuta el primer step y se queda esperando}

\begin{figure}[H]
    \centering
    \includegraphics[height=0.3\textheight]{./part/Proyecto_ejecutivo/memoria_descriptiva/descripcionDelProyecto/manager/uml/1-executeTaskManual}
    \caption[Diagrama de objetos de dominio]{}\label{fig:1-executeTaskManual}
\end{figure}

\subparagraph{estructura de carpetas}

Una de las partes mas importantes de un proyecto de software es que la estructura de carpetas hable sobre cómo está diseñado el software, sobre de qué va el software. qué es lo que hace y sobre que componentes interactua

En el proyecto constará de 4 carpetas principales
\dirtree{%
    .1 Project .
        .2 Adapter.
        .2 Application.
        .2 Domain.
        .2 Bootstrap.
}

Bootstrap será donde hagamos la composición de la aplicación, es decir la inyección de dependencias.

Vamos a desplegar la estructura capa por capa

\textbf{Adapters}

\dirtree{%
.1 Adapter.
    .2 in.
        .3 GRPC.
            .4 CreateEntityGrpcCall.
        .3 Console.
            .4 CreateEntityTerminalCommand.
    .2 out.
        .3 Email.
            .4 SendEmailOnCreationImplementation (*1).
        .3 Mysql.
            .4 SaveEntityImplementation (*2).
}

\textbf{Aplicación}

\dirtree{%
.1 Application.
    .2 Port.
        .3 in.
            .4 Entity.
                .5 CreateUseCase.
                    .6 CreateCommand.
                        .7 CreateCommand (using sendEmail param in this example).
                        .7 CreateUseCase.
                    .6 SomeEventHandler.
                        .7 CreationEvent.
                        .7 CreationEventUseCase.
        .3 out.
            .4 Email.
                .5 SendEmailConCreation (interface for *1).
}

\textbf{Dominio}

\dirtree{%
.1 Domain.
    .2 Core.
        .3 Id.
        .3 Event.
        .3 Error.
    .2 Task.
        .3 Id.
        .3 Task.
        .3 Host.
        .3 Port.
        .3 CommunicationMode.
        .3 ExecutionMode.
        .3 Status.
        .3 Step.
            .4 Id.
            .4 StepEntity.
            .4 StepVo.
            .4 Repository.
                .5 Find.
                .5 Save.
                .5 Search.
                .5 Delete.
                .5 Update.
            .4 Service.
                .5 Finder.
                .5 Creator.
                .5 Updater.
                .5 Eraser.
                .5 Searcher.
        .3 Repository.
            .4 Find.
            .4 Save.
            .4 Search.
            .4 Delete.
            .4 Update.
        .3 Service.
            .4 Finder.
            .4 Creator.
            .4 Updater.
            .4 Eraser.
            .4 Searcher.
    .2 Result.
}



\paragraph{Client Program}

En la \cref{fig:Diagrama UML de despliegue del cliente} podemos ver el diagrama UML de despliegue en el servidor cliente. Contendrá a el programa Cliente que recepciona las llamadas del Manager y ejecutará en la misma máquina el comando recibido. Dicho comando deberá estar disponible en el sistema para ser ejecutado y devolver el resultado que el programa Cliente debe comunicar al Manager de vuelta.

\begin{figure}[H]
    \centering
    \includegraphics[height=0.2\textheight]{./part/Proyecto_ejecutivo/memoria_descriptiva/descripcionDelProyecto/client/uml/clientServerConcept}
    \caption{Diagrama UML de despliegue del cliente}\label{fig:Diagrama UML de despliegue del cliente}
\end{figure}

\subparagraph{Dominio}

Este programa será casi todo infrestructura. Esto es debido a que una vez recepcionado el comando mediante el RPC sólo será necesario ejecutarlo. Siendo tarea del programa ejecutado interpretar dicho comando y transformar esa cadena de caracteres que contiene el \textit{Step} a un dominio interno. Se ve más claro lo que se quiere expresar con un ejemplo. El típico comando de consola en un sistema UNIX

\begin{verbatim}
    ./runMyComand --arg=arg1 --argN=argN
\end{verbatim}

Será el string contenido en el \textit{Step} guardado en el Manager y recepcionado por el \textit{Client} que ejecutará en el mismo sistema operativo. Por lo tanto Las posibilidades son tantas como comandos haya instalados en el servidor cliente. En nuestro caso podremos mandar a ejecutar todos los comandos que queramos que vengan previamente instalados en un sistema UNIX y además el programa de control donde podremos interactuar con el motor de corriente continua. Algunos de los posibles comandos serán:

\begin{verbatim}
    ./pidControl --velocity=30rpm
    ./pidControl --position=180deg
    ./pidControl --setP=1
    ./pidControl --setI=0
    ./pidControl --setD=0
    ./pidControl --enabled=true
    ./pidControl --enabled=false
\end{verbatim}

El diagrama de Dominio de el \textit{Client} podemos verlo en la \cref{fig:Diagrama UML de el dominio de cliente}. Constando únicamente de dos Value Objects para los \textit{Steps} recibidos y para los \textit{Results} obtenidos.

\begin{figure}[H]
    \centering
    \includegraphics[height=0.2\textheight]{./part/Proyecto_ejecutivo/memoria_descriptiva/descripcionDelProyecto/client/uml/clientDomain}
    \caption{Diagrama UML de el dominio de cliente}\label{fig:Diagrama UML de el dominio de cliente}
\end{figure}

\begin{itemize}
    \item StepDomain
    \begin{itemize}
        \item StepVo
    \end{itemize}
    \item ResultDomain
    \begin{itemize}
        \item ResultVo
    \end{itemize}
\end{itemize}

\subparagraph{Casos de uso}

Vamos a describir los casos de uso que podrán ejecutarse en el programa client. En este caso no tendremos Entities porque no necesitamos identificadores.
Tendremos los \textit{Value Object}s para los steps que nos llegaran a modo de request y el necesario para responder. En este sistema no hay persistencia.

\textbf{Execute Unary Step}

Centrando en un caso de uso de este tipo de steps en nuestro programa de control sirve por ejemplo para establecer un parámetro de control del PID. Cualquiera de los siguientes.

\begin{verbatim}
    ./pidControl --velocity=30rpm
\end{verbatim}

Podemos ver en~\cref{fig:Use Case-Execute Unary Step} el diagrama de flujo que responde a este caso de uso

\begin{figure}[H]
    \centering
    \includegraphics[height=0.2\textheight]{./part/Proyecto_ejecutivo/memoria_descriptiva/descripcionDelProyecto/client/uml/executeUnaryStep}
    \caption{Use Case: Execute Unary Step}\label{fig:Use Case-Execute Unary Step}
\end{figure}

\textbf{Execute ClientStream Step}

Centrando en un caso de uso de este tipo de steps en nuestro programa de control sirve por ejemplo para establecer una configuración completa con una sola llamada para nuestro PID:
\begin{verbatim}
    ./pidControl --velocity=30rpm
    ./pidControl --position=180deg
    ./pidControl --setP=1
    ./pidControl --setI=0
    ./pidControl --setD=0
\end{verbatim}

Podemos ver en~\cref{fig:Use Case-Execute ClientStream Step} el diagrama de flujo que responde a este caso de uso

\begin{figure}[H]
    \centering
    \includegraphics[height=0.2\textheight]{./part/Proyecto_ejecutivo/memoria_descriptiva/descripcionDelProyecto/client/uml/executeClientStreamStep}
    \caption{Use Case: Execute ClientStream Step}\label{fig:Use Case-Execute ClientStream Step}
\end{figure}

\textbf{Execute ServerStream Step}

Centrando en un caso de uso de este tipo de steps en nuestro programa de control sirve por ejemplo para habilitar el control del pid durante un determinado tiempo e ir devolviendo el estado la variable de control
\begin{verbatim}
    ./pidControl --enabled=true --time=10s
\end{verbatim}

O dejar el control activado indefinidamente hasta que llege otra request de tipo Unary que lo detenga

\begin{verbatim}
    ./pidControl --enabled=false
\end{verbatim}

Podemos ver en~\cref{fig:Use Case-Execute ServerStream Step} el diagrama de flujo que responde a este caso de uso

\begin{figure}[H]
    \centering
    \includegraphics[height=0.2\textheight]{./part/Proyecto_ejecutivo/memoria_descriptiva/descripcionDelProyecto/client/uml/executeServerStreamStep}
    \caption{Use Case: Execute ServerStream Step}\label{fig:Use Case-Execute ServerStream Step}
\end{figure}

\textbf{Execute Bidirectional Step}

Usando el ejemplo anterior, si no queremos habilitar el control indefinidamente y deshabilitarlo mediante otra request podemos usar este caso de uso.

\begin{verbatim}
    ./pidControl --enabled=true --time=10s
\end{verbatim}

y posteriormente cuando el Cliente decida.

\begin{verbatim}
    ./pidControl --enabled=false
\end{verbatim}

Este caso de uso es el que utilizaremos para el control manual.

Podemos ver en~\cref{fig:Use Case-Execute Bidirectional Step} el diagrama de flujo que responde a este caso de uso

\begin{figure}[H]
    \centering
    \includegraphics[height=0.2\textheight]{./part/Proyecto_ejecutivo/memoria_descriptiva/descripcionDelProyecto/client/uml/executeBidiStep}
    \caption{Use Case: Execute Bidirectional Step}\label{fig:Use Case-Execute Bidirectional Step}
\end{figure}

\subparagraph{Estructura de carpetas}

En el proyecto constará de 4 carpetas principales

\begin{figure}[H]
    \setlength{\DTbaselineskip}{10pt}
    \DTsetlength{0.2em}{1em}{0.2em}{0.4pt}{1.6pt}
    \dirtree{%
        .1 Project .
        .2 Domain.
        .2 Application.
        .2 Adapter.
        .2 Bootstrap.
    }
    \caption{Client: Estructura de carpetas de Proyecto}\label{fig:Client- Estructura de carpetas de Proyecto}
\end{figure}

\textbf{Dominio}

\begin{figure}[H]
    \setlength{\DTbaselineskip}{10pt}
    \DTsetlength{0.2em}{1em}{0.2em}{0.4pt}{1.6pt}
    \dirtree{%
        .1 Domain.
        .2 Step.
        .3 StepVo.
        .3 ResultVo.
        .3 Repository.
        .4 consoleWrite.
        .3 Services.
        .4 UnaryExecutor.
        .4 ClientStreamExecutor.
        .4 ServerStreamExecutor.
        .4 BidirectionalExecutor.
    }
    \caption{Client: Estructura de carpetas de Dominio}\label{fig:Client-Estructura de carpetas de Dominio}
\end{figure}

\textbf{Aplicación}

\begin{figure}[H]
    \setlength{\DTbaselineskip}{10pt}
    \DTsetlength{0.2em}{1em}{0.2em}{0.4pt}{1.6pt}
    \dirtree{%
        .1 Application.
        .2 Port.
        .3 in.
        .4 Step.
        .5 ExecuteUnary.
        .6 Command.
        .6 UseCase.
        .5 ExecuteServerStream (Command).
        .5 ExecuteClientStream (Command).
        .5 ExecuteBidi (Command).
    }
    \caption{Client: Estructura de carpetas de Aplicación}\label{fig:Client-Estructura de carpetas de Aplicación}
\end{figure}

\textbf{Adapters}

\begin{figure}[H]
    \setlength{\DTbaselineskip}{10pt}
    \DTsetlength{0.2em}{1em}{0.2em}{0.4pt}{1.6pt}
    \dirtree{%
        .1 Adapter.
        .2 in.
        .3 GRPC.
        .4 Harán uso de los useCases de aplicación cuando llegue una request RPC.
        .2 out.
        .3 console.
        .4 implementacion de consoleWrite para ejecutar los comandos en el sistema y obtener los resultados.
    }
    \caption{Client: Estructura de carpetas de Infraestructura}\label{fig:Client-Estructura de carpetas de Infraestructura}
\end{figure}







\paragraph{Control Program}
Casos de uso todas las funciones del engine y diseñarlo para que la infra sea a través de consola de comandos (preparando para que luego debido a la comunicación rpc devolver los datos sea muy complicado si lo dividimos en otro comando y decidimos absorver este dominio)

El sistema está pensado para que el programa a ejecutar sea de libre decisión del cliente. pero vamos a aprovechar la oportunidad para crear un programa de control y profundizar tanto en el uso del lenguaje como en los conocimientos relacionados con este master. En concreto el control automático. Uno do los programas que podrá ejecutar será un control PID.



\begin{figure}[H]
    \centering
    \includegraphics[height=0.3\textheight]{./part/Proyecto_ejecutivo/memoria_descriptiva/descripcionDelProyecto/control/uml/controlConcept}
    \caption[Diagrama componentes]{}\label{fig:controlConcept}
\end{figure}

\subparagraph{Dominio}

\begin{figure}[H]
    \centering
    \includegraphics[height=0.4\textheight]{./part/Proyecto_ejecutivo/memoria_descriptiva/descripcionDelProyecto/control/uml/controlDomain}
    \caption[Diagrama de objetos de dominio]{}\label{fig:controlDomain}
\end{figure}

\begin{itemize}
    \item StepDomain
\end{itemize}

\subparagraph{casos de uso}

Vamos a describir los casos de uso que podrán ejecutarse en el programa manager. Tenemos 2 Entities y dos aggregate root.

\begin{itemize}
    \item Step
    \item Result
\end{itemize}

\textbf{ejecutar Step}

\subparagraph{estructura de carpetas}


\tiny
\dirtree{%
    .1 Project .
        .2 Domain.
        .2 Application.
        .2 Adapter.
        .2 Bootstrap.
}
\normalsize

\textbf{Dominio}

\begin{figure}[H]
    \tiny
\dirtree{%
    .1 Domain.
        .2 EnginePidController.
            .3 EnginePidController.
            .3 Service.
                .4 Configurator.
                .4 RPMControllerActivator.
                .4 PositionControllerActivator.
                .4 ControlDisabler.
        .2 Engine.
            .3 EngineVo.
            .3 Repository.
                .4 Find.
        .2 Encoder.
            .3 Encoder.
        .2 Pin.
            .3 EncoderPin.
                .4 EncoderPinInterface: actuan como repositorios.
            .3 OutPin.
                .4 OutPinInterface: actuan como repositorios.
            .3 PWMPin.
                .4 PWMPinInterface: actuan como repositorios.
        .2 Physic.
            .3 Angle.
            .3 LinearSpeed.
            .3 AngularSpeed.
            .3 Frequency.
            .3 Duty.
}
\normalsize
    \caption[Diagrama de objetos de dominio]{}\label{fig:1-controlDomain}
\end{figure}

\textbf{Aplicación}

\tiny
\dirtree{%
.1 Application.
    .2 Port.
        .3 in.
            .4 EnginePidController.
                .5 Configure.
                    .6 Command.
                    .6 UseCase.
                .5 EnableRpmControl (Command).
                .5 EnablePositionControl (Command).
                .5 Disable (Command).
}
\normalsize

\textbf{Adapters}

\tiny
\dirtree{%
    .1 Adapter.
        .2 in.
            .3 GRPC.
                .4 Harán uso de los useCases de aplicación cuando llegue una request RPC.
            .3 Console.
                .4 Por ejemplo si quisieramos ejecutar los casos de uso mediante terminal.
        .2 out.
            .3 pin.
                .4 implementación con librerías para actuar sobre los pines, Se buscará una librería que pueda ser configurada para actuar sobre pines de varias plataformas.
}
\normalsize







\subsubsection{Prestaciones}
    \paragraph{Funcionalidades del sistema}

El usuario el sistema tendrá la capacidad de:

\begin{itemize}
    \item Un programa gestor para introducir y gestionar tareas para ejecutar en sistemas remotos de forma automática o manual.
    \item Un programa cliente para instalar en cualquier servidor con la capacidad de recibir tareas desde el manager que se ejecutarán de forma automática.
    Cualquier programa instalado en dicho servidor podrá ser ejecutado.
    \item Un programa de control instalable en cualquiera de los servidores clientes para actuar sobre un pid que controle motor de corriente continua.
\end{itemize}

\paragraph{Garantías de seguridad y calidad}\label{par:testing}
    Conforme a los principios teóricos establecidos para el diseño de tests la cobertura de este proyecto se centrará en hacer tests unitarios para el dominio.
    En la~\cref{fig:piramidTest} se muestran las distribuciones de test que representan el ideal a la derecha y a el que tienen los sistemas a la izquierda.
    Las pruebas unitarias sólo son posibles cuando hay un buen diseño del software.
    En los proyectos que no disponen de un diseño que posibilite hacer test unitarios se ven abocados a hacer uso de tests que prueban demasiado código y características al mismo tiempo.
    Esto provoca que sean tests que tienen a fallar rápidamente y deben cambiarse habitualmente ante cualquier pequeño cambio.
    Las pruebas de más alto nivel incluyen incluso las interfaces de usuario.
    Debieran estar destinadas a funcionalidades muy establecidas y cuya tendencia al cambio sea muy escasa.

    \begin{figure}[H]
        \centering
        \includegraphics[height=0.35\textheight]{./part/Proyecto_ejecutivo/memoria_descriptiva/prestaciones/TestPiramid}
        \caption{Piramides de distribución de pruegbas: patrón y antipatrón}\label{fig:piramidTest}
    \end{figure}

\paragraph{Garantías de robustez ante nuevos desarrollo}

Uno de los puntos más relevantes de un proyecto de software es garantizar la viabilidad en la continuidad del desarrollo futuro; ya sea por mantenimiento o adición de nuevas características. A lo largo de la vida de un proyecto los técnicos que trabajan en él van cambiando; la acumulación de conocimiento acerca del sistema es un factor crítico. Debemos diseñar un sistema modular que evite tener que conocer todo el proyecto en profundidad para realizar cambios.

La infraestructura es un punto problemático al afectar a todo el sistema. Necesitamos desplegar una réplica del sistema en una máquina para probar y desarrollar las nuevas características, es decir, levantar una réplica en modo desarrollo. Instalar dependencias, garantizar que todo está funcional y que estamos probando un sistema exactamente igual al que están utilzando los clientes para evitar diferencias a la hora de desplegar los cambios en el entorno final.

Las herramientas que desplieguen el sistema de forma automatizada en cualquier dispositivo permitiendo trabajar en un punto concreto de la aplicación sin necesidad de conocerla en todo su extensión son: Docker, scripts y Makefiles.

Dentro de esos contendores Docker dispondremos de todas las herramientas ya instaladas para manejar el sistema sin tener que instalarlas en nuestra propia máquina, evitando problemas de versiones e incompatibilidades. Para este proyecto necesitaremos garantizar:

\begin{itemize}
    \item versión fija de golang para compilar los programas
    \item versión fija de la las librerias
    \item versión fija de la base de datos
    \item configuración del protocolo de comunicación entre sistemas
\end{itemize}

Como este proyecto se basa en 3 softwares entregables: el servicio de manager, el servicio cliente(el cual puede ser requerido en multiples máquinas y habrá de ser desplegada dicha réplica) y el programa de control. Vamos a extraer toda la utilidad de Docker para que se entrege y despliege de forma consistente y sencilla dicho software.

Los archivos Makefile son un conjunto de acciones expresadas en un archivo que nos permiten ejecutarlos desde consola. Con ellos podemos avisar al desarrollador de qué acciones básicas dispone, el simple conocimiento de su existencia ya supondría adquirir un conocimiento y de esta forma queda documentado a modo de ejecutable. Y queda descrito lo que realizan sin necesidad de conocer qué es lo que se ejecuta, simplemente su utilidad. En cada programa dispondremos de las siguientes acciones básicas:

\begin{itemize}
    \item interactuar con el sistema de contenedores docker: levantarlo, pararlo, resetearlo. Sin necesidad de conocer Docker.
    \item interactuar con el protocolo gRPC: recrear, generar y eliminar los archivos proto que definen el protocolo de comunicación. Sin necesidad de conocer gRPC
    \item interactuar con la base de datos: crear la base de datos, resetearla, indr
\end{itemize}







