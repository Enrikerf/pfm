\subsubsection{lenguajes}
    

\textbf{Golang}


Uno de los puntos centrales de este trabajo es exponer el estado del arte en el uso de Golang para el entorno industrial. En dicho estudio se ha consultado distintos papers y publicaciones para extraer primeras conclusiones:

Uno de los puntos más importantes para el desarrollo de software industrial es los recursos consumidos una de las primeras conclusines extraidas de la eficiencia del uso de golang interaccionando con mysql, una de las combinaciones más comerciales concluye que: "... combination of Go and MySQL is superior regarding CPU utilization and memory usage, while Node.js and MySQL combination is superior regarding response time~\cite{Effendy20211955}".

Un estudio más centrado en el uso de recursos ante problemas de algorítmia con altos requerimientos, en particular la implementacion de un árbol de decisiones. ~\cite{Dymora20201} Donde encontramos un punto importante: golang vuelve a no dar ventajas, pero tampoco inconvenientes en materia de tiempos de ejecucion. pero si que pierde en uso de cpu claramente y empata en materia de uso de memoria para mas de 500K registros en este problema en particular. Aunque se admite que la optimización de dicho mecanismo para este problema. lo cual es posible. "Thus, the Go language garbage collector
supports programmers by automatically releasing their programs’ memory when it is no longer needed.
However, tracking and cleaning the memory requires additional resources such as CPU time. The effect
of this can be seen in Figure 5. Of course, the scope of optimizing"~\cite{Dymora20201}

lo que si nos permite extraer es una conclusion y es que para usos intensivos golang es una opción viable pero no da ventaja en este aspecto.

Esto se debe al mecanismo que le da una ventaja tan notoria en el uso menos intensivo: el garbage collector que le requiere un uso adicional de memoria y cpu para ejecuciones con un gran número de registros. Concluye este estudio diciendo: ". Go can be an
attractive alternative in the area of DevOps tools. It is attractive to build something small–medium
that works natively without using a lot of RAM and which runs fast with many things needed for
this task in the language itself"~\cite{Dymora20201}

\begin{figure}[H]
	\centering
	\includegraphics[height=0.3\textheight]{./part/Proyecto_ejecutivo/memoria_constructiva/golang/img/memory_usage}
	\includegraphics[height=0.3\textheight]{./part/Proyecto_ejecutivo/memoria_constructiva/golang/img/cpuUsage}
	\includegraphics[height=0.3\textheight]{./part/Proyecto_ejecutivo/memoria_constructiva/golang/img/compTime}
	\caption[performance golang on algorithm]{performance golang on algorithm\cite{Dymora20201}}\label{fig:performance golang}
\end{figure}

El verdadero motivo para optar por golang como lenguaje para un software viene más de la mano de la facilidad de mantenimiendo, la rapidez de compilacion, el manejo fácil para la concurrencia y eficiente en el uso de recursos. Implementar mecanismos de memoria compartida para procesos concurrente es donde golang si optimiza recursos. Uno de los trabajadores de google que ha contribuido al ecosistema de go nos dice "Everyone knows and thinks about Google in terms of scale of users and scale of servers, but one thing that's not talked about as often is the scale of engineering effort."~\cite{Meyerson2014104+101}

Esto quiere decir que en la mayoría de las compañías el mayor coste no es el de infraestructura, si no el de ingeniería. Es un tradeoff muy importante el reducir el numero de horas dedicadas a mantener y desarrollar software que el hecho de optimizar el uso de máquinas. Si encima nos encontramos con un problema que no requiere de uso intensivo en cuestión algorítmica o tiempo de respuesta encontramos lo mejor de los dos: tenemos la ventaja de reducir los recuros necesarios de infraestructura y los de ingeniería. Si se diera el caso de que el software se enfrenta con el tiempo a problemas de escala siempre nos quedará la opción de aumentar las prestaciones de las máquinas utilizadas pero no tener problemas de aumento de costes de ingeniería

Es en la implementación de algoritmos que sacan ventaja de la concurrencia donde golang puede sacar ventaja respecto a otros lenguages como java o python ~\cite{Jenkins201714}

Estas son las principales conclusiones extraidas de la lectura de las publicaciones consultadas.
\cite{Effendy20211955}
\cite{Dymora20201}
\cite{Meyerson2014104+101}
\cite{Ray202110857}
\cite{Jenkins201714}
\cite{Ding2021321}
\cite{Taheri2021138}
\cite{NoAuthor2021179}
\cite{Dilley2019377}
\cite{Qiu2018}
\cite{Shoumik20181}
\cite{Mladenovic2018}
\cite{Benedict2017437}
\cite{Irawan2017}
\cite{Samaniego2017116}
\cite{Khaitan20152909}
\cite{Leokhin2015656}
\cite{Komendantskaya2014121}
\cite{Mittal2014292}
\cite{WhiteheadII2011209}
Nos permiten concluir que está justificado el uso en nuestro objetivo de diseñar un sistema de las carácteristicas de este proyecto con Golang.

Concluyo mencionando \cite{WhiteheadII2011209}"Despite the relatively young age of the programming language, we believe that Go helps
to fill an interesting niche in the field of programming languages. The unique feature-set and
aims of the language make it worth investigation for systems-level concurrent programming." Habiendo madurado el lenguaje unos años desde la publicación de este artículo encontramos que sigue siendo interesante, está altamente valorado en la comunidad.


\subsubsection{comunicaciones}\label{subsubsec:communications}
    
La interacción entre los distintos servicios que se desarrollan en este proyecto se realizarán mediante una \gls{API}.
Entre los dos paradigmas más utilizados a la hora de crear APIs se encuentran RPC y REST\@.
Se ha realizado un estudio consultando el estado del arte de ambos para justificar el uso de RPC\@.

\textbf{GRPC}

Según el paper original que describe el paradigma RPC “Remote procedure calls (RPC) appear to be a useful paradigm for providing communication across a
network between programs written in a high-level language.“~\cite{Birrell198439}.
RPC permite ejecutar una llamada a un servicio en un servidor remoto mediante formularios predefinidos, obteniendo respuestas con el mismo formato.
El estilo del servidor que realiza la llamada, no se tiene en cuenta por diseño.
Lo cual permite abstraerse de la implementación.

gRPC (Google Remote Procedure Call~\cite{grpc}) es un subtipo del diseño RPC\@.
Es una arquitectura global de alto rendimiento y de código abierto que garantiza la flexibilidad y la velocidad de la arquitectura de microservicios.
Dispone de un \gls{IDL} (interface definition language) empleando Protocol Buffers para definir la interfaz del servicio y el formato de los mensajes de carga útil.
Suministra un mecanismo para traducirlo a diversos lenguajes de desarrollo.
Con un único lenguaje se obtiene la capacidad de describir interfaces sin implicarse en la implementación.

RPC utiliza el estándar HTTP/2, pero ni el servidor ni el programador de la API tienen conocimiento de HTTP. Como resultado, la complejidad disminuye porque no hay que preocuparse por cómo se traducen los principios de RPC a HTTP. gRPC acelera la transferencia de datos entre microservicios, tanto en tiempo de desarrollo como de ejecución.

\textbf{REST}

\gls{REST} (Representational State Transfer) es un paradigma cliente-servidor de comunicación donde se comunican a través de mensajes codificados en formato JSON o XML o compatibles.
En su disertación original del año 2000, Roy T. Fielding define la interacción REST como: “\ldots The server will respond with the representation of a resource (today, it will most often be an HTML, XML or JSON document) and that resource will contain hypermedia links that can be followed to make the state of the system change.
Any such request will in turn receive the representation of a resource\ldots“ ~\cite{FieldingRoyThomas2000Asat}.
En la misma disertación enumera los principios que ha de seguir el estándar:

\begin{itemize}
    \item Arquitectura cliente-servidor.
    Hace hincapié en la separación de responsabilidades
    \item Ausencia de estado.
    El estado se guarda y mantiene en el cliente y no en el servidor
    \item Todas las solicitudes deben declarar si son o no cacheables
    \item Interfaz uniforme
    \item Sistema por capas.
    Tiene relación con la separación de responsabilidades.
\end{itemize}

Cada componente que combina el sistema de microservicios puede mostrarse y hacerse accesible públicamente al usuario o al cliente como un recurso de la API REST. Este recurso puede consultarse mediante los comandos HTTP GET, POST, PUT y DELETE. el usuario envía una consulta a una URL (Uniform Resource Locator) que provoca una respuesta con una carga útil en JSON, XML o cualquier formato de datos compatible.
Esta carga útil representa el recurso que desea el usuario.
Las peticiones comunes de los clientes incluyen:

\begin{itemize}
    \item Un método HTTP que especifica lo que debe procesarse en el recurso.
    \item La ruta del recurso.
    \item La cabecera que contiene datos sobre la consulta.
    \item Una carga útil de mensaje específica del cliente.
\end{itemize}

El servidor de la API envía una cabecera de tipo de contenido que identifica el formato de entrega del mensaje empleado en el cuerpo de la respuesta junto con la carga útil de datos que entrega al usuario que realiza la consulta.
También se incluye en el cuerpo de la respuesta un código de respuesta que informa al usuario del estado del resultado de la llamada a la API\@.

\textbf{HTTP 1.1 frente a HTTP/2}

El protocolo HTTP/2 para transmitir mensajes permite flujos multiplexados y una comunicación bidireccional.
gRPC admite varios tipos de interacciones, incluyendo interacciones de tipo \textit{unary, server streaming, client streaming y bidirectional}.
En contraste, REST utiliza el modelo de petición-respuesta de HTTP 1.1 que utiliza un método de handshaking TCP para cada consulta, lo que puede causar problemas de latencia y no aprovechar las ventajas de HTTP/2.
En general, gRPC puede proporcionar una transmisión de datos más rápida y eficiente y una comunicación más flexible y bidireccional en comparación con REST\@.

\textbf{La estructura de datos de la carga útil}

gRPC utiliza Protocol Buffers para serializar y deserializar datos, lo que permite una transmisión más rápida y eficiente de información.
Este método es más ligero, ya que los mensajes son una estructura más comprimida.
Están en formato binario, lo que hace que el procesamiento sea menos intensivo en CPU.
En el intercambio de datos, serializa y deserializa la información de forma automática.

REST utiliza principalmente JSON o XML para enviar y recibir información.
La facilidad de lectura humana de JSON es una ventaja de REST, pero no es tan rápido o ligero.
Esto se debe al requerimiento de que JSON debe ser serializado y traducido a los lenguajes de programación utilizados en ambos extremos.
Este paso adicional en el proceso de transmisión de datos puede afectar la eficiencia y aumentar la probabilidad de errores.


\textbf{Compatibilidad con los navegadores}\label{GRPCcompatibilidadConNavegadores}

Dado que la mayor parte de la interacción de la API web se produce en línea, la compatibilidad con el navegador es una consideración clave en el debate entre gRPC vs.
REST. La compatibilidad con los navegadores es probablemente una de las principales ventajas de las APIs REST frente a gRPC. Todos los navegadores ofrecen compatibilidad completa con HTTP 1.1. Sin embargo, la compatibilidad HTTP/2 para gRPC los navegadores sigue siendo relativamente restringida.
En la web se necesita traducir la comunicación  HTTP/2 a HTTP 1.1. por lo se hace necesario establecer una capa proxy; añadiendo un elemento de complejidad y mantenimiento.
El soporte de HTTP/2 para los navegadores es algo que se encuentra todavía en desarrollo.

\textbf{Generación de código}

Para el IDL de gRPC existe un compilador llamado protoc que carga los archivos con la extensión de dicho lenguaje (.proto) y genera código nativo para comunicarse con los servicios remotos.
Están soportados múltiples lenguajes de programación.
En REST se debe emplear herramientas de terceros, como Postman, para la generación de código para las consultas de la API. La generación automática del código para la comunicación es especialmente ventajosa para los microservicios que combinan múltiples plataformas y lenguajes.
Facilita la construcción del kit de desarrollo de software (SDK) para cada lenguaje y facilita enfrentar cambios en la interfaz.

\textbf{Justificación de uso de gRPC}

Aunque actualmente la mayoría de las herramientas de terceros no ofrecen soporte gRPC, es una tecnología adecuada para la interacción entre sistemas internos, como microservicios.
Además, permite la inter-operatividad entre distintos lenguajes de programación, lo cual es otro de sus puntos fuertes.
Otro beneficio del uso de gRPC es la transmisión en stream de información sin necesidad de restablecer la comunicación para el envío de cada paquete de datos, lo cual es crítico si se quiere trabajar en sistemas que enfrenten el tiempo real.
El control de dispositivos mediante sistemas PID hace que esta característica sea esencial.
Las conexiones tanto de streaming, por parte del cliente como del servidor, permite establecer canales de comunicación versátiles según la situación a la que se enfrenta el sistema diseñado.
La monitorización de los dispositivos controlados hará necesaria la optimización en el uso del ancho de banda.
Aquí es donde gRPC sobresale, ya que proporciona una comunicación ligera, mayor eficiencia y rapidez.
En la figura~\cref{fig:gRPC vs REST} se resume la comparación entre los dos sistemas.

\begin{figure}[H]
    \centering
    \includegraphics[height=0.4\textheight]{./part/Proyecto_ejecutivo/memoria_constructiva/rpc/img/rpcComparison}
    \caption{GRPC vs REST.\cite{berga_santos_2023}}\label{fig:gRPC vs REST}
\end{figure}
\subsubsection{Infraestructura}\label{subsubsec:infraestructura}
\paragraph{Docker}\label{par:Docker}
    
Uno de los puntos más relevantes de un proyecto de software es garantizar la viabilidad en la continuidad del desarrollo futuro; ya sea por mantenimiento o adición de nuevas características. A lo largo de la vida de un proyecto los técnicos que trabajan en él van cambiando; la acumulación de conocimiento acerca del sistema es un factor crítico. Debemos diseñar un sistema modular que evite tener que conocer todo el proyecto en profundidad para realizar cambios.

La infraestructura es un punto problemático al afectar a todo el sistema. Necesitamos desplegar una réplica del sistema en una máquina para probar y desarrollar las nuevas características, es decir, levantar una réplica en modo desarrollo. Instalar dependencias, garantizar que todo está funcional y que estamos probando un sistema exactamente igual al que están utilzando los clientes para evitar diferencias a la hora de desplegar los cambios en el entorno final.

Las herramientas que desplieguen el sistema de forma automatizada en cualquier dispositivo permitiendo trabajar en un punto concreto de la aplicación sin necesidad de conocerla en todo su extensión son: Docker, scripts y Makefiles.

Dentro de esos contendores Docker dispondremos de todas las herramientas ya instaladas para manejar el sistema sin tener que instalarlas en nuestra propia máquina, evitando problemas de versiones e incompatibilidades. Para este proyecto necesitaremos garantizar:

\begin{itemize}
    \item versión fija de golang para compilar los programas
    \item versión fija de la las librerias
    \item versión fija de la base de datos
    \item configuración del protocolo de comunicación entre sistemas
\end{itemize}

Como este proyecto se basa en 3 softwares entregables: el servicio de manager, el servicio cliente(el cual puede ser requerido en multiples máquinas y habrá de ser desplegada dicha réplica) y el programa de control. Vamos a extraer toda la utilidad de Docker para que se entrege y despliege de forma consistente y sencilla dicho software.

Los archivos Makefile son un conjunto de acciones expresadas en un archivo que nos permiten ejecutarlos desde consola. Con ellos podemos avisar al desarrollador de qué acciones básicas dispone, el simple conocimiento de su existencia ya supondría adquirir un conocimiento y de esta forma queda documentado a modo de ejecutable. Y queda descrito lo que realizan sin necesidad de conocer qué es lo que se ejecuta, simplemente su utilidad. En cada programa dispondremos de las siguientes acciones básicas:

\begin{itemize}
    \item interactuar con el sistema de contenedores docker: levantarlo, pararlo, resetearlo. Sin necesidad de conocer Docker.
    \item interactuar con el protocolo gRPC: recrear, generar y eliminar los archivos proto que definen el protocolo de comunicación. Sin necesidad de conocer gRPC
    \item interactuar con la base de datos: crear la base de datos, resetearla, indr
\end{itemize}

\paragraph{Raspberry}
    
Necesitaremos una plataforma donde desplegar el programa de control que disponga de las características necesarias para interactuar con el hardware del  motor de corriente continua y la electrónica de potencia: el puente H. Las soluciones más comerciales son arduino y raspberry.
Vamos a poner a prueba la versatilidad de Golang para compilar el programa en distintas arquitecturas de chips de CPU de forma trasparente para el programador.
Nos decantamos por raspberry por poseer una interfaz más amigable de cara al desarrollo y las pruebas del programa.
Se entiende que el ámbito del proyecto ya es lo suficientemente complejo y arduino pudiera requerir mayor de esfuerzo debído a no disponer de sistema operativo a la hora de trabajar.

En la~\cref{fig:raspberry pins} se puede ver los pines disponibles para acceder desde el sistema operativo y en la~\cref{fig:Used Pins} hemos señalado los que va a utilizar nuestro diseño.

\begin{figure}[H]
    \centering
    \includegraphics[scale = 0.4]{part/Proyecto_ejecutivo/memoria_constructiva/raspb/img/raspberry}
    \caption{PinSet de Raspberry Pi 4.4\cite{raspberryORG} }\label{fig:raspberry pins}
\end{figure}

\begin{figure}[H]
    \centering
    \includegraphics[scale = 0.4]{part/Proyecto_ejecutivo/memoria_constructiva/raspb/img/gpio-pinout-raspberry-pi-01-used}
    \caption{Pins utilizados en el proyecto}\label{fig:Used Pins}
\end{figure}
\paragraph{Motor CC y puente H}\label{par:docker}
    
El motor seleccionado será el~\cite{EMG30datasheet} Por el conocimiento que se dispone de su uso. En la figura~\cref{fig:EMG Motor} podemos apreciar su aspecto. Para el objeto que nos ocupa cumple con los requerimientos de ser sencillo, barato y disponer de una calidad suficiente.

\begin{figure}[H]
    \centering
    \includegraphics[scale = 0.4]{part/Proyecto_ejecutivo/memoria_constructiva/motor/img/MotorEMG30}
    \caption{Motor EMG30\cite{EMG30datasheet}}\label{fig:EMG Motor}
\end{figure}

Los parámetros que definen sus características más relevantes pueden apreciarse en la Tabla~\ref{tab:EMG30specifications}. El código de colores de las conexiones de las que dispone el motor podemos encontralo en las especificaciones ténicas del mismo. Se extraen en la~\cref{fig:motor connection}


\begin{table}[H]
    \centering
    \begin{tabular}{|l|l|}
        \hline
        Característica & Valor\\
        \hline
        Voltaje nominal & 12 V\\
        \hline
        Torque nominal & 1.5 kg/cm\\
        \hline
        Velocidad nominal & 170 rpm\\
        \hline
        Intensidad nominal & 530 mA\\
        \hline
        Velocidad sin carga& 216 rpm\\
        \hline
        Intensidad sin carga& 150mA\\
        \hline
        Intensidad máxima& 2.5 A\\
        \hline
        Salidad nominal & 4.22 W\\
        \hline
        Pasos por vuelta del codificador& 360 \\
        \hline
        Razón de la reductora & 30:1\\
        \hline
    \end{tabular}
    \caption{Características EMG30 \\ Fuente:\cite{EMG30datasheet}}\label{tab:EMG30specifications}
\end{table}


\begin{figure}[H]
    \centering
    \includegraphics[scale = 0.6]{part/Proyecto_ejecutivo/memoria_constructiva/motor/img/motorConnection}
    \caption{Motor EMG30 \\Fuente: Documento de especificaciones técnicas.\cite{EMG30datasheet}}\label{fig:motor connection}
\end{figure}


    
El componente utilizado para ampliar la señal enviada por el Arduino al motor, sirviendo de intermediario entre éste y los motores, es el módulo de control PWM de motores con un circuito integrado LMD18200T que soporta hasta tres Amperios. Es un puente H controlado por PWM. Satisface los requisitos necesarios en nuestra tarea: Precio, voltaje, amperaje y de implementación sencilla. Se aprecian en la~\cref{fig:Hbridge} las entradas y las salidas de las que dispone. De arriba abajo y de izquierda a derecha disponemos de:

\begin{itemize}
	\item \textbf{GND:} La tierra de la placa de control, en nuestro caso el Arduino.
	\item \textbf{PWM(Pulse width modulation):} la señal del control por ancho de pulsos.
	\item \textbf{Dir:} Señal con el sentido de giro que se desea.
	\item \textbf{Brake:} Freno del motor
	\item \textbf{V+:} Alimentación de 12 voltios.
	\item \textbf{GND:} tierra.
	\item \textbf{OUT1:} Salida de una mitad del puente H.
	\item \textbf{OUT2:} Salida de la otra mitad del puente H
\end{itemize}

El esquema final de conexión entre raspberry, puente H y motor lo podemos contemplar en la~\cref{fig:conectionDiagram}\\

\begin{figure}[H]
		\centering
		\includegraphics[scale = 0.3]{part/Proyecto_ejecutivo/memoria_constructiva/motor/img/modulopot}
		\caption{Módulo de potencia}\label{fig:Hbridge}
\end{figure}

\begin{figure}[H]
		\centering
		\includegraphics[scale = 0.15]{part/Proyecto_ejecutivo/memoria_constructiva/motor/img/connection diagram}
		\caption{Esquema de conexión del módulo de potencia.\\Fuente: elaboración propia.}\label{fig:conectionDiagram}
\end{figure}
\paragraph{Mysql}\label{par:mysql}
    Para la base de datos escogeremos Mysql por ser también la solución más estandar y que más domino, para no tener que añadir mayor esfuerzo de investigación.
    \begin{figure}[H]
        \centering
        \includegraphics[scale = 0.15]{part/Proyecto_ejecutivo/memoria_constructiva/dbSchema}
        \caption{Esquema de la base de datos}\label{fig:Esquema de la base de datos}
    \end{figure}
\paragraph{CI/CD}\label{par:cicd}
    Archivo githubActions