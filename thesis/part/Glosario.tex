
\newglossaryentry{CRUD} {name={CRUD},description={C.R.U.D. or CRUD: es un acrónimo que se refiere a las cuatro operaciones básicas de la gestión de datos en un sistema de software: Crear, Leer (Read), Actualizar (Update) y Eliminar (Delete). Estas operaciones representan las tareas esenciales que cualquier sistema necesita para interactuar con una base de datos o cualquier otra fuente de almacenamiento de datos.}}
\newglossaryentry{Backend}{name={Backend} backend: El backend (o back-end) se refiere a la capa de un sistema de software que se encarga de la lógica de negocio, la gestión de datos y la comunicación con otras aplicaciones o sistemas externos. En términos técnicos, el backend consiste en un conjunto de servidores, aplicaciones y servicios que se ejecutan en segundo plano y proporcionan una interfaz de programación de aplicaciones (API) para que los clientes (front-end o aplicaciones móviles) puedan interactuar con ellos.}
\newglossaryentry{Manager} {name={Manager},description={manager: nuestro backend de gestión de tareas para su ejecucion en sistemas remotos}}
\newglossaryentry{Client} {name={Client},description={client: nuestro backend de ejecucion de tareas para ser albergados en sistemas remotos}}
\newglossaryentry{LayerArchitecture} {name={LayerArchitecture},description={Arquitectura de capas: La arquitectura de capas, también conocida como arquitectura de capas o arquitectura en capas, es un enfoque de diseño de software en el que los componentes del sistema se dividen en capas lógicas y se comunican entre sí a través de interfaces definidas. Cada capa tiene una función específica y se comunica solo con la capa inmediatamente superior o inferior. La arquitectura de capas es una forma de estructurar un sistema de software para mejorar la modularidad, escalabilidad, mantenibilidad y flexibilidad del sistema. Cada capa se enfoca en una tarea específica y puede ser desarrollada, probada y mantenido por separado, lo que permite una mejor organización del código y una mayor facilidad de mantenimiento.}}
\newglossaryentry{HexagonalArchitecture} {name={HexagonalArchitecture},description={Arquitectura hexagonal: una implementación concreta de la arquitectura de capas La arquitectura hexagonal, también conocida como arquitectura de puertos y adaptadores, es un enfoque de diseño de software que tiene como objetivo separar la lógica de negocio del sistema de sus detalles técnicos de implementación. La arquitectura hexagonal se basa en la idea de que el núcleo del sistema es independiente de su entorno y se comunica con el mundo externo a través de puertos y adaptadores. Se centra en la creación de un núcleo de negocio fuerte y flexible, lo que permite una fácil adaptación del sistema a cambios futuros. Además, esta arquitectura también facilita las pruebas unitarias y la integración continua, ya que el núcleo del sistema se puede probar de forma independiente de los adaptadores y puertos.}}
\newglossaryentry{DDD} {name={DDD},description={DDD: (Domain-Driven Design o Diseño Dirigido por el Dominio) es un enfoque de diseño de software que se centra en el modelado del dominio de la aplicación, es decir, en las entidades, objetos, conceptos y relaciones que conforman el núcleo del negocio.\textit{DDD} se basa en la idea de que el software debe estar diseñado para reflejar y dar soporte al negocio que se está modelando.\textit{DDD} se enfoca en definir un lenguaje común (ubiquitous language) que se comparte entre los expertos del dominio de la aplicación y los desarrolladores de software. Este lenguaje común ayuda a asegurar que la implementación del software refleje con precisión el negocio que se está modelando.}}
\newglossaryentry{CQRS} {name={CQRS},description={CQRS significa Command Query Responsibility Segregation (Segregación de responsabilidades de comandos y consultas, en español). Es una técnica de diseño de arquitectura de software que separa la lógica de escritura (comandos) de la lógica de lectura (consultas) en sistemas de información.}}
\newglossaryentry{Entity} {name={Entity},description={Entity: Es un objeto dentro del modelo de dominio que tiene una identidad única y representa un concepto o una cosa del mundo real. Los objetos de entidad tienen atributos y comportamientos y pueden ser almacenados, recuperados y modificados en la base de datos.}}
\newglossaryentry{ValueObject} {name={ValueObject},description={\textit{Value Object}: Es un objeto que representa un valor o una descripción de un objeto, pero no tiene identidad propia. Los objetos de valor son inmutables, es decir, no cambian su estado después de ser creados, y se utilizan para representar datos que no necesitan ser almacenados individualmente en la base de datos.}}
\newglossaryentry{Service} {name={Service},description={Service: Es un objeto que proporciona funcionalidades y operaciones que no están directamente relacionadas con ninguna entidad o valor específico. Los servicios encapsulan la lógica de negocio compleja y se utilizan para ejecutar operaciones que involucran múltiples entidades o valores.}}
\newglossaryentry{Repository} {name={Repository},description={Repository: Es un objeto que proporciona una interfaz para interactuar con la base de datos y abstrae la lógica de almacenamiento de datos del resto de la aplicación. Los repositorios se utilizan para realizar operaciones de almacenamiento, recuperación, actualización y eliminación de entidades.}}
\newglossaryentry{InfrastructureLayer} {name={InfrastructureLayer},description={Infrastructure Layer: Es la capa de la aplicación que proporciona una interfaz para interactuar con los componentes de infraestructura, como bases de datos, sistemas de archivos, servicios web, etc. Esta capa es responsable de proporcionar una implementación concreta de los componentes de infraestructura, y es independiente del resto de la aplicación.}}
\newglossaryentry{Adapter} {name={Adapter},description={Adapter: Es un objeto que se utiliza para adaptar una interfaz a otra. En el contexto de la arquitectura hexagonal, los adaptadores se utilizan para conectar la capa de aplicación con los componentes de infraestructura.}}
\newglossaryentry{ApplicationLayer}{ Application Layer: Es la capa de la aplicación que contiene la lógica de la aplicación y los casos de uso específicos del negocio. Esta capa se comunica con la capa de infraestructura a través de adaptadores y proporciona una interfaz para interactuar con la aplicación.}
\newglossaryentry{UseCase} {name={ApplicationLayer},description={Use Case: Es un escenario específico de uso de la aplicación que involucra múltiples operaciones y procesos. Los casos de uso se utilizan para representar la funcionalidad de la aplicación desde la perspectiva del usuario.}}
\newglossaryentry{Command}{name={UseCase},description={ Command: Es una operación que realiza un cambio en el estado del sistema. Los comandos se utilizan para ejecutar acciones que modifican las entidades o valores del sistema.}}
\newglossaryentry{Query} {name={Command},description={Query: Es una operación que recupera datos del sistema sin modificar su estado. Las consultas se utilizan para obtener información sobre el estado actual de las entidades o valores del sistema.}}
\newglossaryentry{PID} {name={Query},description={PID: "Proporcional, Integral y Derivativo". Se trata de un tipo de controlador que se utiliza en sistemas de control automático para ajustar y mantener una variable de salida en un valor deseado.}}
\newglossaryentry{CI/CD} {name={PID},description={CI/CD es un conjunto de prácticas y herramientas que se utilizan para automatizar el proceso de desarrollo de software desde la integración del código hasta su implementación en producción.}}
