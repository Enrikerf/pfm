
El presente trabajo evalua el uso de Golang como lenguaje de programación enfocado a la gestión de proyectos comerciales. En particular enfocado a soluciones que requieran de la interacción con actuadores o hardware para el control automático.

Se ha llevado a cabo el diseño y ejecución de un proyecto que abarca las complejidades a las que se enfrenta un proyecto de desarrollo de software desde su concepción hasta su entrega. Si bien no es un objetivo tratar toda la profundidad que puede encarar cualquiera de los puntos, si cumple su objetivo de presentar y describir cada uno de ellos. Permitiendo evaluar los aspectos positivos y negativos de Golang para dicho fin. Se han puesto a prueba los conceptos clave que el lenguaje promueve: asincronía y gestión de concurrencia y simplicidad.

El sistema diseñado consta de un servidor que gestiona tareas que hemos llamado Manager. Dichas tareas son comandos que pueden ser ejecutados en servidores remotos que llamamos Clients. En particular uno de los comandos desarrollados ejecutará un programa de control PID que actua sobre un motor de corriente continua.

\begin{itemize}
    \item En el Manager se ha puesto a prueba la facilidad de implementar arquitecturas estándar para la gestión de datos, así como la comunicación con sistemas clientes a través de APIs. Se centra en la implementación de una arquitectura hexagonal, con el fin de evaluar los conceptos generales del lenguaje: gestión del protocolos de comunicación en tiempo real, enrutamiento, inyección de dependencias y estruccturación del código.
    \item En el software cliente, se ha puesto a prueba la versatilidad de Golang para ser compilado en diferentes plataformas sin la necesidad de una máquina virtual. Arquitecturas típicas de PC personal, como Intel, AMD o ARM para ejecutarlo en una Raspberry.
    \item  En la implementación del programa de control automático, se ha puesto a prueba la versatilidad del lenguaje para desarrollar algoritmos de forma limpia y mantenible.
\end{itemize}

El resultado final es un sistema que contiene los elementos típicos de una solución comercial: un sistema web API y servicios intercomunicados a través del mismo, un frontal y servicios clientes.