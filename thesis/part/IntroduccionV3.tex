
El propósito de este trabajo es evaluar el uso de Golang como lenguaje de programación en un entorno industrial. Para ello, se llevará a cabo un proyecto con suficiente complejidad, que permita evaluar los aspectos positivos y negativos de la implementación de Golang en sistemas empresariales. Se pondrán a prueba diferentes conceptos clave que el lenguaje promueve, como la comunicación entre sistemas con distintos protocolos, la asincronía y la gestión de concurrencia, la facilidad de uso del ecosistema, documentación y comunidad del lenguaje, y la facilidad para mantener la simplicidad y la limpieza del código.

Se implementará un sistema que consta de un servicio central de control en un servidor, que enviará comandos a un servidor que los recibirá. En el servidor receptor, se ejecutará un programa de control PID que gestionará la respuesta a los comandos recibidos. En el sistema central de control se pondrá a prueba la facilidad de implementar arquitecturas estándar para la gestión de datos, así como la comunicación con sistemas clientes. El proyecto se centrará en la implementación de la arquitectura hexagonal, una de las arquitecturas más extendidas en sistemas empresariales, con el fin de evaluar los conceptos generales del lenguaje.

En el software cliente, se probará la capacidad de Golang para ser compilado en diferentes plataformas sin la necesidad de una máquina virtual. El objetivo es compilar el software en arquitecturas típicas de PC personal, como Intel, AMD o ARM para ejecutarlo en una Raspberry.

En la implementación del programa de control automático, se pondrá a prueba la versatilidad del lenguaje para desarrollar algoritmos de forma limpia y mantenible.

El resultado final del proyecto será un sistema que contenga los elementos típicos de una solución comercial: un sistema web API y servicios intercomunicados a través del mismo, un frontal y servicios clientes. Se utilizarán herramientas de diseño como la arquitectura de capas, DDD y testing para garantizar la posibilidad de cambio y mantenimiento en el futuro, y el cumplimiento de las expectativas del cliente.

En conclusión, este proyecto permitirá evaluar de forma exhaustiva el uso de Golang en sistemas empresariales y analizar los aspectos positivos y negativos de su implementación. Se espera obtener información valiosa que permita tomar decisiones informadas en cuanto a la elección de lenguajes de programación para sistemas empresariales.