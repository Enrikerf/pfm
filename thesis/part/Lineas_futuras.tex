\begin{itemize}
    \item crear un sistema de crontab para que las tareas automaticas se puedan ejecutar no según son creadas si no en un tiempo requerido y configurado. Esto es una tarea básica, pero crear un sistema de temporización parecido al CRONTAB de linux se escapaba del scope. Además el sistema de ejecución automática asíncrona y gestión por eventos deja muy localizada y de fácil acceso esta iteración. Habrá simplemente que añadir un string parecido al Crontab a las variables de las tareas y en el Looper tener en cuenta que solo se habrá de recoger de base de datos las tareas que cumplan con los requerimientos de tiempo además de los ya existentes
    \item la ejecución manual de las tareas si pudiera ser bidireccional cuando los navegadores lo permitan o investigar alternativas
    \item separación del programa cliente del programa de control
    \item Creación de programas de control variados que pongan a prueba el sistema
    \item test de flujo de carga
    \item mejora del hardware y que no esté en protoboard
    \item uso para el control del robot de mi pfg
    \item implementar el concepto de context nativo de golang para el control de timeouts de las gorutines.
\end{itemize}
