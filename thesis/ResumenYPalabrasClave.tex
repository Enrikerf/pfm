\textbf{Resumen}

Se ha diseñado y desarrollado un sistema capaz de gestionar y ejecutar tareas en servidores remotos.
Como ejemplo de aplicación remota se dispone de un programa de control PID para un motor de corriente continua.
Se ha obtenido un ejemplo práctico de aplicación de una metodología de desarrollo de software desde su diseño hasta su entrega.
Un diseño estructurado en capas, separando la solución tecnológica concreta de la lógica que resuelve el problema.
Se ha creado un lenguaje expresivo en el nombre y la estructura de todos los componentes del software transmitiendo el problema.
Dicho lenguaje facilita la comunicación efectiva del problema enfrentado a todo aquel que necesite interactuar con el sistema.
El diseño está enfocado en la gestión del principal problema de desarrollo del software: la evolución.
El sistema consta de tres programas independientes:

\begin{itemize}
    \item Un gestor de la interacción con los usuarios.
    \item Un cliente que responde a las ordenes de ejecución del gestor.
    Ejecutable en paralelo en tantas máquinas como se requiera.
    \item Un control PID para ser ejecutado por el programa cliente.
\end{itemize}

El trabajo evalúa Go ante un proyecto enfocado en la interacción con actuadores o hardware para el control automático.
Se han puesto a prueba los conceptos clave del lenguaje: asincronía y gestión de concurrencia y simplicidad.

Se ha realizado el ejercicio de adaptación de los documentos utilizados en un proyecto de ejecución de tipo industrial a un proyecto de software.
El diseño, en un documento de proyecto ejecutivo.
El desarrollo, en un documento de ejecución de obra.

\textbf{Palabras Clave}

Go,
API,
REST,
RPC,
CRUD,
control PID,
Arquitectura de capas,
Arquitectura hexagonal,
DDD,
CQRS,
lenguaje ubicuo,
CI/CD.

\newpage

\textbf{Summary}

A system capable of managing and executing tasks in remote systems has been designed and developed.
As an example of a remote application, the system also consists of a PID control program for a DC motor.
A practical example of the application of a software development methodology from design to delivery has been obtained.
The design is structured in layers, separating the concrete technological solution from the logic that solves the problem, facilitating the iterative evolution of the software.
An expressive language has been created to convey the problem solved.
This language facilitates effective communication among all project participants.
The design is focused on effectively handling the main problem of software development: evolution.
The resulting system consists of three independent programs:

\begin{itemize}
    \item A user interaction manager.
    \item A client that responds to the manager's execution commands.
    Executable in parallel on as many machines as required.
    \item A PID control to be executed by the client program.
\end{itemize}

The work evaluates Go to face a project focused on interaction with actuators or hardware for automatic control.
The key concepts of the language have been tested: asynchrony and concurrency management and simplicity.

An exercise has been carried out to adapt the documents used in an industrial execution project to a software project.
The design in an executive project document.
The development in a work execution document.

\textbf{Key words}

Go,
API,
REST,
RPC,
CRUD,
PID control,
LayerArchitecture,
HexagonalArchitecture,
DDD,
CQRS,
Ubiquitous Language,
CI/CD.